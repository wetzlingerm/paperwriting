
\guideline[g:preliminaries:lookup]
    {Use the preliminaries as a dense look up for the main body.}

\goodbadexample[{\cite[Sec.~2.1 \& 3.3]{Wetzlinger2024ARCH1}}]{
    \location{Main Body} \\
    Given a random direction $\ell \in \mathbb{R}^r, \left\| \ell \right\|_2 = 1$, we compute the support vector $\widehat{y}(\ell)$ associated to the support function of the computed outer approximation $\widehat{\mathcal{Y}}([0,t_{\text{end}}])$, i.e.,
    \begin{equation*}
        \widehat{y}(\ell) = \rho \big( \widehat{\mathcal{Y}}([0,t_{\text{end}}]), \ell \big) = \max_{k \in \{0,\dots,\omega - 1\}} \rho \big( \widehat{\mathcal{Y}}(\tau_k) , \ell \big) ,
    \end{equation*}
    \highlightpart{where the support function in a direction $\ell \in \mathbb{R}^n$ is defined by $\rho(\mathcal{S},\ell) \coloneqq \max_{x \in \mathcal{S}} \ell^\top x$}.
}{
    \location{Preliminaries} \\
    \highlightpart{The support function in a direction $\ell \in \mathbb{R}^n$ is defined by $\rho(\mathcal{S},\ell) \coloneqq \max_{x \in \mathcal{S}} \ell^\top x$} (...). \\
    \location{Main Body} \\
    Given a random direction $\ell \in \mathbb{R}^r, \left\| \ell \right\|_2 = 1$, we compute the support vector $\widehat{y}(\ell)$ associated to the support function of the computed outer approximation $\widehat{\mathcal{Y}}([0,t_{\text{end}}])$, i.e.,
    \begin{equation*}
        \widehat{y}(\ell) = \rho \big( \widehat{\mathcal{Y}}([0,t_{\text{end}}]), \ell \big) = \max_{k \in \{0,\dots,\omega - 1\}} \rho \big( \widehat{\mathcal{Y}}(\tau_k) , \ell \big) .
    \end{equation*}
}

\noindent
Preliminaries should be written so that readers already familiar with the background theory or notation can easily skip them.
Avoid introducing essential concepts only at the point of first use, especially if they are referenced multiple times throughout the paper.
Even if a concept is used only once, it's generally better to separate prior knowledge from your own contributions by placing it in the preliminaries section.
