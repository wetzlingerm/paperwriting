
\guideline[g:preliminaries:problemstatement]
    {Introduce all necessary information before stating the problem.}

\goodbadexample[{\cite[Adapted from Sec.~2~\&~3]{Wetzlinger2023HSCC}}]{
    \textit{Problem 1. (Verification)}
    ~ Given an LTI system $\highlightpartmath{\dot{x} = Ax(t) + Bu(t)}$ where the initial state $x(0)$ and the input $u(t)$ are uncertain within the initial set $\mathcal{X}^0 \subset \mathbb{R}^n$ and the input set $\mathcal{U} \subset \mathbb{R}^m$, respectively, as well as a polytope $\mathcal{K} = \highlightpartmath{\{ z \in \mathbb{R}^r \, | Hz \leq f \}}$, decide whether the output set $\mathcal{Y}(t)$ obtained from the output equation $\highlightpartmath{y(t) = Cx(t) + Wv(t)}$ where $v(t) \in \mathcal{V} \subset \mathbb{R}^o$ stays within the safe set $\mathcal{K}$ at all times $t \in [0,t_\text{end}]$:
    \begin{equation*}
        \forall t \in [0, t_\text{end}]\colon \mathcal{Y}(t) \subseteq \mathcal{K} .
    \end{equation*}
    \badexpl{The highlighted parts are prerequisite definitions, which distracts from the actual problem statement.}
}{
    \location{Preliminaries} \\
    \textit{Definition 3. (Polytopes)} The halfspace representation of a polytope $\mathcal{P} \subset \mathbb{R}^n$ is (...):
    \begin{equation*}
        \mathcal{P} \coloneqq \{ x \in \mathbb{R}^n \, | \, Hx \leq f \}
    \end{equation*}
    (...) We use the shorthand notation $\mathcal{P} = \highlightpartmath{\langle H, f \rangle_H}$. \\
    \location{Problem Statement} \\
    We consider \highlightpart{LTI systems of the form}
    \begin{align*}
        \dot{x}(t) &= Ax(t) + Bu(t) , \tag{5a} \\
        y(t) &= Cx(t) + Wv(t) . \tag{5b}
    \end{align*}
    (...) The initial state $x(0)$, the input $u(t)$, and $v(t)$ are uncertain within the \highlightpart{initial set $\mathcal{X}^0 \subset \mathbb{R}^n$, the input set $\mathcal{U} \subset \mathbb{R}^m$, and the output uncertainty set $\mathcal{V} \subset \mathbb{R}^o$}, respectively. (...) \\
    (...) which extends to the \highlightpart{output set $\mathcal{Y}(t)$ obtained from a set-based evaluation of (5b)}. (...) \\
    \textit{Problem 1. (Verification)}
    ~ Given an \highlightpart{LTI system (5)} with \highlightpart{initial set $\mathcal{X}^0 \subset \mathbb{R}^n$, input set $\mathcal{U} \subset \mathbb{R}^m$, and output uncertainty set $\mathcal{V} \subset \mathbb{R}^o$}, as well as a safe set $\mathcal{K} = \highlightpartmath{\langle H, f \rangle_H}$, decide whether
    \begin{equation*}
        \forall t \in [0, t_\text{end}]\colon \mathcal{Y}(t) \subseteq \mathcal{K} ,
    \end{equation*}
    that is, whether the \highlightpart{output set $\mathcal{Y}(t)$} stays within the safe set $\mathcal{K}$ at all times $t \in [0,t_\text{end}]$.
}

\noindent
The problem statement should be concise and focused on the main question.
Present all necessary background, notation, and prerequisites beforehand—either in the preliminaries or immediately above the statement.
Avoid overloading the problem statement with contextual details; lay the groundwork first so the problem can be stated cleanly.
