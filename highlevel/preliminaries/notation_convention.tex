
\guideline[g:preliminaries:notation_convention]
    {Ensure your notation follows standard conventions.}

\goodbadexample[{\cite[Sec.~II.B.]{Wetzlinger2024CSL}}]{
    We consider autonomous nonlinear continuous-time systems
    \begin{equation*}
        \highlightpartmath{\dot{s}}(t) = \highlightpartmath{g}\big(\highlightpartmath{s}(t)\big) ,
    \end{equation*}
    where $\highlightpartmath{s} \in \mathbb{R}^n$ is the state vector and $\highlightpartmath{g}\colon \mathbb{R}^n \to \mathbb{R}^n$ is sufficiently smooth.
}{
    We consider autonomous nonlinear continuous-time systems
    \begin{equation*}
        \highlightpartmath{\dot{x}}(t) = \highlightpartmath{f}\big(\highlightpartmath{x}(t)\big) ,
    \end{equation*}
    where $\highlightpartmath{x} \in \mathbb{R}^n$ is the state vector and $\highlightpartmath{f}\colon \mathbb{R}^n \to \mathbb{R}^n$ is sufficiently smooth.
}

\noindent
While every notation is valid if properly introduced, some are more familiar to your audience than others.
Avoid unnecessary deviations from the existing conventions within your research topic.
This also helps readers in studying multiple papers concurrently, as their translation effort is minimized.
