
\guideline[g:preliminaries:subsections]
    {Organize different topics into separate subsections.}

\goodexample[{\cite[Sec.~II.]{Wetzlinger2025TAC}}]{
    II. Preliminaries \\
    We introduce some general notation, basics of set-based arithmetic, and fundamentals on forward reachability analysis required for the main body of this article. \\
    \highlightpart{\textit{A. Notations}} \\
    The set of real numbers is denoted by $\mathbb{R}$, the set of natural numbers without zero is denoted by $\mathbb{N}$, and the subset $\{a, a+1, \dots, b\} \subset \mathbb{N}$ for $0 < a < b$, is denoted by $\mathbb{N}_{[a,b]}$.
    (...) \\
    \highlightpart{\textit{B. Set-Based Arithmetic}} \\
    For convex sets $\mathcal{S}_1, \mathcal{S}_2 \subset \mathbb{R}^n$ as well as a matrix $M \in \mathbb{R}^{m \times n}$, we formally define (...) \\
    \highlightpart{\textit{C. Forward Reachable Set Computation}} \\
    For an LTI system of the form $\dot{x}(t) = Ax(t) + u(t)$, let the solution trajectory at time $t \in \mathbb{R}$ (...)
}

\noindent
Subsections provide a natural break between topics, removing the need for a smooth transition between paragraphs. This are especially useful when there isn't a clear or logical connection between topics.
A separate subsection for the notation is optional.
