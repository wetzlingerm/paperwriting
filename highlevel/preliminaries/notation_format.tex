
\guideline[g:preliminaries:notation_paragraph]
    {Start by introducing general notation.}

\goodexample[{\cite[Sec.~2]{Wetzlinger2024ARCH2}}]{
    \location{Beginning of Preliminaries}
    \highlightpart{Let us first introduce some notation:}
    Vectors are denoted by lowercase letters, matrices by uppercase letters, and sets by calligraphic letters.
    For a vector $v \in \mathbb{R}^n$, $v_(i)$ denotes its $i$th coordinate; for a matrix $M \in \mathbb{R}^{m \times n}$, we use $M_{(i,\cdot)}$ and $M_{(\cdot,j)}$ to denote the $i$th row and $j$th column, respectively.
    (...)
}

\noindent
It is impossible to foresee the many different audiences that your work may reach.
Since there likely exist conflicting conventions across those audiences, it is better to avoid making any assumptions on the readers' knowledge by explicitly introducing your notation.
While some notation can be considered school knowledge and thus be omitted, it is preferable to clarify specific notation than to open yourself up to a potential misunderstanding.
