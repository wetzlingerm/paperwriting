
\guideline[g:preliminaries:notation_paragraph]
    {Start by introducing general notation.}

\goodexample[{\cite[Sec.~2]{Wetzlinger2024ARCH2}}]{
    Let us first introduce some notation:
    Vectors are denoted by lowercase letters, matrices by uppercase letters, and sets by calligraphic letters.
    For a vector $v \in \mathbb{R}^n$, v(i) denotes its $i$th coordinate; for a matrix $M \in \mathbb{R}^{m \times n}$, we use $M_{(i,\cdot)}$ and $M_{(\cdot,j)}$ to denote the $i$th row and $j$th column, respectively.
    The transposes of a vector and matrix are written as $v^\top$ and $M^\top$, respectively.
    Comparisons between two vectors, such as $v_1 \leq v_2$, are evaluated element-wise.
    We use $\mathbf{0}$ and $\mathbf{1}$ for an all-zero or all-ones vector or matrix of proper size, and $I_n$ to denote the $n$-dimensional identity matrix.
    The canonical basis vector in the $i$th dimension is written as $e_i \in \mathbb{R}^n$.
    The closed Euclidean ball around the origin with radius $\varepsilon > 0$ is denoted by $B_\varepsilon$.
    For a matrix $M \in \mathbb{R}^{m \times n}$, $\mathrm{span}(M) \subseteq \mathbb{R}^m$ is the set $\{Mx | x \in \mathbb{R}^n\}$, and $\mathrm{span}(M)^\bot$ is its orthogonal complement, i.e., the set $\{y \in \mathbb{R}^m | \forall x \in \mathbb{R}^n \colon y\top Mx = 0\}$.
    The symbols $\top$ and $\bot$ denote true and false, respectively.
}

\noindent ...
% bad: not introduce notation at all and assume everyone understands
% no need to introduce notation that is known from school
% when in doubt, better to clarify than to open yourself up to ambiguity
