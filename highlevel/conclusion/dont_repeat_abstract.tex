
\guideline[g:conclusion:repeat_abstract]
    {Avoid repeating the abstract.}

\goodbadexample[{\cite[Abstract \& Sec.~VI.]{Wetzlinger2024CSL}}]{
    \location{Abstract} \\
    (...) we directly obtain sound inner approximations by using the Minkowski difference in a reachability algorithm for nonlinear systems. \\
    \location{Conclusion} \\
    We compute a sound inner approximation of the reachable set for autonomous nonlinear systems using the Minkowski difference \highlightempty{}.
}{
    \location{Abstract} \\
    (...) we directly obtain sound inner approximations by using the Minkowski difference in a reachability algorithm for nonlinear systems. \\
    \location{Conclusion} \\
    We compute a sound inner approximation of the reachable set for autonomous nonlinear systems using the Minkowski difference \highlightpart{between the reachable set due to the affine dynamics and the abstraction error due to the Lagrange remainder}.
}

\noindent
The abstract serves as an outlook on the paper, offering only a preliminary overview before the reader has full context.
In contrast, the conclusion should build upon the complete content, providing a deeper synthesis rather than merely repeating the abstract.
