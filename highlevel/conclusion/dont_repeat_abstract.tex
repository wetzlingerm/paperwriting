
\guideline[g:conclusion:repeat_abstract]
    {Avoid repeating the abstract.}

\goodbadexample[{\cite[Abstract \& Sec.~5]{Wetzlinger2024ARCH1}}]{
    \location{Abstract}
    (...), we propose a randomized generation of verification benchmarks in this paper.
    To this end, we leverage a reachability algorithm that can compute reachable sets with arbitrary precision. \\
    \location{Conclusion}
    After generating a random linear system \highlightempty{}, we compute the reachable set of known tightness with respect to the exact reachable set.
    \badexpl{The summary repeats the abstract almost verbatim, providing no additional information.}
}{
    \location{Abstract}
    (...), we propose a randomized generation of verification benchmarks in this paper.
    To this end, we leverage a reachability algorithm that can compute reachable sets with arbitrary precision. \\
    \location{Conclusion}
    After generating a random linear system from \highlightpart{user inputs, such as the state dimension or ranges for the eigenvalues of the state matrix}, we compute the reachable set of known tightness with respect to the exact reachable set.
    \goodexpl{The summary contains additional information not present in the abstract.}
}

\noindent
The abstract serves as an outlook on the paper, offering only a preliminary overview before the reader has full context.
In contrast, the conclusion should build upon the complete content, providing a deeper synthesis rather than merely repeating the abstract.
