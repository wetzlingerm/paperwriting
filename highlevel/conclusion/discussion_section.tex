
\guideline[g:conclusion:discussion_section]
    {Discuss your work in a broader frame of reference.}

\goodexample[{\cite[Sec.~VIII.]{Wetzlinger2023TAC}}]{
    Moreover, for systems where some states have no initial uncertainty and are not influenced by uncertain inputs, \highlightpart{our proposed algorithm cannot falsify the system} since the computed inner approximation of the reachable set will always be empty. (...)
    Fortunately, these \highlightpart{cases are easy to detect and one can use classical safety falsification techniques (...) instead}.
}

\noindent
A discussion helps readers understand how your approach fits into the existing body of work---not at the high-level conceptual view of the introduction, but in terms of specific properties such as assumptions, applicability, or overall quality of results.
This section is not about judging your method as better or worse, but about clarifying how its characteristics compare with those of related approaches.
Include a dedicated discussion section only if the material is substantial;
in shorter papers, it can be integrated into the results or conclusion.
