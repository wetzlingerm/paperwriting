
\guideline[g:universals:claims]
    {Give a reason for claims.}

\goodbadexample[{\cite[Sec.~IX.]{Wetzlinger2023TAC}}]{
    \highlightempty{} Our approach is competitive regarding the computation time, even for high-dimensional systems.
}{
    An \highlightpart{evaluation on benchmarks representing the current limits for the state-of-the-art reachability tools demonstrates} that our approach is competitive regarding the computation time, even for high-dimensional systems.
}

\goodbadexample[{\cite[Sec.~VI.B.]{Wetzlinger2025TAC}}]{
    For three compact, convex, and nonempty sets $\mathcal{S}_1, \mathcal{S}_2, \mathcal{S}_3 \subset \mathbb{R}^n$, we have
    \begin{equation*}
        \textsc{conv}(\mathcal{S}_1 \ominus \mathcal{S}_3, \mathcal{S}_2 \ominus \mathcal{S}_3) \subseteq
        \textsc{conv}(\mathcal{S}_1, \mathcal{S}_2) \ominus \mathcal{S}_3 .
    \end{equation*}
    \highlightempty{}
}{
    \textit{Lemma 1 (Distributivity of Minkowski difference over convex hull)}: For three compact, convex, and nonempty sets $\mathcal{S}_1, \mathcal{S}_2, \mathcal{S}_3 \subset \mathbb{R}^n$, we have
    \begin{equation*}
        \textsc{conv}(\mathcal{S}_1 \ominus \mathcal{S}_3, \mathcal{S}_2 \ominus \mathcal{S}_3) \subseteq
        \textsc{conv}(\mathcal{S}_1, \mathcal{S}_2) \ominus \mathcal{S}_3 .
    \end{equation*}
    \highlightpart{\textit{Proof}}: See appendix. \hfill $\square$
}

\goodbadexample[{\cite[Sec.~1]{Wetzlinger2021HSCC}}]{
    Since exact reachable sets can only be computed for a limited number of system classes \highlightempty{}, reachability algorithms compute over-approximations to establish soundness.
}{
    Since exact reachable sets can only be computed for a limited number of system classes \highlightpart{[36]}, reachability algorithms compute over-approximations to establish soundness.
}

\noindent
In scientific writing, every claim about a behavior or result should be supported by clear reasoning and never left without a solid basis.
Depending on the complexity of the statement, this justification may take the form of a brief explanation, a formal proof, or a reference to established work where the claim is substantiated.
