
\guideline[g:universals:introductory_sentences]
    {Use introductory sentences to outline the following content.}

\goodexample[{\cite[Sec.~IV.]{Wetzlinger2023TAC}}]{
    To achieve this, we first derive closed-form expressions describing how the individual errors depend on the values of each parameter \highlightpart{in Section IV-A}.
    Next, we present our automated parameter tuning algorithm \highlightpart{in Section IV-B} and prove its convergence \highlightpart{in Section IV-C}.
    Finally, we discuss further improvements to the algorithm \highlightpart{in Section IV-D} and describe the extension to output sets \highlightpart{in Section IV-E}.
}

\noindent
Introductory sentences at the beginning of a section help readers build a mental framework, making it easier to connect the details that follow to an overarching structure.
They are especially useful in longer sections; if there is a visible substructure, refer to all subsections.
Omit introductory sentences for shallow sections and when the section title already provides sufficient context.
