
\guideline[g:universals:flow]
    {Use appropriate transitions between paragraphs.}

\goodbadexample[{\cite[Sec.~III.B.]{Wetzlinger2024CSL}}]{
    We now derive our novel set-based computation for inner approximations of the time-point and time-interval reachable sets. \highlightempty{} \\
    \textit{Lemma 1}: Let (...) \\
    \textit{Proof}: We first (...) \hfill $\square$ \\
    \highlightempty{} \\
    \textit{Proposition 1 (Time-point reachable set)}: For (...) \\
    \textit{Proof}: Each successor state (...) \hfill $\square$ \\
    \highlightempty{} \\
    \textit{Proposition 2 (Time-interval reachable set)}: For (...) \\
    \textit{Proof}: Using (...) \hfill $\square$
}{
    We now derive our novel set-based computation for inner approximations of the time-point and time-interval reachable sets.
    \highlightpart{Let us first show} how the Minkowski difference can be used as an inner approximation. \\
    \textit{Lemma 1}: Let (...) \\
    \textit{Proof}: We first (...) \hfill $\square$ \\
    Crucially, \highlightpart{Lemma 1 enables us} to use the Minkowski difference for computing inner approximations. \\
    \textit{Proposition 1 (Time-point reachable set)}: For (...) \\
    \textit{Proof}: Each successor state (...) \hfill $\square$ \\
    \highlightpart{Next, we show} how to compute an inner approximation of the time-interval reachable set $\widecheck{\mathcal{R}}(\tau_0) \subseteq \mathcal{R}(\tau_0)$. \\
    \textit{Proposition 2 (Time-interval reachable set)}: For (...) \\
    \textit{Proof}: Using (...) \hfill $\square$
}

\noindent
Paragraphs form distinct units of thought but should still connect smoothly to maintain a coherent flow of presentation.
Use clear transitions---either explicit cues (as illustrated in the above example) or implicit links that refer back to the preceding topic in the opening sentence.
Avoid abrupt shifts where the relationship between consecutive paragraphs is unclear.
