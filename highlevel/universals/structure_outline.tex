
\guideline[g:universals:outline]
    {Before writing, outline the content using headings and keyphrases.}

\goodexample[{Adapted from \cite{Wetzlinger2024CSL}}]{
    \textit{I. Introduction}
    \begin{compactitemize}
        \item Motivation: Reachability analysis and inner approximations
        \item Related work: Approaches based on outer approximations, Hamilton-Jacobi reachability, dissipativity-based approaches
        \item Contributions: Summary of main idea, including figure
    \end{compactitemize}
    \textit{II. Preliminaries and Problem Statement} \\
    \textit{A. Notation and Set Operations}
    \begin{compactitemize}
        \item Vectors, matrices, sets, linear programs
        \item Center, box, convex hull, Minkowski sum/difference
    \end{compactitemize}
    \textit{B. Problem Statement}
    \begin{compactitemize}
        \item Autonomous nonlinear continuous-time systems
        \item Definition: Exact reachable set
        \item Goal: Compute an inner approximation
    \end{compactitemize}
    \textit{III. Reachability Analysis} \\
    \textit{A. Set-Based Integration}
    \begin{compactitemize}
        \item Taylor expansion with Lagrange remainder
        \item Integration of resulting linear differential inclusion
        \item Time-point and time-interval reachable set
    \end{compactitemize}
    \textit{B. Computation of an Inner Approximation}
    \begin{compactitemize}
        \item Lemma: Minkowski difference for special case
        \item Propositions: Inner approximation of time-point and time-interval reachable set
    \end{compactitemize}
    \textit{IV. Polynomial-Time Implementation} \\
    \textit{A. Set Representations and Operations}
    \begin{compactitemize}
        \item Definitions: Support function, polytope, (constrained) zonotope
        \item Time complexity of operations used in reachability algorithm
    \end{compactitemize}
    \textit{B. Reachability Algorithm}
    \begin{compactitemize}
        \item Full algorithm, walkthrough, time complexity
    \end{compactitemize}
    \textit{V. Numerical Examples}
    \begin{compactitemize}
        \item Table: Benchmarks and results comparison
        \item Discuss computation time and tightness
        \item Figure: Selected benchmark with tightness over time
    \end{compactitemize}
    \textit{VI. Conclusion}
    \begin{compactitemize}
        \item Summary
        \item Future work: Non-autonomous nonlinear systems
    \end{compactitemize}
}

\newpage

\noindent
Focus on the overall structure of the paper rather than specific details---there is no need to list references in the related work or write out equations.
Some content may already take shape as propositions, tables, or figures, which can be included in the outline.
A clear outline helps maintain focus on the main contributions and prevents getting lost in minor details.
Keep in mind that the outline is not fixed and will likely evolve as the writing progresses.
