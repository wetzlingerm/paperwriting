
\guideline[g:universals:layers]
    {Support layered reading.}

\goodexample[{\cite[Sec.~IV.]{Wetzlinger2024CSL}}]{
    \textit{IV. Polynomial-Time Implementation}
    \goodexpl{\textit{Note}: The example is graphically represented in Figure~\ref{fig:layers}.}
    The choice of set representations for computing an inner approximation with the formulae (20) and (23) determines the time complexity of the resulting reachability algorithm in the state dimension.
    Our proposed implementation achieves a polynomial time complexity, as detailed subsequently.
    \goodexpl{(Depth 0) The introductory paragraph limits the scope of the section by the terms ``set representations'', ``reachability algorithm'', and ``time complexity'', providing readers with a clear expectation for what will follow.}
    \textit{A. Set Representations and Operations}
    \goodexpl{(Depth 1) The title refers to the first term, ``set representations'', thus limiting the scope of this subsection.}
    Let us now introduce support functions, polytopes, and (constrained) zonotopes, as well as required operations.
    \goodexpl{(Depth 1) The introductory sentence lists three terms of equal depth, which we expect to follow subsequently, and hints at deeper elements.}
    \textit{Definition 2 (Support Function [21, Def.\ 1])}: The support function (...)
    \goodexpl{(Depth 2) As expected, the title of the definition ``support function'' matches the first element from the list above.}
    \textit{Definition 3 (Polytope [22, Sec.\ 1.1])}: A polytope is (...)
    \goodexpl{(Depth 2) The title of this next set representation definition matches the second element from the list.}
    The evaluation of the linear map $M\mathcal{P}$ of a polytope (...)
    \goodexpl{(Depth 3) The definition is followed by a paragraph on set operations.}
    \textit{Definition 4 (Constrained) Zonotope [24, Def.\ 1\&3]}: Using (...)
    \goodexpl{(Depth 2) By the title of the definition, we see that this element corresponds to the third element from the list at the beginning of the subsection.}
    We require the following set operations for zonotopes and constrained zonotopes: (...)
    \goodexpl{(Depth 3) Similarly to the second list element, we expand further on this subtopic.}
    \textit{B. Reachability Algorithm}
    \goodexpl{(Depth 1) We are back at a higher depth, now focusing on the two other terms---``reachability algorithm'' and ``time complexity''---from the introductory sentence to the section.}
    Algorithm 1 computes a sequence of inner approximations of the time-point and time-interval reachable sets derived in Propositions 1 and 2. (...)
    \goodexpl{(Depth 2) The reachability algorithm is introduced first.}
    In each time step $k \in \{0,...,\omega-1\}$, we first (...)
    \goodexpl{(Depth 3) A paragraph details the individual steps of the algorithm.}
    Algorithm 1 only uses set operations introduced in Section IV-A, the most time-consuming of which (...)
    \goodexpl{(Depth 2) Finally, the time complexity of the algorithm is discussed.}
}

\newpage

\noindent
Different readers approach a paper at different depths: some want the full story, others look only for the main ideas, and some search for technical details.
A well-structured paper supports all of them by making information easy to locate without requiring a front-to-back reading.
Typically, these layers of depth align with structural elements---sections, subsections, and paragraphs---each offering a progressively finer level of detail.
The principle is recursive: the whole paper should be skimmable at the section level, each section at the paragraph level, and each paragraph at the level of individual thoughts.


\begin{figure}[t]
    \centering
    
\begin{tikzpicture}[
    every node/.style={font=\footnotesize},
    stealtharrow/.style={->,>=stealth',shorten >= 1pt}
]
    % Do not exceed total width of 10
    % \draw[red] (-0.5,0.75) rectangle (9.5,-6.0);

    % intersection is at (0,0)
    \draw[stealtharrow] (-0.3,0) -- ++(9.5,0) node[pos=0.99,yshift=0.25cm,anchor=east] {Running text};
    \draw[stealtharrow] (0,0.2) -- ++(0,-5.75) node[pos=0.85,anchor=south,rotate=90] {Depth};

    \def\xoffset{0.4}
    \def\yoffset{-1.4}
    \def\dx{0.6}
    \def\dy{0.5}
    \foreach \y in { 0, 1, 2, 3 }{
        \draw[] (0.15,\yoffset - \y*\dy) -- ++(-0.3,0) node[left] {\y};
    }

    \begin{scope}[xshift=\xoffset cm,yshift=\yoffset cm]

    \filldraw[color_guideline] (0,0) circle(0.05cm);
    \draw[color_guideline,semithick]
        (0,0)           -- ++(\dx,0)    % depth 0: title + introductory sentence
        -- ++(0,-1*\dy) -- ++(\dx,0)    % depth 1: A. + introductory sentence
        -- ++(0,-1*\dy) -- ++(\dx,0)    % depth 2: support function
        -- ++(0,+0*\dy) -- ++(\dx,0)    % depth 2: polytopes
        -- ++(0,-1*\dy) -- ++(\dx,0)    % depth 3: polytopes -- set operations
           ++(0,+1*\dy) -- ++(\dx,0)    % depth 2: (constrained) zonotopes
        -- ++(0,-1*\dy) -- ++(\dx,0)    % depth 3: (constrained) zonotopes -- set operations
           ++(0,+2*\dy) -- ++(\dx,0)    % depth 1: B.
        -- ++(0,-1*\dy) -- ++(\dx,0)    % depth 2: reachability algorithm
        -- ++(0,-1*\dy) -- ++(\dx,0)    % depth 3: reachability algorithm -- steps
           ++(0,+1*\dy) -- ++(\dx,0);   % depth 2: time complexity

    \foreach \x/\y/\L in {
        2*\dx / -1*\dy / 5*\dx ,
        4*\dx / -2*\dy / 1*\dx ,
        9*\dx / -2*\dy / 1*\dx
    }{
        \draw[color_guideline,dashed] (\x,\y) -- ++(\L,0);
    }

    \draw[decorate,decoration={brace,amplitude=5pt}] (0*\dx,0.75) -- ++(11*\dx,0)
        node[midway,yshift=10pt]{Section IV.};
    \draw[decorate,decoration={brace,amplitude=5pt}] (1*\dx,0.15) -- ++(6*\dx,0)
        node[midway,yshift=10pt]{Section IV.A.};
    \draw[decorate,decoration={brace,amplitude=5pt}] (7*\dx,0.15) -- ++(4*\dx,0)
        node[midway,yshift=10pt]{Section IV.B.};

    \draw[stealtharrow] (2.5*\dx,-4*\dy) -- ++(0,2*\dy) node[pos=0,anchor=north] {Def.\ 2};
    \draw[stealtharrow] (3.5*\dx,-5*\dy) -- ++(0,3*\dy) node[pos=0,anchor=north] {Def.\ 3};
    \draw[stealtharrow] (4.5*\dx,-6.75*\dy) -- ++(0,3.75*\dy)
        node[pos=0,anchor=east,yshift=0.25*\dy cm] {Def.\ 3 (set operations)};
    \draw[stealtharrow] (5.5*\dx,-5*\dy) -- ++(0,3*\dy) node[pos=0,anchor=north] {Def.\ 4};
    \draw[stealtharrow] (6.5*\dx,-7.75*\dy) -- ++(0,4.75*\dy)
        node[pos=0,anchor=east,yshift=0.25*\dy cm] {Def.\ 4 (set operations)};

    \draw[stealtharrow] (8.5*\dx,-7.75*\dy) -- ++(0,5.75*\dy)
        node[pos=0,anchor=west,yshift=0.25*\dy cm,align=left] {Reachability algorithm};
    \draw[stealtharrow] (9.5*\dx,-6.25*\dy) -- ++(0,3.25*\dy)
        node[pos=0,anchor=west,yshift=0.25*\dy cm,align=left] {Reachability algorithm \\ (details)};
    \draw[stealtharrow] (10.5*\dx,-4.75*\dy) -- ++(0,2.75*\dy)
        node[pos=0,anchor=west,yshift=0.25*\dy cm,align=left] {Time complexity};

    % next chapter
    \filldraw[color_guideline] (11.5*\dx,0) circle(0.05cm);
    \draw[color_guideline,semithick] (11.5*\dx,0) -- ++(\dx,0);
    \draw[color_guideline,semithick,dotted] (12.5*\dx,0) -- ++(1.5*\dx,0);

    \draw[decorate,decoration={brace,amplitude=5pt}] (11.5*\dx,0.75) -- ++(1.5,0)
        node[midway,yshift=10pt]{Section V.};

    \end{scope}

\end{tikzpicture}

    \caption{Graphical representation of the example for Guideline \ref{g:universals:layers}.}
    % lower layers can be skipped, continue along the dashed line
    % allows for quickly finding necessary information if required
    % allows for reading a paper on a certain depth, e.g., skip details, focus on high-level idea
    \label{fig:layers}
\end{figure}

% non-layered:
% often shorter text because of lack of structural explanation (transition sentences and the like)
% since the structure is unclear, one must traverse the entire path to find some specific information
% heavier burden on what a reader must keep in mind when reading
% difficult to separate high-level ideas from details, since mixed together
