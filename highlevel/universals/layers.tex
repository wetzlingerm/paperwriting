
\guideline[g:universals:layers]
    {Support layered reading.}

% Use an except of a paper, show which parts correspond to which level of depth
\goodexample{
    ...
}

\noindent ...
% different types of readers: full read, read for main ideas, read for specific information
% applies recursively: sections in the entire paper, paragraphs in a section, thoughts in a paragraph

% Note: Axes must be equal in both subfigures
\begin{figure}[t]
    \centering

    \begin{subfigure}{0.99\textwidth}
        \captionsetup{aboveskip=2pt, belowskip=6pt}  % skip below for more separation to subfigure below
        \centering
        
\begin{tikzpicture}[
    every node/.style={font=\footnotesize},
    stealtharrow/.style={->,>=stealth',shorten >= 1pt}
]
    % Do not exceed total width of 10
    % \draw[red] (-0.5,0.75) rectangle (9.5,-6.0);

    % intersection is at (0,0)
    \draw[stealtharrow] (-0.3,0) -- ++(9.5,0) node[pos=0.99,yshift=0.25cm,anchor=east] {Running text};
    \draw[stealtharrow] (0,0.2) -- ++(0,-5.75) node[pos=0.85,anchor=south,rotate=90] {Depth};

    \def\xoffset{0.4}
    \def\yoffset{-1.4}
    \def\dx{0.6}
    \def\dy{0.5}
    \foreach \y in { 0, 1, 2, 3 }{
        \draw[] (0.15,\yoffset - \y*\dy) -- ++(-0.3,0) node[left] {\y};
    }

    \begin{scope}[xshift=\xoffset cm,yshift=\yoffset cm]

    \filldraw[color_guideline] (0,0) circle(0.05cm);
    \draw[color_guideline,semithick]
        (0,0)           -- ++(\dx,0)    % depth 0: title + introductory sentence
        -- ++(0,-1*\dy) -- ++(\dx,0)    % depth 1: A. + introductory sentence
        -- ++(0,-1*\dy) -- ++(\dx,0)    % depth 2: support function
        -- ++(0,+0*\dy) -- ++(\dx,0)    % depth 2: polytopes
        -- ++(0,-1*\dy) -- ++(\dx,0)    % depth 3: polytopes -- set operations
           ++(0,+1*\dy) -- ++(\dx,0)    % depth 2: (constrained) zonotopes
        -- ++(0,-1*\dy) -- ++(\dx,0)    % depth 3: (constrained) zonotopes -- set operations
           ++(0,+2*\dy) -- ++(\dx,0)    % depth 1: B.
        -- ++(0,-1*\dy) -- ++(\dx,0)    % depth 2: reachability algorithm
        -- ++(0,-1*\dy) -- ++(\dx,0)    % depth 3: reachability algorithm -- steps
           ++(0,+1*\dy) -- ++(\dx,0);   % depth 2: time complexity

    \foreach \x/\y/\L in {
        2*\dx / -1*\dy / 5*\dx ,
        4*\dx / -2*\dy / 1*\dx ,
        9*\dx / -2*\dy / 1*\dx
    }{
        \draw[color_guideline,dashed] (\x,\y) -- ++(\L,0);
    }

    \draw[decorate,decoration={brace,amplitude=5pt}] (0*\dx,0.75) -- ++(11*\dx,0)
        node[midway,yshift=10pt]{Section IV.};
    \draw[decorate,decoration={brace,amplitude=5pt}] (1*\dx,0.15) -- ++(6*\dx,0)
        node[midway,yshift=10pt]{Section IV.A.};
    \draw[decorate,decoration={brace,amplitude=5pt}] (7*\dx,0.15) -- ++(4*\dx,0)
        node[midway,yshift=10pt]{Section IV.B.};

    \draw[stealtharrow] (2.5*\dx,-4*\dy) -- ++(0,2*\dy) node[pos=0,anchor=north] {Def.\ 2};
    \draw[stealtharrow] (3.5*\dx,-5*\dy) -- ++(0,3*\dy) node[pos=0,anchor=north] {Def.\ 3};
    \draw[stealtharrow] (4.5*\dx,-6.75*\dy) -- ++(0,3.75*\dy)
        node[pos=0,anchor=east,yshift=0.25*\dy cm] {Def.\ 3 (set operations)};
    \draw[stealtharrow] (5.5*\dx,-5*\dy) -- ++(0,3*\dy) node[pos=0,anchor=north] {Def.\ 4};
    \draw[stealtharrow] (6.5*\dx,-7.75*\dy) -- ++(0,4.75*\dy)
        node[pos=0,anchor=east,yshift=0.25*\dy cm] {Def.\ 4 (set operations)};

    \draw[stealtharrow] (8.5*\dx,-7.75*\dy) -- ++(0,5.75*\dy)
        node[pos=0,anchor=west,yshift=0.25*\dy cm,align=left] {Reachability algorithm};
    \draw[stealtharrow] (9.5*\dx,-6.25*\dy) -- ++(0,3.25*\dy)
        node[pos=0,anchor=west,yshift=0.25*\dy cm,align=left] {Reachability algorithm \\ (details)};
    \draw[stealtharrow] (10.5*\dx,-4.75*\dy) -- ++(0,2.75*\dy)
        node[pos=0,anchor=west,yshift=0.25*\dy cm,align=left] {Time complexity};

    % next chapter
    \filldraw[color_guideline] (11.5*\dx,0) circle(0.05cm);
    \draw[color_guideline,semithick] (11.5*\dx,0) -- ++(\dx,0);
    \draw[color_guideline,semithick,dotted] (12.5*\dx,0) -- ++(1.5*\dx,0);

    \draw[decorate,decoration={brace,amplitude=5pt}] (11.5*\dx,0.75) -- ++(1.5,0)
        node[midway,yshift=10pt]{Section V.};

    \end{scope}

\end{tikzpicture}

        \caption{Structure supporting layered reading.}
        % lower layers can be skipped, continue along the dashed line
        % allows for quickly finding necessary information if required
        % allows for reading a paper on a certain depth, e.g., skip details, focus on high-level idea
        \label{fig:layers:layered}
    \end{subfigure}

    \begin{subfigure}{0.99\textwidth}
        \captionsetup{aboveskip=2pt, belowskip=0pt}
        \centering
        
\begin{tikzpicture}[
    every node/.style={font=\small}
]
    % Do not exceed total width of 10
    % \draw[red] (-0.5,0.75) rectangle (9.5,-3.0);

    % intersection is at (0,0)
    \draw[black,->,>=stealth'] (-0.3,0) -- ++(9.5,0) node[pos=0.99,yshift=0.25cm,anchor=east] {Running text};
    \draw[black,->,>=stealth'] (0,0.2) -- ++(0,-3) node[pos=0.8,anchor=south,rotate=90] {Depth};

    \begin{scope}[xshift=0.4cm,yshift=-0.4cm]
    \def\dx{0.6}
    \def\dy{0.5}

    % --- specific to bad case

    \filldraw[color_example_bad] (0,0) circle(0.05cm);
    \draw[color_example_bad,semithick]
        (0,0)           -- ++(\dx,0)
        -- ++(0,-1*\dy) -- ++(\dx,0)
        -- ++(0,-2*\dy) -- ++(\dx,0)
        -- ++(0,-1*\dy) -- ++(\dx,0)
        -- ++(0,+2*\dy) -- ++(2*\dx,0)
        -- ++(0,-1*\dy) -- ++(\dx,0)
        -- ++(0,+1*\dy) -- ++(\dx,0)
        -- ++(0,-2*\dy) -- ++(\dx,0)
        -- ++(0,+3*\dy) -- ++(\dx,0)
        -- ++(0,-1*\dy) -- ++(\dx,0);

    \filldraw[color_example_bad] (12*\dx,0) circle(0.05cm);
    \draw[color_example_bad,semithick] (12*\dx,0) -- ++(\dx,0);
    \draw[color_example_bad,semithick,dotted] (13*\dx,0) -- ++(0.5*\dx,0);

    \end{scope}

\end{tikzpicture}

        \caption{Structure not supporting layered reading.}
        % often shorter text because of lack of structural explanation (transition sentences and the like)
        % since the structure is unclear, one must traverse the entire path to find some specific information
        % heavier burden on what a reader must keep in mind when reading
        % difficult to separate high-level ideas from details, since mixed together
        \label{fig:layers:nonlayered}
    \end{subfigure}

    \caption{Different structures for the layers of depth of the running text.}
    \label{fig:layers}
\end{figure}
