
\guideline[g:mainbody:proofs_appendix]
    {Consider moving proofs to an appendix.}

\goodexample[{\cite[Proposition 3 \& Appendix]{Wetzlinger2025TAC}}]{
    \textit{Proposition 3 (Time-point AE backward reachable set)}:
    The backward reachable set $\mathcal{R}_{\forall\exists}(-t)$ defined in (32) can be computed by
    \begin{equation*}
        \mathcal{R}_{\forall\exists}(-t) = e^{-At} \big( (\mathcal{X}_{\text{end}} \oplus \mathcal{Z}_{\mathcal{}W}(t)) \ominus \mathcal{Z}_{\mathcal{U}}(t) \big) .
    \end{equation*}
    \highlightpart{\textit{Proof:} See Appendix.}
    \\
    (...)
    \highlightpart{\textit{Proof of Proposition 3:}} We have
    \begin{align*}
        &x_0 \in e^{-At} \big( (\mathcal{X}_{\text{end}} \oplus -\mathcal{Z}_{\mathcal{W}}(t) ) \ominus \mathcal{Z}_{\mathcal{U}}(t) \big) \\
        &\Leftrightarrow \forall z_u \in \mathcal{Z}_\mathcal{U}(t)\colon e^{At} x_0 + z_u \in \mathcal{X}_{\text{end}} \ominus -\mathcal{Z}_{\mathcal{W}}(t) \\
        &\Leftrightarrow \forall z_u \in \mathcal{Z}_\mathcal{U}(t) \exists z_w \in \mathcal{Z}_{\mathcal{W}}(t) \colon e^{At} x_0 + z_u + z_w \in \mathcal{X}_{\text{end}} \\
        &\Leftrightarrow \forall u(\cdot) \in \mathbb{U} \exists w(\cdot) \in \mathbb{W}\colon \xi(t; x_0, u(\cdot), w(\cdot)) \in \mathcal{X}_{\text{end}}
    \end{align*}
    which is equal to the definition in (32). \hfill $\square$
}

\noindent
Moving proofs to an appendix helps maintain the focus on the statement and its relation to the surrounding content.
In particular, technical proofs require a lot of attention from the reader, and such a deep dive may lead to losing sight of the bigger picture.
Still, if the proofs are few and simple, it is preferable to keep them in the main body.
Avoid keeping some in the main body and moving others to the appendix---make it an all-or-nothing decision.
