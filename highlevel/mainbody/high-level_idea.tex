
\guideline[g:mainbody:highlevel_idea]
    {For complex derivations, begin with a separate paragraph outlining the high-level idea.}

\goodexample[{\cite[Sec.~V.B.1)]{Wetzlinger2025TAC}}]{
    Let us first \highlightpart{introduce our high-level idea} for computing an outer approximation of (43):
    Note that the intersection of any number of $\mathcal{R}_\exists(-\tau; u^*(\cdot))$ in (43) always leads to a sound outer approximation.
    Obviously, we want a tight outer approximation, that is, a small intersection stemming from a well selected, finite number of input trajectories $u^*(\cdot) \in \mathbb{U}$.
    For this selection, we use a heuristic approach via support function reachability, which simplifies the intersection of the individual reachable sets in (43).
}

\noindent
Some derivations in the main body extend across multiple paragraphs and cannot be easily separated into standalone steps.
To avoid overloading the reader's memory with detailed information that only becomes meaningful at the end, introduce the high-level idea first.
This allows the reader to connect the upcoming details to a clear overarching concept, making the progression easier to follow and understand.
