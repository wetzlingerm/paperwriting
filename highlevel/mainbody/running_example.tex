
\guideline[g:mainbody:example]
    {Consider a short example to clarify hard-to-follow notation.}

\goodexample[{Adapted from \cite[Definition~4]{Wetzlinger2023TAC}}]{
    Given a zonotope $\mathcal{Z} = \langle c, G \rangle_Z \subset \mathbb{R}^n$ and a desired zonotope order $\rho^* \geq 1$, the operation $\textsc{reduce}(\mathcal{Z},\rho^*) \supseteq \mathcal{Z}$ returns an enclosing zonotope with order smaller or equal to $\rho^*$,
    \begin{align*}
        \textsc{reduce}(\mathcal{Z},\rho^*) &= \langle c, [G_{\text{keep}} ~ G_{\text{red}}] \rangle_Z , \\
        G_{\text{keep}} &= [G_{(\cdot,\pi_{\chi+1})} \dots G_{(\cdot,\pi_{\gamma})}], \\
        G_{\text{red}} &= \textsc{diag} \bigg( \sum_{i=1}^\chi |G_{(\cdot,\pi_i)}| \bigg) ,
    \end{align*}
    where $\pi_1, \dots, \pi_\gamma$ are the indices of the sorted generators
    \begin{equation*}
        \lVert G_{(\cdot,\pi_1)} \rVert_1 - \lVert G_{(\cdot,\pi_1)} \rVert_\infty
        \leq \dots \leq
        \lVert G_{(\cdot,\pi_\gamma)} \rVert_1 - \lVert G_{(\cdot,\pi_\gamma)} \rVert_\infty
    \end{equation*}
    and $\chi = \gamma - \lfloor (\rho^*-1)n \rfloor$ is the number of reduced generators. \\
    \highlightpart{\textit{Example}:}
    Consider the zonotope $\mathcal{Z} = \langle c, G \rangle \subset \mathbb{R}^2$ with
    \begin{equation*}
        c = \begin{pmatrix} 0 \\ 0 \end{pmatrix}, \quad
        G = \begin{pmatrix}
                1  & 2 & 5  & 3 & -3 \\
                -6 & 3 & -4 & 0 &  5
            \end{pmatrix}
    \end{equation*}
    and the desired zonotope order $\rho^* = 2$.
    Hence, we have the ordering $\pi = [4, 1, 2, 5, 3]$ and $\chi = 3$ so that
    $G_{\text{keep}} = \left(\begin{smallmatrix} 1 & 3 \\ -6 & 0 \end{smallmatrix}\right),
    G_{\text{red}} = \left(\begin{smallmatrix} 10 & 0 \\ 0 & 12 \end{smallmatrix}\right)$
    and
    {\setlength{\abovedisplayskip}{6pt}
    \begin{equation*}
        \textsc{reduce}(\mathcal{Z},\rho^*) = \bigg\langle
            \begin{pmatrix} 0 \\ 0 \end{pmatrix},
            \begin{pmatrix}
                1  & 3 & 10 & 0 \\
                -6 & 0 & 0 & 12
            \end{pmatrix} \bigg\rangle_Z
        \supseteq \mathcal{Z} .
    \end{equation*}
    }
}

\noindent
When a particular bit of notation is cumbersome but not inherently difficult, include a short and concise example to aid comprehension.
