
\guideline[g:mainbody:figure]
    {Consider a figure for explaining core ideas or workflows.}

\goodexample[{\cite[Adapted from Sec. 3.3 \& Fig.~1]{Wetzlinger2024ARCH1}}]{
    Our \highlightpart{main idea consists in generating sets}, e.g., using (9)-(10) and re-positioning the center afterward. The \highlightpart{procedure is illustrated in Figure 1}: (...)
    
    \begin{tikzpicture}[scale=0.85]

\definecolor{unsafe}{RGB}{255,140,140}			% HSV: 360, 45, 100
\definecolor{unsafeborder}{RGB}{230,34,34}		% HSV: 360, 85, 90
\definecolor{reachouter}{RGB}{140,190,255}		% HSV: 214, 45, 100
\definecolor{reachouterborder}{RGB}{34,119,230}	% HSV: 214, 85, 90

\useasboundingbox (-5.5,-2.65) rectangle (5.9,3.1);
% \useasboundingbox (-8.5,-2.65) rectangle (5.9,3.1);

% axes
\draw[thin,gray,step=1] (-4.25,-2.25) grid (5.25,2.25);
\draw[->,>=stealth'] (-4.5,0) -- (5.5,0) node[above] {$y_1$};
\draw[->,>=stealth'] (0,-2.5) -- (0,2.5) node[above] {$y_2$};
\filldraw[] (0,0) circle(1pt);

\begin{axis}[at={(-4cm,-2cm)},
	scale only axis,
	width=9cm,height=4cm,
	xtick=\empty,ytick=\empty,
	axis line style={draw=none},
	xmin=-4,xmax=5,ymin=-2,ymax=2]
	
	\addplot[semithick,draw=reachouterborder,fill=reachouter] table[row sep=crcr] {
		3.605820 1.155740 \\ 
		3.605820 1.155740 \\ 
		3.424280 1.201030 \\ 
		3.422510 1.201700 \\ 
		3.262660 1.238200 \\ 
		3.260840 1.238830 \\ 
		3.098860 1.272570 \\ 
		3.096990 1.273170 \\ 
		2.933180 1.304140 \\ 
		2.931280 1.304700 \\ 
		2.765960 1.332910 \\ 
		2.764010 1.333440 \\ 
		2.597490 1.358890 \\ 
		2.595510 1.359380 \\ 
		2.428090 1.382090 \\ 
		2.426070 1.382550 \\ 
		2.258060 1.402520 \\ 
		2.256020 1.402950 \\ 
		2.087700 1.420210 \\ 
		2.085630 1.420600 \\ 
		1.917300 1.435180 \\ 
		1.915220 1.435540 \\ 
		1.747150 1.447460 \\ 
		1.747150 1.447460 \\ 
		1.745050 1.447780 \\ 
		1.577540 1.457090 \\ 
		1.577530 1.457090 \\ 
		1.575430 1.457370 \\ 
		1.408730 1.464090 \\ 
		1.408720 1.464090 \\ 
		1.406610 1.464340 \\ 
		1.240990 1.468510 \\ 
		1.240980 1.468510 \\ 
		1.238870 1.468720 \\ 
		1.074590 1.470390 \\ 
		1.074580 1.470390 \\ 
		1.072470 1.470570 \\ 
		0.909780 1.469780 \\ 
		0.909770 1.469780 \\ 
		0.907660 1.469930 \\ 
		0.746800 1.466730 \\ 
		0.746790 1.466730 \\ 
		0.744690 1.466840 \\ 
		0.585900 1.461290 \\ 
		0.585890 1.461290 \\ 
		0.583790 1.461370 \\ 
		0.427290 1.453520 \\ 
		0.427290 1.453520 \\ 
		0.425210 1.453570 \\ 
		0.271220 1.443480 \\ 
		0.271210 1.443480 \\ 
		0.269150 1.443500 \\ 
		0.117870 1.431240 \\ 
		0.117870 1.431240 \\ 
		0.115820 1.431220 \\ 
		-0.032530 1.416840 \\ 
		-0.032530 1.416840 \\ 
		-0.034550 1.416800 \\ 
		-0.179790 1.400370 \\ 
		-0.179800 1.400370 \\ 
		-0.181790 1.400300 \\ 
		-0.323740 1.381890 \\ 
		-0.323740 1.381890 \\ 
		-0.325710 1.381790 \\ 
		-0.464190 1.361470 \\ 
		-0.464190 1.361470 \\ 
		-0.466130 1.361340 \\ 
		-0.600970 1.339180 \\ 
		-0.600970 1.339180 \\ 
		-0.602870 1.339030 \\ 
		-0.733930 1.315110 \\ 
		-0.733930 1.315110 \\ 
		-0.735800 1.314920 \\ 
		-0.862920 1.289310 \\ 
		-0.862920 1.289310 \\ 
		-0.864750 1.289110 \\ 
		-0.987800 1.261880 \\ 
		-0.987810 1.261880 \\ 
		-0.989590 1.261650 \\ 
		-1.108450 1.232890 \\ 
		-1.110190 1.232640 \\ 
		-1.224740 1.202420 \\ 
		-1.226430 1.202140 \\ 
		-1.336560 1.170560 \\ 
		-1.338200 1.170250 \\ 
		-1.443810 1.137370 \\ 
		-1.445410 1.137050 \\ 
		-1.546410 1.102950 \\ 
		-1.547960 1.102600 \\ 
		-1.644280 1.067370 \\ 
		-1.645770 1.067010 \\ 
		-1.737330 1.030720 \\ 
		-1.738770 1.030340 \\ 
		-1.825520 0.993090 \\ 
		-1.826900 0.992690 \\ 
		-1.908790 0.954550 \\ 
		-1.910110 0.954130 \\ 
		-1.987090 0.915190 \\ 
		-1.988360 0.914760 \\ 
		-2.060400 0.875090 \\ 
		-2.061610 0.874640 \\ 
		-2.128680 0.834330 \\ 
		-2.129820 0.833870 \\ 
		-2.191870 0.793030 \\ 
		-2.191930 0.792990 \\ 
		-2.192990 0.792530 \\ 
		-2.250020 0.751240 \\ 
		-2.250080 0.751200 \\ 
		-2.250140 0.751160 \\ 
		-2.251110 0.750710 \\ 
		-2.303130 0.709030 \\ 
		-2.303180 0.708990 \\ 
		-2.303240 0.708950 \\ 
		-2.303300 0.708910 \\ 
		-2.304170 0.708470 \\ 
		-2.351220 0.666490 \\ 
		-2.351260 0.666450 \\ 
		-2.351320 0.666410 \\ 
		-2.351370 0.666370 \\ 
		-2.351430 0.666330 \\ 
		-2.352200 0.665910 \\ 
		-2.394300 0.623690 \\ 
		-2.394340 0.623650 \\ 
		-2.394380 0.623610 \\ 
		-2.394440 0.623570 \\ 
		-2.394490 0.623530 \\ 
		-2.394550 0.623500 \\ 
		-2.395200 0.623100 \\ 
		-2.432390 0.580710 \\ 
		-2.432430 0.580670 \\ 
		-2.432470 0.580630 \\ 
		-2.432520 0.580590 \\ 
		-2.432570 0.580550 \\ 
		-2.432630 0.580520 \\ 
		-2.433230 0.580120 \\ 
		-2.465540 0.537630 \\ 
		-2.465580 0.537590 \\ 
		-2.465610 0.537550 \\ 
		-2.465650 0.537510 \\ 
		-2.465700 0.537470 \\ 
		-2.465750 0.537430 \\ 
		-2.466310 0.537030 \\ 
		-2.493790 0.494520 \\ 
		-2.493820 0.494480 \\ 
		-2.493850 0.494440 \\ 
		-2.493890 0.494400 \\ 
		-2.493930 0.494360 \\ 
		-2.493970 0.494320 \\ 
		-2.494480 0.493910 \\ 
		-2.517180 0.451450 \\ 
		-2.517200 0.451410 \\ 
		-2.517230 0.451370 \\ 
		-2.517260 0.451330 \\ 
		-2.517300 0.451290 \\ 
		-2.517340 0.451250 \\ 
		-2.517800 0.450840 \\ 
		-2.535770 0.408490 \\ 
		-2.535790 0.408450 \\ 
		-2.535810 0.408410 \\ 
		-2.535840 0.408380 \\ 
		-2.535870 0.408340 \\ 
		-2.535910 0.408300 \\ 
		-2.536320 0.407880 \\ 
		-2.549630 0.365730 \\ 
		-2.549640 0.365690 \\ 
		-2.549660 0.365650 \\ 
		-2.549690 0.365610 \\ 
		-2.549710 0.365570 \\ 
		-2.549750 0.365530 \\ 
		-2.550110 0.365110 \\ 
		-2.558820 0.323210 \\ 
		-2.558830 0.323170 \\ 
		-2.558850 0.323130 \\ 
		-2.558870 0.323090 \\ 
		-2.558890 0.323050 \\ 
		-2.558920 0.323010 \\ 
		-2.559230 0.322600 \\ 
		-2.563430 0.281020 \\ 
		-2.563430 0.280980 \\ 
		-2.563440 0.280940 \\ 
		-2.563460 0.280900 \\ 
		-2.563480 0.280860 \\ 
		-2.563500 0.280820 \\ 
		-2.563770 0.280410 \\ 
		-2.563530 0.239210 \\ 
		-2.563530 0.239170 \\ 
		-2.563540 0.239130 \\ 
		-2.563550 0.239090 \\ 
		-2.563560 0.239050 \\ 
		-2.563580 0.239010 \\ 
		-2.563800 0.238600 \\ 
		-2.559220 0.197850 \\ 
		-2.559210 0.197810 \\ 
		-2.559220 0.197770 \\ 
		-2.559220 0.197730 \\ 
		-2.559230 0.197690 \\ 
		-2.559250 0.197650 \\ 
		-2.559420 0.197240 \\ 
		-2.550580 0.157000 \\ 
		-2.550580 0.156960 \\ 
		-2.550570 0.156920 \\ 
		-2.550580 0.156880 \\ 
		-2.550580 0.156840 \\ 
		-2.550590 0.156800 \\ 
		-2.550720 0.156390 \\ 
		-2.537730 0.116710 \\ 
		-2.537720 0.116670 \\ 
		-2.537710 0.116640 \\ 
		-2.537710 0.116600 \\ 
		-2.537710 0.116560 \\ 
		-2.537720 0.116520 \\ 
		-2.537800 0.116110 \\ 
		-2.520760 0.077050 \\ 
		-2.520740 0.077010 \\ 
		-2.520730 0.076970 \\ 
		-2.520730 0.076940 \\ 
		-2.520730 0.076900 \\ 
		-2.520730 0.076860 \\ 
		-2.520770 0.076450 \\ 
		-2.499780 0.038060 \\ 
		-2.499770 0.038030 \\ 
		-2.499750 0.037990 \\ 
		-2.499740 0.037950 \\ 
		-2.499740 0.037920 \\ 
		-2.499730 0.037880 \\ 
		-2.499730 0.037480 \\ 
		-2.474920 -0.000190 \\ 
		-2.474900 -0.000220 \\ 
		-2.474880 -0.000260 \\ 
		-2.474870 -0.000300 \\ 
		-2.474860 -0.000340 \\ 
		-2.474850 -0.000370 \\ 
		-2.474800 -0.000770 \\ 
		-2.446280 -0.037660 \\ 
		-2.446260 -0.037700 \\ 
		-2.446240 -0.037730 \\ 
		-2.446220 -0.037770 \\ 
		-2.446210 -0.037810 \\ 
		-2.446200 -0.037840 \\ 
		-2.446110 -0.038240 \\ 
		-2.414000 -0.074310 \\ 
		-2.413970 -0.074340 \\ 
		-2.413940 -0.074380 \\ 
		-2.413920 -0.074410 \\ 
		-2.413910 -0.074450 \\ 
		-2.413890 -0.074490 \\ 
		-2.413760 -0.074870 \\ 
		-2.378190 -0.110080 \\ 
		-2.378160 -0.110120 \\ 
		-2.378130 -0.110150 \\ 
		-2.378110 -0.110180 \\ 
		-2.378080 -0.110220 \\ 
		-2.378070 -0.110260 \\ 
		-2.377900 -0.110640 \\ 
		-2.338990 -0.144940 \\ 
		-2.338960 -0.144970 \\ 
		-2.338920 -0.145010 \\ 
		-2.338900 -0.145040 \\ 
		-2.338870 -0.145070 \\ 
		-2.338850 -0.145110 \\ 
		-2.338640 -0.145490 \\ 
		-2.296530 -0.178840 \\ 
		-2.296490 -0.178870 \\ 
		-2.296460 -0.178900 \\ 
		-2.296430 -0.178940 \\ 
		-2.296400 -0.178970 \\ 
		-2.296370 -0.179010 \\ 
		-2.296130 -0.179380 \\ 
		-2.250950 -0.211750 \\ 
		-2.250910 -0.211780 \\ 
		-2.250870 -0.211810 \\ 
		-2.250840 -0.211840 \\ 
		-2.250800 -0.211880 \\ 
		-2.250780 -0.211910 \\ 
		-2.250500 -0.212270 \\ 
		-2.202380 -0.243620 \\ 
		-2.202340 -0.243650 \\ 
		-2.202300 -0.243680 \\ 
		-2.202260 -0.243720 \\ 
		-2.202230 -0.243750 \\ 
		-2.202200 -0.243780 \\ 
		-2.201880 -0.244130 \\ 
		-2.150970 -0.274440 \\ 
		-2.150930 -0.274460 \\ 
		-2.150880 -0.274490 \\ 
		-2.150840 -0.274530 \\ 
		-2.150800 -0.274560 \\ 
		-2.150770 -0.274590 \\ 
		-2.150420 -0.274930 \\ 
		-2.096860 -0.304150 \\ 
		-2.096810 -0.304180 \\ 
		-2.096770 -0.304210 \\ 
		-2.096720 -0.304240 \\ 
		-2.096680 -0.304270 \\ 
		-2.096650 -0.304300 \\ 
		-2.096260 -0.304640 \\ 
		-2.040190 -0.332750 \\ 
		-2.040140 -0.332780 \\ 
		-2.040090 -0.332800 \\ 
		-2.040050 -0.332830 \\ 
		-2.040000 -0.332860 \\ 
		-2.039960 -0.332890 \\ 
		-2.039550 -0.333220 \\ 
		-1.981120 -0.360200 \\ 
		-1.981060 -0.360220 \\ 
		-1.981010 -0.360250 \\ 
		-1.980960 -0.360280 \\ 
		-1.980920 -0.360310 \\ 
		-1.980870 -0.360340 \\ 
		-1.980430 -0.360650 \\ 
		-1.919780 -0.386470 \\ 
		-1.919720 -0.386500 \\ 
		-1.919670 -0.386520 \\ 
		-1.919620 -0.386550 \\ 
		-1.919570 -0.386580 \\ 
		-1.919520 -0.386610 \\ 
		-1.919050 -0.386910 \\ 
		-1.856320 -0.411550 \\ 
		-1.856260 -0.411580 \\ 
		-1.856210 -0.411600 \\ 
		-1.856150 -0.411630 \\ 
		-1.856100 -0.411660 \\ 
		-1.856050 -0.411680 \\ 
		-1.855770 -0.411850 \\ 
		-1.794900 -0.433930 \\ 
		-1.792590 -0.434790 \\ 
		-1.731770 -0.455140 \\ 
		-1.729400 -0.455960 \\ 
		-1.667090 -0.475200 \\ 
		-1.664650 -0.475980 \\ 
		-1.600980 -0.494070 \\ 
		-1.598480 -0.494810 \\ 
		-1.533590 -0.511770 \\ 
		-1.531040 -0.512460 \\ 
		-1.465050 -0.528270 \\ 
		-1.462460 -0.528910 \\ 
		-1.395510 -0.543580 \\ 
		-1.392870 -0.544180 \\ 
		-1.325090 -0.557690 \\ 
		-1.322430 -0.558240 \\ 
		-1.253950 -0.570600 \\ 
		-1.251250 -0.571110 \\ 
		-1.182200 -0.582320 \\ 
		-1.179490 -0.582780 \\ 
		-1.109990 -0.592850 \\ 
		-1.107250 -0.593270 \\ 
		-1.037440 -0.602190 \\ 
		-1.034690 -0.602560 \\ 
		-0.964690 -0.610360 \\ 
		-0.961920 -0.610690 \\ 
		-0.891860 -0.617360 \\ 
		-0.889080 -0.617640 \\ 
		-0.819060 -0.623200 \\ 
		-0.816290 -0.623440 \\ 
		-0.746440 -0.627890 \\ 
		-0.743670 -0.628090 \\ 
		-0.674100 -0.631460 \\ 
		-0.671330 -0.631610 \\ 
		-0.602150 -0.633920 \\ 
		-0.599410 -0.634030 \\ 
		-0.530730 -0.635280 \\ 
		-0.528000 -0.635350 \\ 
		-0.459920 -0.635570 \\ 
		-0.457260 -0.635590 \\ 
		-0.389910 -0.634800 \\ 
		-0.387230 -0.634790 \\ 
		-0.320670 -0.633000 \\ 
		-0.318030 -0.632950 \\ 
		-0.252370 -0.630190 \\ 
		-0.249760 -0.630100 \\ 
		-0.185090 -0.626400 \\ 
		-0.182520 -0.626270 \\ 
		-0.118940 -0.621650 \\ 
		-0.116420 -0.621480 \\ 
		-0.054000 -0.615970 \\ 
		-0.051520 -0.615770 \\ 
		0.009640 -0.609390 \\ 
		0.012070 -0.609150 \\ 
		0.071900 -0.601930 \\ 
		0.074240 -0.601660 \\ 
		0.132710 -0.593620 \\ 
		0.135020 -0.593320 \\ 
		0.191980 -0.584510 \\ 
		0.194260 -0.584170 \\ 
		0.249660 -0.574610 \\ 
		0.251750 -0.574260 \\ 
		0.305670 -0.563960 \\ 
		0.307770 -0.563570 \\ 
		0.359950 -0.552590 \\ 
		0.361980 -0.552180 \\ 
		0.412450 -0.540540 \\ 
		0.414360 -0.540120 \\ 
		0.463120 -0.527840 \\ 
		0.464980 -0.527380 \\ 
		0.511890 -0.514520 \\ 
		0.513770 -0.514020 \\ 
		0.558740 -0.500620 \\ 
		0.560460 -0.500130 \\ 
		0.603620 -0.486180 \\ 
		0.605130 -0.485710 \\ 
		0.646480 -0.471220 \\ 
		0.648140 -0.470660 \\ 
		0.687320 -0.455790 \\ 
		0.688790 -0.455250 \\ 
		0.726090 -0.439910 \\ 
		0.727590 -0.439320 \\ 
		0.762760 -0.423640 \\ 
		0.763930 -0.423140 \\ 
		0.797300 -0.406990 \\ 
		0.798640 -0.406360 \\ 
		0.829740 -0.390010 \\ 
		0.830770 -0.389490 \\ 
		0.860010 -0.372730 \\ 
		0.861200 -0.372070 \\ 
		0.888160 -0.355180 \\ 
		0.889220 -0.354550 \\ 
		0.914090 -0.337450 \\ 
		0.915070 -0.336800 \\ 
		0.937870 -0.319520 \\ 
		0.938770 -0.318860 \\ 
		0.959450 -0.301470 \\ 
		0.959500 -0.301430 \\ 
		0.960200 -0.300870 \\ 
		0.978930 -0.283250 \\ 
		0.978970 -0.283210 \\ 
		0.979720 -0.282540 \\ 
		0.996310 -0.264920 \\ 
		0.996350 -0.264880 \\ 
		0.996980 -0.264240 \\ 
		1.011570 -0.246520 \\ 
		1.011600 -0.246480 \\ 
		1.012190 -0.245810 \\ 
		1.024720 -0.228100 \\ 
		1.024750 -0.228070 \\ 
		1.025170 -0.227500 \\ 
		1.035760 -0.209690 \\ 
		1.035790 -0.209650 \\ 
		1.036220 -0.208970 \\ 
		1.044790 -0.191270 \\ 
		1.045160 -0.190560 \\ 
		1.051770 -0.172910 \\ 
		1.052050 -0.172230 \\ 
		1.056730 -0.154630 \\ 
		1.056950 -0.153920 \\ 
		1.059720 -0.136480 \\ 
		1.059860 -0.135790 \\ 
		1.060750 -0.118490 \\ 
		1.060780 -0.118220 \\ 
		1.059870 -0.101640 \\ 
		1.059870 -0.101000 \\ 
		1.057120 -0.084040 \\ 
		1.057050 -0.083380 \\ 
		1.052550 -0.066670 \\ 
		1.052400 -0.066000 \\ 
		1.046190 -0.049550 \\ 
		1.045980 -0.048910 \\ 
		1.038090 -0.032710 \\ 
		1.037830 -0.032100 \\ 
		1.028300 -0.016170 \\ 
		1.027950 -0.015540 \\ 
		1.016890 0.000010 \\ 
		1.016870 0.000050 \\ 
		1.016500 0.000610 \\ 
		1.003870 0.015880 \\ 
		1.003410 0.016470 \\ 
		0.989340 0.031360 \\ 
		0.988890 0.031860 \\ 
		0.973330 0.046450 \\ 
		0.972820 0.046950 \\ 
		0.955900 0.061140 \\ 
		0.955270 0.061690 \\ 
		0.937100 0.075420 \\ 
		0.936590 0.075820 \\ 
		0.916980 0.089270 \\ 
		0.916380 0.089690 \\ 
		0.895630 0.102660 \\ 
		0.894860 0.103160 \\ 
		0.873090 0.115600 \\ 
		0.872420 0.116000 \\ 
		0.849440 0.128060 \\ 
		0.848760 0.128430 \\ 
		0.824720 0.140030 \\ 
		0.823820 0.140480 \\ 
		0.799000 0.151510 \\ 
		0.798260 0.151860 \\ 
		0.772340 0.162490 \\ 
		0.771380 0.162890 \\ 
		0.747220 0.172040 \\ 
		0.722940 0.180530 \\ 
		0.697670 0.188680 \\ 
		0.672500 0.196160 \\ 
		0.646130 0.203370 \\ 
		0.619640 0.210030 \\ 
		0.592730 0.216210 \\ 
		0.566020 0.221820 \\ 
		0.537870 0.227190 \\ 
		0.510020 0.231980 \\ 
		0.482140 0.236280 \\ 
		0.453810 0.240160 \\ 
		0.425620 0.243550 \\ 
		0.396670 0.246550 \\ 
		0.368220 0.249050 \\ 
		0.339820 0.251100 \\ 
		0.311500 0.252710 \\ 
		0.283380 0.253880 \\ 
		0.255290 0.254620 \\ 
		0.228150 0.254930 \\ 
		0.204180 0.254840 \\ 
		0.177040 0.254330 \\ 
		0.150280 0.253430 \\ 
		0.123700 0.252130 \\ 
		0.097920 0.250470 \\ 
		0.072280 0.248420 \\ 
		0.046980 0.246000 \\ 
		0.022730 0.243290 \\ 
		-0.003610 0.239910 \\ 
		-0.005470 0.239680 \\ 
		-0.009610 0.239210 \\ 
		-0.011510 0.239010 \\ 
		-0.015730 0.238600 \\ 
		-0.017680 0.238430 \\ 
		-0.021980 0.238080 \\ 
		-0.023960 0.237930 \\ 
		-0.028350 0.237640 \\ 
		-0.030360 0.237520 \\ 
		-0.034820 0.237300 \\ 
		-0.035740 0.237250 \\ 
		-0.037790 0.237160 \\ 
		-0.042310 0.237000 \\ 
		-0.043250 0.236970 \\ 
		-0.045330 0.236910 \\ 
		-0.049910 0.236820 \\ 
		-0.050860 0.236800 \\ 
		-0.052960 0.236780 \\ 
		-0.057590 0.236750 \\ 
		-0.066120 0.236750 \\ 
		-0.070800 0.236800 \\ 
		-0.072940 0.236830 \\ 
		-0.077650 0.236950 \\ 
		-0.078630 0.236970 \\ 
		-0.080790 0.237040 \\ 
		-0.085540 0.237230 \\ 
		-0.086520 0.237270 \\ 
		-0.088690 0.237380 \\ 
		-0.093480 0.237640 \\ 
		-0.094470 0.237700 \\ 
		-0.096650 0.237830 \\ 
		-0.101430 0.238170 \\ 
		-0.102430 0.238240 \\ 
		-0.104610 0.238410 \\ 
		-0.109410 0.238830 \\ 
		-0.110400 0.238910 \\ 
		-0.112590 0.239120 \\ 
		-0.117380 0.239610 \\ 
		-0.118380 0.239710 \\ 
		-0.120560 0.239950 \\ 
		-0.125350 0.240520 \\ 
		-0.126340 0.240640 \\ 
		-0.128520 0.240910 \\ 
		-0.133290 0.241560 \\ 
		-0.134280 0.241690 \\ 
		-0.136450 0.242000 \\ 
		-0.141190 0.242730 \\ 
		-0.142180 0.242880 \\ 
		-0.144690 0.243260 \\ 
		-0.146840 0.243610 \\ 
		-0.151550 0.244410 \\ 
		-0.152530 0.244580 \\ 
		-0.154670 0.244960 \\ 
		-0.159330 0.245840 \\ 
		-0.160300 0.246030 \\ 
		-0.162420 0.246450 \\ 
		-0.167030 0.247400 \\ 
		-0.167990 0.247600 \\ 
		-0.168060 0.247620 \\ 
		-0.170150 0.248070 \\ 
		-0.174700 0.249110 \\ 
		-0.175650 0.249330 \\ 
		-0.175720 0.249340 \\ 
		-0.177780 0.249840 \\ 
		-0.182260 0.250950 \\ 
		-0.183190 0.251180 \\ 
		-0.183260 0.251200 \\ 
		-0.185280 0.251730 \\ 
		-0.189690 0.252920 \\ 
		-0.190610 0.253170 \\ 
		-0.190670 0.253190 \\ 
		-0.296650 0.282410 \\ 
		-0.298640 0.282980 \\ 
		-0.298800 0.283020 \\ 
		-0.303120 0.284290 \\ 
		-0.304010 0.284560 \\ 
		-0.304080 0.284580 \\ 
		-0.306020 0.285180 \\ 
		-0.310240 0.286530 \\ 
		-0.311120 0.286810 \\ 
		-0.311180 0.286830 \\ 
		-0.313080 0.287460 \\ 
		-0.317200 0.288880 \\ 
		-0.318050 0.289180 \\ 
		-0.318110 0.289200 \\ 
		-0.319960 0.289870 \\ 
		-0.323960 0.291370 \\ 
		-0.324790 0.291680 \\ 
		-0.324850 0.291710 \\ 
		-0.326640 0.292410 \\ 
		-0.327450 0.292740 \\ 
		-0.327500 0.292760 \\ 
		-0.329240 0.293500 \\ 
		-0.330010 0.293840 \\ 
		-0.330070 0.293870 \\ 
		-0.331740 0.294630 \\ 
		-0.332480 0.294990 \\ 
		-0.332540 0.295020 \\ 
		-0.334140 0.295820 \\ 
		-0.334860 0.296190 \\ 
		-0.334910 0.296220 \\ 
		-0.336440 0.297050 \\ 
		-0.337120 0.297440 \\ 
		-0.337170 0.297470 \\ 
		-0.338620 0.298330 \\ 
		-0.339270 0.298730 \\ 
		-0.339310 0.298760 \\ 
		-0.340690 0.299660 \\ 
		-0.341300 0.300070 \\ 
		-0.341340 0.300100 \\ 
		-0.342640 0.301020 \\ 
		-0.343200 0.301450 \\ 
		-0.343240 0.301480 \\ 
		-0.344450 0.302430 \\ 
		-0.344980 0.302870 \\ 
		-0.346090 0.303850 \\ 
		-0.346580 0.304310 \\ 
		-0.347600 0.305310 \\ 
		-0.348040 0.305770 \\ 
		-0.348960 0.306800 \\ 
		-0.349350 0.307280 \\ 
		-0.350170 0.308330 \\ 
		-0.350520 0.308820 \\ 
		-0.351230 0.309890 \\ 
		-0.351530 0.310390 \\ 
		-0.352140 0.311480 \\ 
		-0.352380 0.311990 \\ 
		-0.352880 0.313100 \\ 
		-0.353070 0.313610 \\ 
		-0.353450 0.314750 \\ 
		-0.353590 0.315270 \\ 
		-0.353850 0.316410 \\ 
		-0.353930 0.316940 \\ 
		-0.354070 0.318100 \\ 
		-0.354070 0.322380 \\ 
		-0.353960 0.323570 \\ 
		-0.353870 0.324110 \\ 
		-0.353630 0.325310 \\ 
		-0.353480 0.325850 \\ 
		-0.353110 0.327060 \\ 
		-0.352900 0.327610 \\ 
		-0.352400 0.328820 \\ 
		-0.352130 0.329370 \\ 
		-0.351490 0.330590 \\ 
		-0.351160 0.331140 \\ 
		-0.350380 0.332350 \\ 
		-0.349990 0.332910 \\ 
		-0.349080 0.334120 \\ 
		-0.348630 0.334670 \\ 
		-0.347570 0.335880 \\ 
		-0.347050 0.336430 \\ 
		-0.345860 0.337640 \\ 
		-0.345280 0.338190 \\ 
		-0.345230 0.338230 \\ 
		-0.344210 0.339170 \\ 
		-0.342870 0.340370 \\ 
		-0.342220 0.340910 \\ 
		-0.342180 0.340950 \\ 
		-0.340700 0.342140 \\ 
		-0.339990 0.342680 \\ 
		-0.339940 0.342720 \\ 
		-0.338320 0.343900 \\ 
		-0.337540 0.344430 \\ 
		-0.337490 0.344470 \\ 
		-0.335730 0.345630 \\ 
		-0.334890 0.346160 \\ 
		-0.334830 0.346200 \\ 
		-0.332920 0.347350 \\ 
		-0.332020 0.347870 \\ 
		-0.329970 0.349000 \\ 
		-0.329010 0.349510 \\ 
		-0.251070 0.389250 \\ 
		-0.248880 0.390360 \\ 
		-0.248760 0.390420 \\ 
		-0.247730 0.390920 \\ 
		-0.245400 0.392010 \\ 
		-0.244310 0.392500 \\ 
		-0.241840 0.393560 \\ 
		-0.240680 0.394040 \\ 
		-0.238080 0.395070 \\ 
		-0.238040 0.395090 \\ 
		-0.236820 0.395550 \\ 
		-0.234090 0.396560 \\ 
		-0.234050 0.396570 \\ 
		-0.232760 0.397020 \\ 
		-0.229890 0.398000 \\ 
		-0.228550 0.398430 \\ 
		-0.225550 0.399370 \\ 
		-0.224150 0.399790 \\ 
		-0.221020 0.400690 \\ 
		-0.219560 0.401090 \\ 
		-0.216300 0.401950 \\ 
		-0.214790 0.402330 \\ 
		-0.211410 0.403160 \\ 
		-0.209840 0.403520 \\ 
		-0.206340 0.404300 \\ 
		-0.204710 0.404640 \\ 
		-0.201100 0.405370 \\ 
		-0.199420 0.405690 \\ 
		-0.195690 0.406370 \\ 
		-0.193970 0.406670 \\ 
		-0.190130 0.407310 \\ 
		-0.188360 0.407580 \\ 
		-0.184420 0.408160 \\ 
		-0.182600 0.408420 \\ 
		-0.178560 0.408940 \\ 
		-0.146620 0.413040 \\ 
		-0.146150 0.413110 \\ 
		-0.144330 0.413360 \\ 
		-0.140290 0.413890 \\ 
		-0.138430 0.414120 \\ 
		-0.134300 0.414590 \\ 
		-0.101960 0.418200 \\ 
		-0.101180 0.418300 \\ 
		-0.099320 0.418530 \\ 
		-0.095180 0.419000 \\ 
		-0.093280 0.419200 \\ 
		-0.089060 0.419610 \\ 
		-0.055800 0.422790 \\ 
		-0.055200 0.422860 \\ 
		-0.053290 0.423060 \\ 
		-0.049070 0.423470 \\ 
		-0.047130 0.423650 \\ 
		-0.042820 0.424000 \\ 
		-0.008920 0.426710 \\ 
		-0.008340 0.426760 \\ 
		-0.006400 0.426930 \\ 
		-0.002090 0.427290 \\ 
		-0.000110 0.427430 \\ 
		0.004270 0.427720 \\ 
		0.038550 0.429930 \\ 
		0.039370 0.429990 \\ 
		0.041360 0.430140 \\ 
		0.045740 0.430430 \\ 
		0.047750 0.430540 \\ 
		0.052210 0.430770 \\ 
		0.087210 0.432480 \\ 
		0.087920 0.432530 \\ 
		0.089940 0.432650 \\ 
		0.094390 0.432870 \\ 
		0.096440 0.432960 \\ 
		0.100960 0.433120 \\ 
		0.136400 0.434320 \\ 
		0.137180 0.434360 \\ 
		0.139230 0.434450 \\ 
		0.143750 0.434610 \\ 
		0.145820 0.434660 \\ 
		0.150400 0.434760 \\ 
		0.186290 0.435430 \\ 
		0.187100 0.435460 \\ 
		0.189180 0.435510 \\ 
		0.193760 0.435610 \\ 
		0.195860 0.435630 \\ 
		0.200490 0.435660 \\ 
		0.236680 0.435790 \\ 
		0.237560 0.435810 \\ 
		0.239660 0.435840 \\ 
		0.244290 0.435860 \\ 
		0.250530 0.435860 \\ 
		0.255210 0.435820 \\ 
		0.291810 0.435400 \\ 
		0.294760 0.435400 \\ 
		0.299430 0.435360 \\ 
		0.301580 0.435320 \\ 
		0.306290 0.435200 \\ 
		0.343140 0.434230 \\ 
		0.343990 0.434220 \\ 
		0.346130 0.434190 \\ 
		0.350840 0.434070 \\ 
		0.353000 0.434000 \\ 
		0.357750 0.433810 \\ 
		0.394810 0.432270 \\ 
		0.395720 0.432250 \\ 
		0.397880 0.432180 \\ 
		0.402630 0.431990 \\ 
		0.404800 0.431890 \\ 
		0.409560 0.431630 \\ 
		0.446810 0.429510 \\ 
		0.447720 0.429480 \\ 
		0.449890 0.429370 \\ 
		0.454660 0.429110 \\ 
		0.456840 0.428970 \\ 
		0.461630 0.428640 \\ 
		0.498980 0.425940 \\ 
		0.499900 0.425890 \\ 
		0.502080 0.425750 \\ 
		0.506870 0.425410 \\ 
		0.509050 0.425240 \\ 
		0.513840 0.424830 \\ 
		0.551260 0.421540 \\ 
		0.552180 0.421480 \\ 
		0.554360 0.421310 \\ 
		0.559160 0.420890 \\ 
		0.561340 0.420690 \\ 
		0.566140 0.420200 \\ 
		0.603880 0.416280 \\ 
		0.604400 0.416230 \\ 
		0.606580 0.416030 \\ 
		0.611370 0.415540 \\ 
		0.613560 0.415300 \\ 
		0.618340 0.414730 \\ 
		0.655850 0.410220 \\ 
		0.656570 0.410150 \\ 
		0.658750 0.409910 \\ 
		0.663540 0.409340 \\ 
		0.665710 0.409060 \\ 
		0.670480 0.408420 \\ 
		0.707870 0.403300 \\ 
		0.708570 0.403220 \\ 
		0.710740 0.402940 \\ 
		0.715510 0.402300 \\ 
		0.717680 0.401980 \\ 
		0.722420 0.401260 \\ 
		0.759410 0.395560 \\ 
		0.760330 0.395430 \\ 
		0.762500 0.395120 \\ 
		0.767240 0.394400 \\ 
		0.769400 0.394050 \\ 
		0.774110 0.393250 \\ 
		0.810820 0.386930 \\ 
		0.811740 0.386790 \\ 
		0.813890 0.386450 \\ 
		0.818600 0.385650 \\ 
		0.818670 0.385630 \\ 
		0.820810 0.385250 \\ 
		0.825470 0.384370 \\ 
		0.862370 0.377340 \\ 
		0.862690 0.377290 \\ 
		0.862760 0.377270 \\ 
		0.864890 0.376890 \\ 
		0.869560 0.376010 \\ 
		0.869630 0.376000 \\ 
		0.871740 0.375580 \\ 
		0.876360 0.374620 \\ 
		0.912320 0.367070 \\ 
		0.913150 0.366920 \\ 
		0.913220 0.366900 \\ 
		0.915340 0.366480 \\ 
		0.919950 0.365530 \\ 
		0.920020 0.365510 \\ 
		0.922110 0.365050 \\ 
		0.926670 0.364020 \\ 
		0.962140 0.355860 \\ 
		0.962970 0.355690 \\ 
		0.963040 0.355670 \\ 
		0.965130 0.355220 \\ 
		0.969680 0.354180 \\ 
		0.969750 0.354160 \\ 
		0.971810 0.353670 \\ 
		0.976290 0.352560 \\ 
		1.011220 0.343780 \\ 
		1.012040 0.343600 \\ 
		1.012100 0.343580 \\ 
		1.014160 0.343090 \\ 
		1.018650 0.341980 \\ 
		1.018720 0.341960 \\ 
		1.020740 0.341430 \\ 
		1.025150 0.340240 \\ 
		1.059730 0.330780 \\ 
		1.060230 0.330660 \\ 
		1.060290 0.330640 \\ 
		1.062320 0.330110 \\ 
		1.066730 0.328920 \\ 
		1.066790 0.328900 \\ 
		1.068780 0.328340 \\ 
		1.073100 0.327070 \\ 
		1.106710 0.317080 \\ 
		1.107500 0.316860 \\ 
		1.107560 0.316840 \\ 
		1.109550 0.316280 \\ 
		1.113870 0.315010 \\ 
		1.113930 0.314990 \\ 
		1.115880 0.314390 \\ 
		1.120100 0.313040 \\ 
		1.153230 0.302360 \\ 
		1.153690 0.302230 \\ 
		1.153760 0.302210 \\ 
		1.155700 0.301610 \\ 
		1.159920 0.300260 \\ 
		1.159990 0.300240 \\ 
		1.161890 0.299610 \\ 
		1.166000 0.298180 \\ 
		1.198290 0.286900 \\ 
		1.198740 0.286750 \\ 
		1.198800 0.286730 \\ 
		1.200700 0.286100 \\ 
		1.204820 0.284680 \\ 
		1.204880 0.284650 \\ 
		1.206730 0.283990 \\ 
		1.206750 0.283980 \\ 
		1.210750 0.282480 \\ 
		1.241840 0.270710 \\ 
		1.242560 0.270460 \\ 
		1.242620 0.270430 \\ 
		1.244470 0.269770 \\ 
		1.244500 0.269760 \\ 
		1.248500 0.268260 \\ 
		1.248560 0.268230 \\ 
		1.250350 0.267530 \\ 
		1.250380 0.267520 \\ 
		1.282930 0.254170 \\ 
		1.283840 0.253780 \\ 
		1.285060 0.253320 \\ 
		1.285120 0.253300 \\ 
		1.286910 0.252600 \\ 
		1.286940 0.252590 \\ 
		1.287000 0.252570 \\ 
		1.288730 0.251830 \\ 
		1.320150 0.237860 \\ 
		1.321140 0.237400 \\ 
		1.322210 0.236990 \\ 
		1.322230 0.236970 \\ 
		1.322290 0.236950 \\ 
		1.324030 0.236220 \\ 
		1.324080 0.236190 \\ 
		1.325750 0.235420 \\ 
		1.355960 0.220850 \\ 
		1.357050 0.220300 \\ 
		1.357940 0.219920 \\ 
		1.357960 0.219910 \\ 
		1.358020 0.219890 \\ 
		1.359690 0.219120 \\ 
		1.359740 0.219090 \\ 
		1.361350 0.218290 \\ 
		1.390260 0.203120 \\ 
		1.391280 0.202570 \\ 
		1.392190 0.202150 \\ 
		1.392240 0.202120 \\ 
		1.393840 0.201320 \\ 
		1.393890 0.201300 \\ 
		1.395430 0.200460 \\ 
		1.422970 0.184720 \\ 
		1.423840 0.184210 \\ 
		1.424810 0.183720 \\ 
		1.424860 0.183690 \\ 
		1.426400 0.182860 \\ 
		1.426450 0.182830 \\ 
		1.427900 0.181960 \\ 
		1.454010 0.165660 \\ 
		1.454930 0.165060 \\ 
		1.455760 0.164610 \\ 
		1.455810 0.164580 \\ 
		1.457260 0.163710 \\ 
		1.457310 0.163690 \\ 
		1.458690 0.162790 \\ 
		1.483280 0.145940 \\ 
		1.484160 0.145300 \\ 
		1.484910 0.144860 \\ 
		1.484950 0.144830 \\ 
		1.486330 0.143940 \\ 
		1.486370 0.143910 \\ 
		1.487670 0.142990 \\ 
		1.510670 0.125600 \\ 
		1.511380 0.125040 \\ 
		1.512220 0.124490 \\ 
		1.512260 0.124460 \\ 
		1.513560 0.123540 \\ 
		1.513600 0.123510 \\ 
		1.514800 0.122560 \\ 
		1.536150 0.104670 \\ 
		1.536860 0.104040 \\ 
		1.537590 0.103520 \\ 
		1.537640 0.103490 \\ 
		1.538840 0.102530 \\ 
		1.538880 0.102500 \\ 
		1.539990 0.101520 \\ 
		1.559610 0.083140 \\ 
		1.560270 0.082490 \\ 
		1.560940 0.081960 \\ 
		1.560980 0.081920 \\ 
		1.562100 0.080950 \\ 
		1.562130 0.080910 \\ 
		1.563150 0.079910 \\ 
		1.580960 0.061060 \\ 
		1.581510 0.060450 \\ 
		1.582190 0.059850 \\ 
		1.582220 0.059820 \\ 
		1.583240 0.058820 \\ 
		1.583270 0.058780 \\ 
		1.584200 0.057760 \\ 
		1.600140 0.038460 \\ 
		1.600660 0.037800 \\ 
		1.601250 0.037220 \\ 
		1.601280 0.037190 \\ 
		1.602200 0.036160 \\ 
		1.602230 0.036120 \\ 
		1.603050 0.035070 \\ 
		1.617040 0.015390 \\ 
		1.617070 0.015360 \\ 
		1.617490 0.014720 \\ 
		1.618050 0.014090 \\ 
		1.618080 0.014060 \\ 
		1.618900 0.013010 \\ 
		1.618920 0.012970 \\ 
		1.619630 0.011900 \\ 
		1.631640 -0.008170 \\ 
		1.632030 -0.008890 \\ 
		1.632510 -0.009510 \\ 
		1.632540 -0.009550 \\ 
		1.633250 -0.010620 \\ 
		1.633850 -0.011720 \\ 
		1.643810 -0.032160 \\ 
		1.644130 -0.032880 \\ 
		1.644570 -0.033550 \\ 
		1.645170 -0.034650 \\ 
		1.645670 -0.035760 \\ 
		1.653510 -0.056540 \\ 
		1.653740 -0.057250 \\ 
		1.654130 -0.057950 \\ 
		1.654620 -0.059060 \\ 
		1.655000 -0.060200 \\ 
		1.660670 -0.081290 \\ 
		1.660840 -0.082030 \\ 
		1.661150 -0.082730 \\ 
		1.661520 -0.083870 \\ 
		1.661780 -0.085010 \\ 
		1.665240 -0.106380 \\ 
		1.665320 -0.107130 \\ 
		1.665560 -0.107850 \\ 
		1.665820 -0.108990 \\ 
		1.665960 -0.110160 \\ 
		1.667140 -0.131780 \\ 
		1.667140 -0.132520 \\ 
		1.667320 -0.133290 \\ 
		1.667450 -0.134450 \\ 
		1.667450 -0.136550 \\ 
		1.666320 -0.158390 \\ 
		1.666300 -0.158570 \\ 
		1.666350 -0.158950 \\ 
		1.666350 -0.161050 \\ 
		1.666230 -0.162230 \\ 
		1.662740 -0.184260 \\ 
		1.662600 -0.184970 \\ 
		1.662600 -0.185760 \\ 
		1.662480 -0.186950 \\ 
		1.662240 -0.188140 \\ 
		1.656340 -0.210320 \\ 
		1.656100 -0.211080 \\ 
		1.656030 -0.211810 \\ 
		1.655790 -0.213010 \\ 
		1.655410 -0.214210 \\ 
		1.647070 -0.236500 \\ 
		1.646760 -0.237260 \\ 
		1.646600 -0.238020 \\ 
		1.646230 -0.239230 \\ 
		1.645720 -0.240440 \\ 
		1.634900 -0.262810 \\ 
		1.634500 -0.263570 \\ 
		1.634260 -0.264340 \\ 
		1.633750 -0.265540 \\ 
		1.633110 -0.266760 \\ 
		1.619780 -0.289170 \\ 
		1.619760 -0.289210 \\ 
		1.619240 -0.290010 \\ 
		1.618970 -0.290660 \\ 
		1.618950 -0.290700 \\ 
		1.618300 -0.291920 \\ 
		1.618280 -0.291960 \\ 
		1.617500 -0.293170 \\ 
		1.601640 -0.315590 \\ 
		1.601610 -0.315630 \\ 
		1.601070 -0.316340 \\ 
		1.600670 -0.317080 \\ 
		1.600650 -0.317120 \\ 
		1.599870 -0.318340 \\ 
		1.599840 -0.318380 \\ 
		1.598930 -0.319590 \\ 
		1.580500 -0.341970 \\ 
		1.580470 -0.342010 \\ 
		1.579830 -0.342750 \\ 
		1.579370 -0.343460 \\ 
		1.579340 -0.343500 \\ 
		1.578420 -0.344720 \\ 
		1.578390 -0.344750 \\ 
		1.577330 -0.345970 \\ 
		1.556330 -0.368280 \\ 
		1.556290 -0.368320 \\ 
		1.555580 -0.369030 \\ 
		1.555020 -0.369770 \\ 
		1.554990 -0.369810 \\ 
		1.553930 -0.371020 \\ 
		1.553900 -0.371060 \\ 
		1.552700 -0.372260 \\ 
		1.529100 -0.394460 \\ 
		1.529060 -0.394500 \\ 
		1.528190 -0.395280 \\ 
		1.527620 -0.395930 \\ 
		1.527580 -0.395970 \\ 
		1.526390 -0.397180 \\ 
		1.526370 -0.397200 \\ 
		1.526330 -0.397240 \\ 
		1.524990 -0.398430 \\ 
		1.524970 -0.398450 \\ 
		1.498760 -0.420500 \\ 
		1.498710 -0.420540 \\ 
		1.497850 -0.421230 \\ 
		1.497140 -0.421940 \\ 
		1.497120 -0.421960 \\ 
		1.497080 -0.422000 \\ 
		1.495750 -0.423200 \\ 
		1.495730 -0.423220 \\ 
		1.495680 -0.423250 \\ 
		1.494200 -0.424440 \\ 
		1.494180 -0.424460 \\ 
		1.465300 -0.446350 \\ 
		1.464350 -0.447040 \\ 
		1.463560 -0.447750 \\ 
		1.463540 -0.447760 \\ 
		1.463500 -0.447800 \\ 
		1.462020 -0.448990 \\ 
		1.462000 -0.449010 \\ 
		1.461950 -0.449050 \\ 
		1.460330 -0.450230 \\ 
		1.460300 -0.450250 \\ 
		1.428800 -0.471900 \\ 
		1.427770 -0.472580 \\ 
		1.426900 -0.473280 \\ 
		1.426870 -0.473300 \\ 
		1.426820 -0.473340 \\ 
		1.425200 -0.474520 \\ 
		1.425180 -0.474530 \\ 
		1.425120 -0.474570 \\ 
		1.423360 -0.475740 \\ 
		1.423330 -0.475750 \\ 
		1.389200 -0.497130 \\ 
		1.388100 -0.497800 \\ 
		1.387140 -0.498500 \\ 
		1.387120 -0.498520 \\ 
		1.387060 -0.498550 \\ 
		1.385300 -0.499720 \\ 
		1.385270 -0.499740 \\ 
		1.385210 -0.499770 \\ 
		1.383310 -0.500920 \\ 
		1.383280 -0.500940 \\ 
		1.346590 -0.521960 \\ 
		1.345200 -0.522730 \\ 
		1.344280 -0.523340 \\ 
		1.344260 -0.523360 \\ 
		1.344200 -0.523390 \\ 
		1.342290 -0.524540 \\ 
		1.342260 -0.524560 \\ 
		1.340220 -0.525690 \\ 
		1.340190 -0.525710 \\ 
		1.300900 -0.546370 \\ 
		1.299550 -0.547050 \\ 
		1.298370 -0.547770 \\ 
		1.298340 -0.547780 \\ 
		1.296300 -0.548910 \\ 
		1.296270 -0.548930 \\ 
		1.294080 -0.550040 \\ 
		1.294050 -0.550060 \\ 
		1.252170 -0.570310 \\ 
		1.250550 -0.571070 \\ 
		1.249440 -0.571690 \\ 
		1.249410 -0.571700 \\ 
		1.247220 -0.572810 \\ 
		1.247190 -0.572830 \\ 
		1.244860 -0.573920 \\ 
		1.244830 -0.573930 \\ 
		1.200380 -0.593740 \\ 
		1.198860 -0.594400 \\ 
		1.197510 -0.595090 \\ 
		1.197470 -0.595100 \\ 
		1.195150 -0.596190 \\ 
		1.195110 -0.596210 \\ 
		1.192650 -0.597270 \\ 
		1.192610 -0.597280 \\ 
		1.145620 -0.616610 \\ 
		1.143940 -0.617270 \\ 
		1.142570 -0.617910 \\ 
		1.142540 -0.617930 \\ 
		1.140070 -0.618990 \\ 
		1.140040 -0.619010 \\ 
		1.137430 -0.620040 \\ 
		1.137400 -0.620060 \\ 
		1.087890 -0.638840 \\ 
		1.086200 -0.639460 \\ 
		1.084670 -0.640120 \\ 
		1.084640 -0.640140 \\ 
		1.082030 -0.641170 \\ 
		1.082000 -0.641190 \\ 
		1.079260 -0.642190 \\ 
		1.079220 -0.642210 \\ 
		1.027230 -0.660420 \\ 
		1.025310 -0.661070 \\ 
		1.023830 -0.661660 \\ 
		1.023800 -0.661670 \\ 
		1.021060 -0.662680 \\ 
		1.021020 -0.662690 \\ 
		1.018150 -0.663660 \\ 
		1.018100 -0.663680 \\ 
		0.963680 -0.681270 \\ 
		0.961750 -0.681870 \\ 
		0.960120 -0.682470 \\ 
		0.960070 -0.682490 \\ 
		0.957200 -0.683460 \\ 
		0.957160 -0.683470 \\ 
		0.954160 -0.684410 \\ 
		0.954110 -0.684420 \\ 
		0.897290 -0.701360 \\ 
		0.895350 -0.701910 \\ 
		0.893580 -0.702510 \\ 
		0.893530 -0.702530 \\ 
		0.890530 -0.703470 \\ 
		0.890490 -0.703480 \\ 
		0.887360 -0.704380 \\ 
		0.887310 -0.704400 \\ 
		0.828130 -0.720620 \\ 
		0.826060 -0.721170 \\ 
		0.824250 -0.721740 \\ 
		0.824210 -0.721750 \\ 
		0.821080 -0.722650 \\ 
		0.821030 -0.722660 \\ 
		0.817770 -0.723530 \\ 
		0.817730 -0.723540 \\ 
		0.756260 -0.739020 \\ 
		0.754100 -0.739540 \\ 
		0.752230 -0.740080 \\ 
		0.752180 -0.740090 \\ 
		0.748920 -0.740960 \\ 
		0.748880 -0.740970 \\ 
		0.745500 -0.741790 \\ 
		0.745450 -0.741800 \\ 
		0.681740 -0.756490 \\ 
		0.679530 -0.756980 \\ 
		0.677550 -0.757510 \\ 
		0.677510 -0.757520 \\ 
		0.674130 -0.758340 \\ 
		0.674080 -0.758350 \\ 
		0.670580 -0.759130 \\ 
		0.670520 -0.759140 \\ 
		0.604650 -0.773000 \\ 
		0.602380 -0.773460 \\ 
		0.600310 -0.773960 \\ 
		0.600260 -0.773980 \\ 
		0.596760 -0.774750 \\ 
		0.596710 -0.774760 \\ 
		0.593100 -0.775500 \\ 
		0.593040 -0.775510 \\ 
		0.525060 -0.788500 \\ 
		0.522670 -0.788940 \\ 
		0.520590 -0.789400 \\ 
		0.520540 -0.789410 \\ 
		0.516920 -0.790140 \\ 
		0.516870 -0.790150 \\ 
		0.513140 -0.790840 \\ 
		0.513080 -0.790850 \\ 
		0.443080 -0.802930 \\ 
		0.440710 -0.803320 \\ 
		0.438470 -0.803780 \\ 
		0.438420 -0.803790 \\ 
		0.434690 -0.804470 \\ 
		0.434630 -0.804480 \\ 
		0.430800 -0.805110 \\ 
		0.430740 -0.805120 \\ 
		0.358780 -0.816260 \\ 
		0.356330 -0.816620 \\ 
		0.354040 -0.817040 \\ 
		0.353980 -0.817050 \\ 
		0.350150 -0.817680 \\ 
		0.350090 -0.817690 \\ 
		0.346150 -0.818270 \\ 
		0.346090 -0.818280 \\ 
		0.272270 -0.828430 \\ 
		0.269740 -0.828760 \\ 
		0.267400 -0.829150 \\ 
		0.267350 -0.829150 \\ 
		0.263410 -0.829740 \\ 
		0.263350 -0.829740 \\ 
		0.259310 -0.830270 \\ 
		0.259250 -0.830280 \\ 
		0.183650 -0.839400 \\ 
		0.181090 -0.839690 \\ 
		0.178660 -0.840050 \\ 
		0.178600 -0.840060 \\ 
		0.174560 -0.840590 \\ 
		0.174500 -0.840590 \\ 
		0.170370 -0.841060 \\ 
		0.170310 -0.841070 \\ 
		0.093030 -0.849140 \\ 
		0.090420 -0.849390 \\ 
		0.087920 -0.849720 \\ 
		0.087860 -0.849720 \\ 
		0.083730 -0.850190 \\ 
		0.083670 -0.850200 \\ 
		0.079450 -0.850610 \\ 
		0.079380 -0.850620 \\ 
		0.000530 -0.857590 \\ 
		-0.002140 -0.857800 \\ 
		-0.004690 -0.858090 \\ 
		-0.004760 -0.858100 \\ 
		-0.008980 -0.858510 \\ 
		-0.009040 -0.858520 \\ 
		-0.013340 -0.858870 \\ 
		-0.013410 -0.858880 \\ 
		-0.093750 -0.864710 \\ 
		-0.096460 -0.864890 \\ 
		-0.099070 -0.865150 \\ 
		-0.099130 -0.865150 \\ 
		-0.103440 -0.865510 \\ 
		-0.103500 -0.865510 \\ 
		-0.107880 -0.865800 \\ 
		-0.107950 -0.865800 \\ 
		-0.189660 -0.870480 \\ 
		-0.192410 -0.870620 \\ 
		-0.195080 -0.870840 \\ 
		-0.195140 -0.870840 \\ 
		-0.199520 -0.871130 \\ 
		-0.199590 -0.871140 \\ 
		-0.204040 -0.871360 \\ 
		-0.204110 -0.871370 \\ 
		-0.287080 -0.874840 \\ 
		-0.289870 -0.874940 \\ 
		-0.292580 -0.875120 \\ 
		-0.292650 -0.875130 \\ 
		-0.297100 -0.875350 \\ 
		-0.297170 -0.875360 \\ 
		-0.301690 -0.875520 \\ 
		-0.301760 -0.875520 \\ 
		-0.385870 -0.877780 \\ 
		-0.388700 -0.877830 \\ 
		-0.391460 -0.877970 \\ 
		-0.391520 -0.877980 \\ 
		-0.396040 -0.878140 \\ 
		-0.396110 -0.878140 \\ 
		-0.400690 -0.878230 \\ 
		-0.400750 -0.878240 \\ 
		-0.485880 -0.879240 \\ 
		-0.488750 -0.879250 \\ 
		-0.491550 -0.879350 \\ 
		-0.491610 -0.879360 \\ 
		-0.496190 -0.879450 \\ 
		-0.496260 -0.879450 \\ 
		-0.500890 -0.879480 \\ 
		-0.501020 -0.879480 \\ 
		-0.587040 -0.879200 \\ 
		-0.589890 -0.879170 \\ 
		-0.592710 -0.879230 \\ 
		-0.592780 -0.879230 \\ 
		-0.597410 -0.879260 \\ 
		-0.597480 -0.879260 \\ 
		-0.597530 -0.879260 \\ 
		-0.602210 -0.879210 \\ 
		-0.602280 -0.879210 \\ 
		-0.689070 -0.877630 \\ 
		-0.691970 -0.877560 \\ 
		-0.694800 -0.877570 \\ 
		-0.694860 -0.877570 \\ 
		-0.694920 -0.877570 \\ 
		-0.699600 -0.877530 \\ 
		-0.699670 -0.877530 \\ 
		-0.704380 -0.877410 \\ 
		-0.704450 -0.877410 \\ 
		-0.791870 -0.874500 \\ 
		-0.794800 -0.874380 \\ 
		-0.797700 -0.874350 \\ 
		-0.797770 -0.874350 \\ 
		-0.802480 -0.874230 \\ 
		-0.802550 -0.874230 \\ 
		-0.807300 -0.874040 \\ 
		-0.807370 -0.874040 \\ 
		-0.895290 -0.869780 \\ 
		-0.898230 -0.869620 \\ 
		-0.901150 -0.869540 \\ 
		-0.901220 -0.869540 \\ 
		-0.905970 -0.869350 \\ 
		-0.906040 -0.869350 \\ 
		-0.910810 -0.869090 \\ 
		-0.910880 -0.869090 \\ 
		-0.999150 -0.863450 \\ 
		-1.002100 -0.863240 \\ 
		-1.005040 -0.863130 \\ 
		-1.005110 -0.863120 \\ 
		-1.009880 -0.862860 \\ 
		-1.009950 -0.862860 \\ 
		-1.014740 -0.862520 \\ 
		-1.014810 -0.862520 \\ 
		-1.103290 -0.855490 \\ 
		-1.106250 -0.855240 \\ 
		-1.109200 -0.855080 \\ 
		-1.109270 -0.855070 \\ 
		-1.114060 -0.854730 \\ 
		-1.114130 -0.854730 \\ 
		-1.118920 -0.854320 \\ 
		-1.118990 -0.854310 \\ 
		-1.207530 -0.845880 \\ 
		-1.210490 -0.845580 \\ 
		-1.213460 -0.845370 \\ 
		-1.213530 -0.845370 \\ 
		-1.218320 -0.844950 \\ 
		-1.218390 -0.844950 \\ 
		-1.223180 -0.844460 \\ 
		-1.223250 -0.844450 \\ 
		-1.311700 -0.834600 \\ 
		-1.314660 -0.834250 \\ 
		-1.317630 -0.834000 \\ 
		-1.317700 -0.833990 \\ 
		-1.322490 -0.833500 \\ 
		-1.322560 -0.833500 \\ 
		-1.327350 -0.832930 \\ 
		-1.327420 -0.832920 \\ 
		-1.415630 -0.821640 \\ 
		-1.418570 -0.821250 \\ 
		-1.421540 -0.820940 \\ 
		-1.421610 -0.820940 \\ 
		-1.426390 -0.820370 \\ 
		-1.426470 -0.820360 \\ 
		-1.431230 -0.819720 \\ 
		-1.431300 -0.819710 \\ 
		-1.519110 -0.806990 \\ 
		-1.522050 -0.806540 \\ 
		-1.525000 -0.806190 \\ 
		-1.525080 -0.806190 \\ 
		-1.529840 -0.805540 \\ 
		-1.529910 -0.805530 \\ 
		-1.534660 -0.804810 \\ 
		-1.534730 -0.804800 \\ 
		-1.621980 -0.790630 \\ 
		-1.624880 -0.790140 \\ 
		-1.627840 -0.789740 \\ 
		-1.627910 -0.789730 \\ 
		-1.632650 -0.789010 \\ 
		-1.632720 -0.789000 \\ 
		-1.637430 -0.788200 \\ 
		-1.637500 -0.788190 \\ 
		-1.724030 -0.772570 \\ 
		-1.726920 -0.772020 \\ 
		-1.729850 -0.771580 \\ 
		-1.729920 -0.771570 \\ 
		-1.734630 -0.770770 \\ 
		-1.734700 -0.770760 \\ 
		-1.739360 -0.769880 \\ 
		-1.739430 -0.769860 \\ 
		-1.825080 -0.752790 \\ 
		-1.827930 -0.752200 \\ 
		-1.830850 -0.751700 \\ 
		-1.830920 -0.751690 \\ 
		-1.835580 -0.750810 \\ 
		-1.835650 -0.750800 \\ 
		-1.840270 -0.749840 \\ 
		-1.840330 -0.749830 \\ 
		-1.924930 -0.731300 \\ 
		-1.927750 -0.730660 \\ 
		-1.930640 -0.730110 \\ 
		-1.930710 -0.730100 \\ 
		-1.935320 -0.729140 \\ 
		-1.935390 -0.729130 \\ 
		-1.939940 -0.728090 \\ 
		-1.940010 -0.728080 \\ 
		-2.023400 -0.708090 \\ 
		-2.026170 -0.707400 \\ 
		-2.029020 -0.706810 \\ 
		-2.029090 -0.706800 \\ 
		-2.033650 -0.705760 \\ 
		-2.033710 -0.705750 \\ 
		-2.038200 -0.704630 \\ 
		-2.038270 -0.704620 \\ 
		-2.120270 -0.683190 \\ 
		-2.122990 -0.682450 \\ 
		-2.125810 -0.681810 \\ 
		-2.125880 -0.681790 \\ 
		-2.130360 -0.680680 \\ 
		-2.130430 -0.680660 \\ 
		-2.134840 -0.679470 \\ 
		-2.134900 -0.679450 \\ 
		-2.215360 -0.656580 \\ 
		-2.218030 -0.655790 \\ 
		-2.220800 -0.655100 \\ 
		-2.220870 -0.655090 \\ 
		-2.225270 -0.653890 \\ 
		-2.225340 -0.653880 \\ 
		-2.229660 -0.652600 \\ 
		-2.229720 -0.652590 \\ 
		-2.308450 -0.628290 \\ 
		-2.311070 -0.627450 \\ 
		-2.313790 -0.626710 \\ 
		-2.313860 -0.626700 \\ 
		-2.318170 -0.625430 \\ 
		-2.318240 -0.625410 \\ 
		-2.322460 -0.624060 \\ 
		-2.322520 -0.624040 \\ 
		-2.399370 -0.598320 \\ 
		-2.401910 -0.597440 \\ 
		-2.404580 -0.596650 \\ 
		-2.404650 -0.596640 \\ 
		-2.408870 -0.595290 \\ 
		-2.408930 -0.595270 \\ 
		-2.413050 -0.593840 \\ 
		-2.413110 -0.593820 \\ 
		-2.487890 -0.566700 \\ 
		-2.490360 -0.565780 \\ 
		-2.492980 -0.564940 \\ 
		-2.493040 -0.564920 \\ 
		-2.497150 -0.563500 \\ 
		-2.497220 -0.563480 \\ 
		-2.501220 -0.561980 \\ 
		-2.501270 -0.561950 \\ 
		-2.573830 -0.533450 \\ 
		-2.573890 -0.533430 \\ 
		-2.573940 -0.533400 \\ 
		-2.574000 -0.533380 \\ 
		-2.574050 -0.533350 \\ 
		-2.574100 -0.533320 \\ 
		-2.574540 -0.533060 \\ 
		-2.574590 -0.533040 \\ 
		-2.574660 -0.533020 \\ 
		-2.578770 -0.531600 \\ 
		-2.578830 -0.531580 \\ 
		-2.582830 -0.530080 \\ 
		-2.582890 -0.530050 \\ 
		-2.582950 -0.530030 \\ 
		-2.653100 -0.500170 \\ 
		-2.653160 -0.500140 \\ 
		-2.653210 -0.500120 \\ 
		-2.653270 -0.500090 \\ 
		-2.653310 -0.500060 \\ 
		-2.653360 -0.500030 \\ 
		-2.654550 -0.499260 \\ 
		-2.657710 -0.498170 \\ 
		-2.657770 -0.498140 \\ 
		-2.661770 -0.496640 \\ 
		-2.661830 -0.496620 \\ 
		-2.661880 -0.496600 \\ 
		-2.661940 -0.496570 \\ 
		-2.729530 -0.465370 \\ 
		-2.729580 -0.465350 \\ 
		-2.729630 -0.465320 \\ 
		-2.729680 -0.465290 \\ 
		-2.729730 -0.465260 \\ 
		-2.729770 -0.465230 \\ 
		-2.731060 -0.464310 \\ 
		-2.731100 -0.464280 \\ 
		-2.731390 -0.464060 \\ 
		-2.733770 -0.463230 \\ 
		-2.733830 -0.463210 \\ 
		-2.737830 -0.461710 \\ 
		-2.737890 -0.461690 \\ 
		-2.737950 -0.461660 \\ 
		-2.738000 -0.461640 \\ 
		-2.738060 -0.461610 \\ 
		-2.802910 -0.429100 \\ 
		-2.802960 -0.429070 \\ 
		-2.803010 -0.429040 \\ 
		-2.803050 -0.429010 \\ 
		-2.803100 -0.428980 \\ 
		-2.803140 -0.428950 \\ 
		-2.804340 -0.428000 \\ 
		-2.804380 -0.427970 \\ 
		-2.804990 -0.427440 \\ 
		-2.806770 -0.426820 \\ 
		-2.806840 -0.426800 \\ 
		-2.810840 -0.425300 \\ 
		-2.810890 -0.425280 \\ 
		-2.810950 -0.425250 \\ 
		-2.811010 -0.425230 \\ 
		-2.811060 -0.425200 \\ 
		-2.811110 -0.425180 \\ 
		-2.873060 -0.391380 \\ 
		-2.873110 -0.391350 \\ 
		-2.873160 -0.391320 \\ 
		-2.873200 -0.391290 \\ 
		-2.873240 -0.391260 \\ 
		-2.873280 -0.391230 \\ 
		-2.874390 -0.390250 \\ 
		-2.874430 -0.390220 \\ 
		-2.875240 -0.389410 \\ 
		-2.876530 -0.388970 \\ 
		-2.876590 -0.388950 \\ 
		-2.880590 -0.387450 \\ 
		-2.880650 -0.387420 \\ 
		-2.880710 -0.387400 \\ 
		-2.880760 -0.387380 \\ 
		-2.880820 -0.387350 \\ 
		-2.880870 -0.387320 \\ 
		-2.880920 -0.387290 \\ 
		-2.939800 -0.352250 \\ 
		-2.939840 -0.352220 \\ 
		-2.939890 -0.352190 \\ 
		-2.939930 -0.352160 \\ 
		-2.939970 -0.352120 \\ 
		-2.940000 -0.352090 \\ 
		-2.941020 -0.351090 \\ 
		-2.941050 -0.351050 \\ 
		-2.941970 -0.350030 \\ 
		-2.941990 -0.350010 \\ 
		-2.942860 -0.349710 \\ 
		-2.942920 -0.349690 \\ 
		-2.946920 -0.348190 \\ 
		-2.946980 -0.348160 \\ 
		-2.947030 -0.348140 \\ 
		-2.947090 -0.348120 \\ 
		-2.947140 -0.348090 \\ 
		-2.947190 -0.348060 \\ 
		-2.947240 -0.348040 \\ 
		-2.947290 -0.348010 \\ 
		-3.002940 -0.311740 \\ 
		-3.002980 -0.311710 \\ 
		-3.003020 -0.311680 \\ 
		-3.003060 -0.311650 \\ 
		-3.003090 -0.311620 \\ 
		-3.003130 -0.311580 \\ 
		-3.004050 -0.310550 \\ 
		-3.004080 -0.310520 \\ 
		-3.004890 -0.309470 \\ 
		-3.004920 -0.309430 \\ 
		-3.005030 -0.309270 \\ 
		-3.005560 -0.309080 \\ 
		-3.005630 -0.309060 \\ 
		-3.009630 -0.307560 \\ 
		-3.009680 -0.307540 \\ 
		-3.009740 -0.307510 \\ 
		-3.009800 -0.307490 \\ 
		-3.009850 -0.307460 \\ 
		-3.009900 -0.307440 \\ 
		-3.009950 -0.307410 \\ 
		-3.010000 -0.307380 \\ 
		-3.010040 -0.307350 \\ 
		-3.062300 -0.269910 \\ 
		-3.062340 -0.269880 \\ 
		-3.062380 -0.269850 \\ 
		-3.062410 -0.269810 \\ 
		-3.062440 -0.269780 \\ 
		-3.062470 -0.269750 \\ 
		-3.063290 -0.268690 \\ 
		-3.063320 -0.268660 \\ 
		-3.064030 -0.267580 \\ 
		-3.064050 -0.267550 \\ 
		-3.064230 -0.267210 \\ 
		-3.064480 -0.267130 \\ 
		-3.064540 -0.267110 \\ 
		-3.068540 -0.265610 \\ 
		-3.068600 -0.265590 \\ 
		-3.068660 -0.265560 \\ 
		-3.068710 -0.265540 \\ 
		-3.068760 -0.265510 \\ 
		-3.068820 -0.265480 \\ 
		-3.068860 -0.265460 \\ 
		-3.068910 -0.265430 \\ 
		-3.068950 -0.265400 \\ 
		-3.068990 -0.265370 \\ 
		-3.117700 -0.226790 \\ 
		-3.117740 -0.226760 \\ 
		-3.117770 -0.226720 \\ 
		-3.117810 -0.226690 \\ 
		-3.117830 -0.226660 \\ 
		-3.117860 -0.226620 \\ 
		-3.118570 -0.225550 \\ 
		-3.118590 -0.225510 \\ 
		-3.119200 -0.224410 \\ 
		-3.119210 -0.224380 \\ 
		-3.119430 -0.223890 \\ 
		-3.119480 -0.223880 \\ 
		-3.123480 -0.222380 \\ 
		-3.123540 -0.222350 \\ 
		-3.123590 -0.222330 \\ 
		-3.123650 -0.222310 \\ 
		-3.123700 -0.222280 \\ 
		-3.123750 -0.222250 \\ 
		-3.123800 -0.222220 \\ 
		-3.123850 -0.222200 \\ 
		-3.123890 -0.222170 \\ 
		-3.123930 -0.222130 \\ 
		-3.123970 -0.222100 \\ 
		-3.168970 -0.182430 \\ 
		-3.169010 -0.182400 \\ 
		-3.169040 -0.182360 \\ 
		-3.169070 -0.182330 \\ 
		-3.169090 -0.182290 \\ 
		-3.169110 -0.182260 \\ 
		-3.169720 -0.181160 \\ 
		-3.169740 -0.181130 \\ 
		-3.170230 -0.180010 \\ 
		-3.170240 -0.179980 \\ 
		-3.170450 -0.179340 \\ 
		-3.174270 -0.177910 \\ 
		-3.174330 -0.177890 \\ 
		-3.174380 -0.177870 \\ 
		-3.174440 -0.177840 \\ 
		-3.174490 -0.177820 \\ 
		-3.174540 -0.177790 \\ 
		-3.174590 -0.177760 \\ 
		-3.174640 -0.177730 \\ 
		-3.174680 -0.177700 \\ 
		-3.174720 -0.177670 \\ 
		-3.174760 -0.177640 \\ 
		-3.174790 -0.177610 \\ 
		-3.215940 -0.136880 \\ 
		-3.215970 -0.136850 \\ 
		-3.216000 -0.136820 \\ 
		-3.216030 -0.136780 \\ 
		-3.216050 -0.136750 \\ 
		-3.216060 -0.136710 \\ 
		-3.216560 -0.135600 \\ 
		-3.216570 -0.135560 \\ 
		-3.216940 -0.134430 \\ 
		-3.216950 -0.134390 \\ 
		-3.217130 -0.133620 \\ 
		-3.220740 -0.132270 \\ 
		-3.220800 -0.132250 \\ 
		-3.220850 -0.132220 \\ 
		-3.220910 -0.132200 \\ 
		-3.220960 -0.132170 \\ 
		-3.221010 -0.132140 \\ 
		-3.221060 -0.132120 \\ 
		-3.221110 -0.132090 \\ 
		-3.221150 -0.132060 \\ 
		-3.221190 -0.132030 \\ 
		-3.221230 -0.132000 \\ 
		-3.221270 -0.131960 \\ 
		-3.221300 -0.131930 \\ 
		-3.258440 -0.090210 \\ 
		-3.258460 -0.090170 \\ 
		-3.258490 -0.090140 \\ 
		-3.258510 -0.090100 \\ 
		-3.258530 -0.090070 \\ 
		-3.258540 -0.090030 \\ 
		-3.258920 -0.088900 \\ 
		-3.258930 -0.088860 \\ 
		-3.259180 -0.087710 \\ 
		-3.259190 -0.087670 \\ 
		-3.259290 -0.086780 \\ 
		-3.262720 -0.085500 \\ 
		-3.262780 -0.085480 \\ 
		-3.262840 -0.085450 \\ 
		-3.262890 -0.085430 \\ 
		-3.262950 -0.085400 \\ 
		-3.263000 -0.085380 \\ 
		-3.263050 -0.085350 \\ 
		-3.263090 -0.085320 \\ 
		-3.263140 -0.085290 \\ 
		-3.263180 -0.085260 \\ 
		-3.263210 -0.085230 \\ 
		-3.263250 -0.085190 \\ 
		-3.263280 -0.085160 \\ 
		-3.263310 -0.085130 \\ 
		-3.296300 -0.042460 \\ 
		-3.296320 -0.042420 \\ 
		-3.296340 -0.042390 \\ 
		-3.296360 -0.042350 \\ 
		-3.296370 -0.042310 \\ 
		-3.296380 -0.042270 \\ 
		-3.296640 -0.041130 \\ 
		-3.296650 -0.041090 \\ 
		-3.296780 -0.039920 \\ 
		-3.296790 -0.039890 \\ 
		-3.296800 -0.038880 \\ 
		-3.300060 -0.037660 \\ 
		-3.300120 -0.037640 \\ 
		-3.300170 -0.037610 \\ 
		-3.300230 -0.037590 \\ 
		-3.300280 -0.037560 \\ 
		-3.300330 -0.037540 \\ 
		-3.300380 -0.037510 \\ 
		-3.300430 -0.037480 \\ 
		-3.300470 -0.037450 \\ 
		-3.300510 -0.037420 \\ 
		-3.300550 -0.037390 \\ 
		-3.300580 -0.037360 \\ 
		-3.300610 -0.037320 \\ 
		-3.300640 -0.037290 \\ 
		-3.300670 -0.037250 \\ 
		-3.329360 0.006310 \\ 
		-3.329390 0.006350 \\ 
		-3.329400 0.006380 \\ 
		-3.329420 0.006420 \\ 
		-3.329430 0.006460 \\ 
		-3.329430 0.006500 \\ 
		-3.329570 0.007660 \\ 
		-3.329570 0.007700 \\ 
		-3.329580 0.008870 \\ 
		-3.329580 0.008910 \\ 
		-3.329470 0.010020 \\ 
		-3.332580 0.011190 \\ 
		-3.332640 0.011210 \\ 
		-3.332700 0.011230 \\ 
		-3.332750 0.011260 \\ 
		-3.332810 0.011280 \\ 
		-3.332860 0.011310 \\ 
		-3.332910 0.011340 \\ 
		-3.332950 0.011360 \\ 
		-3.333000 0.011390 \\ 
		-3.333040 0.011430 \\ 
		-3.333070 0.011460 \\ 
		-3.333110 0.011490 \\ 
		-3.333140 0.011520 \\ 
		-3.333170 0.011560 \\ 
		-3.333190 0.011590 \\ 
		-3.333210 0.011630 \\ 
		-3.357480 0.056030 \\ 
		-3.357500 0.056070 \\ 
		-3.357520 0.056100 \\ 
		-3.357530 0.056140 \\ 
		-3.357530 0.056180 \\ 
		-3.357530 0.056220 \\ 
		-3.357550 0.057390 \\ 
		-3.357540 0.057430 \\ 
		-3.357430 0.058620 \\ 
		-3.357420 0.058660 \\ 
		-3.357190 0.059820 \\ 
		-3.359730 0.060810 \\ 
		-3.360210 0.061000 \\ 
		-3.360260 0.061020 \\ 
		-3.360320 0.061050 \\ 
		-3.360370 0.061070 \\ 
		-3.360420 0.061100 \\ 
		-3.360470 0.061130 \\ 
		-3.360520 0.061150 \\ 
		-3.360560 0.061180 \\ 
		-3.360600 0.061220 \\ 
		-3.360640 0.061250 \\ 
		-3.360670 0.061280 \\ 
		-3.360710 0.061310 \\ 
		-3.360730 0.061350 \\ 
		-3.360760 0.061380 \\ 
		-3.360780 0.061420 \\ 
		-3.360800 0.061450 \\ 
		-3.380510 0.106630 \\ 
		-3.380520 0.106670 \\ 
		-3.380530 0.106710 \\ 
		-3.380540 0.106750 \\ 
		-3.380540 0.106780 \\ 
		-3.380540 0.106820 \\ 
		-3.380420 0.108010 \\ 
		-3.380420 0.108050 \\ 
		-3.380170 0.109250 \\ 
		-3.380160 0.109280 \\ 
		-3.379800 0.110450 \\ 
		-3.382270 0.111490 \\ 
		-3.382720 0.111690 \\ 
		-3.382780 0.111710 \\ 
		-3.382830 0.111740 \\ 
		-3.382880 0.111770 \\ 
		-3.382930 0.111790 \\ 
		-3.382980 0.111820 \\ 
		-3.383020 0.111850 \\ 
		-3.383060 0.111880 \\ 
		-3.383100 0.111910 \\ 
		-3.383130 0.111950 \\ 
		-3.383160 0.111980 \\ 
		-3.383190 0.112010 \\ 
		-3.383220 0.112050 \\ 
		-3.383240 0.112090 \\ 
		-3.383250 0.112120 \\ 
		-3.383270 0.112160 \\ 
		-3.398300 0.158050 \\ 
		-3.398310 0.158090 \\ 
		-3.398310 0.158130 \\ 
		-3.398310 0.158170 \\ 
		-3.398310 0.158210 \\ 
		-3.398310 0.158250 \\ 
		-3.398060 0.159440 \\ 
		-3.398050 0.159480 \\ 
		-3.397680 0.160680 \\ 
		-3.397660 0.160720 \\ 
		-3.397170 0.161900 \\ 
		-3.399550 0.162980 \\ 
		-3.399990 0.163190 \\ 
		-3.400040 0.163220 \\ 
		-3.400090 0.163240 \\ 
		-3.400140 0.163270 \\ 
		-3.400180 0.163300 \\ 
		-3.400230 0.163330 \\ 
		-3.400270 0.163360 \\ 
		-3.400300 0.163390 \\ 
		-3.400340 0.163420 \\ 
		-3.400370 0.163460 \\ 
		-3.400400 0.163490 \\ 
		-3.400420 0.163530 \\ 
		-3.400440 0.163560 \\ 
		-3.400460 0.163600 \\ 
		-3.400480 0.163630 \\ 
		-3.400490 0.163670 \\ 
		-3.410710 0.210220 \\ 
		-3.410720 0.210260 \\ 
		-3.410720 0.210290 \\ 
		-3.410720 0.210330 \\ 
		-3.410710 0.210370 \\ 
		-3.410700 0.210410 \\ 
		-3.410320 0.211620 \\ 
		-3.410310 0.211660 \\ 
		-3.409800 0.212860 \\ 
		-3.409780 0.212900 \\ 
		-3.409160 0.214080 \\ 
		-3.411440 0.215210 \\ 
		-3.411860 0.215430 \\ 
		-3.411910 0.215450 \\ 
		-3.411960 0.215480 \\ 
		-3.412010 0.215510 \\ 
		-3.412050 0.215540 \\ 
		-3.412090 0.215570 \\ 
		-3.412130 0.215600 \\ 
		-3.412160 0.215630 \\ 
		-3.412190 0.215670 \\ 
		-3.412220 0.215700 \\ 
		-3.412250 0.215740 \\ 
		-3.412270 0.215770 \\ 
		-3.412280 0.215810 \\ 
		-3.412300 0.215850 \\ 
		-3.412310 0.215880 \\ 
		-3.412310 0.215920 \\ 
		-3.417620 0.263050 \\ 
		-3.417620 0.263090 \\ 
		-3.417620 0.263130 \\ 
		-3.417610 0.263170 \\ 
		-3.417600 0.263210 \\ 
		-3.417590 0.263250 \\ 
		-3.417080 0.264460 \\ 
		-3.417060 0.264500 \\ 
		-3.416420 0.265710 \\ 
		-3.416390 0.265750 \\ 
		-3.415640 0.266930 \\ 
		-3.417820 0.268100 \\ 
		-3.418220 0.268330 \\ 
		-3.418270 0.268360 \\ 
		-3.418310 0.268380 \\ 
		-3.418360 0.268410 \\ 
		-3.418400 0.268440 \\ 
		-3.418430 0.268480 \\ 
		-3.418470 0.268510 \\ 
		-3.418500 0.268540 \\ 
		-3.418530 0.268580 \\ 
		-3.418550 0.268610 \\ 
		-3.418570 0.268650 \\ 
		-3.418590 0.268680 \\ 
		-3.418610 0.268720 \\ 
		-3.418620 0.268760 \\ 
		-3.418620 0.268800 \\ 
		-3.418620 0.268830 \\ 
		-3.418900 0.316480 \\ 
		-3.418900 0.316520 \\ 
		-3.418890 0.316560 \\ 
		-3.418880 0.316600 \\ 
		-3.418870 0.316640 \\ 
		-3.418850 0.316680 \\ 
		-3.418210 0.317890 \\ 
		-3.418180 0.317930 \\ 
		-3.417400 0.319140 \\ 
		-3.417370 0.319180 \\ 
		-3.416480 0.320360 \\ 
		-3.418560 0.321580 \\ 
		-3.418940 0.321810 \\ 
		-3.418990 0.321840 \\ 
		-3.419030 0.321870 \\ 
		-3.419070 0.321900 \\ 
		-3.419110 0.321930 \\ 
		-3.419140 0.321970 \\ 
		-3.419170 0.322000 \\ 
		-3.419200 0.322030 \\ 
		-3.419220 0.322070 \\ 
		-3.419250 0.322100 \\ 
		-3.419260 0.322140 \\ 
		-3.419280 0.322180 \\ 
		-3.419290 0.322220 \\ 
		-3.419290 0.322250 \\ 
		-3.419290 0.322290 \\ 
		-3.419290 0.322330 \\ 
		-3.414440 0.370420 \\ 
		-3.414430 0.370460 \\ 
		-3.414420 0.370500 \\ 
		-3.414410 0.370540 \\ 
		-3.414390 0.370580 \\ 
		-3.414370 0.370620 \\ 
		-3.413590 0.371830 \\ 
		-3.413560 0.371870 \\ 
		-3.412640 0.373090 \\ 
		-3.412610 0.373130 \\ 
		-3.411580 0.374300 \\ 
		-3.413540 0.375560 \\ 
		-3.413900 0.375810 \\ 
		-3.413950 0.375840 \\ 
		-3.413990 0.375870 \\ 
		-3.414030 0.375900 \\ 
		-3.414060 0.375930 \\ 
		-3.414090 0.375960 \\ 
		-3.414120 0.376000 \\ 
		-3.414140 0.376030 \\ 
		-3.414160 0.376070 \\ 
		-3.414180 0.376100 \\ 
		-3.414200 0.376140 \\ 
		-3.414210 0.376180 \\ 
		-3.414210 0.376220 \\ 
		-3.414210 0.376250 \\ 
		-3.414210 0.376290 \\ 
		-3.414210 0.376330 \\ 
		-3.404130 0.424800 \\ 
		-3.404120 0.424830 \\ 
		-3.404100 0.424870 \\ 
		-3.404080 0.424910 \\ 
		-3.404060 0.424950 \\ 
		-3.404030 0.424990 \\ 
		-3.403110 0.426210 \\ 
		-3.403080 0.426250 \\ 
		-3.402030 0.427460 \\ 
		-3.401990 0.427500 \\ 
		-3.400820 0.428670 \\ 
		-3.402670 0.429970 \\ 
		-3.403010 0.430220 \\ 
		-3.403050 0.430250 \\ 
		-3.403090 0.430280 \\ 
		-3.403120 0.430310 \\ 
		-3.403150 0.430350 \\ 
		-3.403180 0.430380 \\ 
		-3.403200 0.430420 \\ 
		-3.403220 0.430450 \\ 
		-3.403240 0.430490 \\ 
		-3.403260 0.430520 \\ 
		-3.403270 0.430560 \\ 
		-3.403270 0.430600 \\ 
		-3.403270 0.430640 \\ 
		-3.403270 0.430680 \\ 
		-3.403270 0.430720 \\ 
		-3.403260 0.430750 \\ 
		-3.387860 0.479510 \\ 
		-3.387850 0.479550 \\ 
		-3.387830 0.479590 \\ 
		-3.387810 0.479630 \\ 
		-3.387780 0.479670 \\ 
		-3.387750 0.479710 \\ 
		-3.386690 0.480920 \\ 
		-3.386650 0.480960 \\ 
		-3.385460 0.482170 \\ 
		-3.385410 0.482210 \\ 
		-3.384110 0.483370 \\ 
		-3.385840 0.484710 \\ 
		-3.386150 0.484970 \\ 
		-3.386190 0.485000 \\ 
		-3.386220 0.485030 \\ 
		-3.386250 0.485070 \\ 
		-3.386280 0.485100 \\ 
		-3.386310 0.485140 \\ 
		-3.386330 0.485170 \\ 
		-3.386340 0.485210 \\ 
		-3.386360 0.485240 \\ 
		-3.386370 0.485280 \\ 
		-3.386370 0.485320 \\ 
		-3.386380 0.485360 \\ 
		-3.386370 0.485400 \\ 
		-3.386370 0.485440 \\ 
		-3.386360 0.485480 \\ 
		-3.386340 0.485520 \\ 
		-3.365560 0.534490 \\ 
		-3.365540 0.534530 \\ 
		-3.365510 0.534570 \\ 
		-3.365490 0.534610 \\ 
		-3.365450 0.534650 \\ 
		-3.365420 0.534690 \\ 
		-3.364220 0.535900 \\ 
		-3.364180 0.535940 \\ 
		-3.362840 0.537130 \\ 
		-3.362790 0.537170 \\ 
		-3.361360 0.538330 \\ 
		-3.362950 0.539710 \\ 
		-3.363240 0.539970 \\ 
		-3.363270 0.540010 \\ 
		-3.363300 0.540040 \\ 
		-3.363330 0.540070 \\ 
		-3.363360 0.540110 \\ 
		-3.363380 0.540140 \\ 
		-3.363390 0.540180 \\ 
		-3.363410 0.540220 \\ 
		-3.363420 0.540250 \\ 
		-3.363420 0.540290 \\ 
		-3.363430 0.540330 \\ 
		-3.363420 0.540370 \\ 
		-3.363420 0.540410 \\ 
		-3.363410 0.540450 \\ 
		-3.363390 0.540490 \\ 
		-3.363370 0.540530 \\ 
		-3.337120 0.589640 \\ 
		-3.337100 0.589680 \\ 
		-3.337070 0.589720 \\ 
		-3.337040 0.589760 \\ 
		-3.337000 0.589800 \\ 
		-3.336960 0.589840 \\ 
		-3.335620 0.591040 \\ 
		-3.335570 0.591080 \\ 
		-3.334090 0.592270 \\ 
		-3.334040 0.592300 \\ 
		-3.332470 0.593450 \\ 
		-3.333930 0.594870 \\ 
		-3.334190 0.595140 \\ 
		-3.334220 0.595170 \\ 
		-3.334250 0.595200 \\ 
		-3.334270 0.595240 \\ 
		-3.334300 0.595270 \\ 
		-3.334310 0.595310 \\ 
		-3.334330 0.595350 \\ 
		-3.334340 0.595380 \\ 
		-3.334340 0.595420 \\ 
		-3.334340 0.595460 \\ 
		-3.334340 0.595500 \\ 
		-3.334340 0.595540 \\ 
		-3.334330 0.595580 \\ 
		-3.334310 0.595620 \\ 
		-3.334290 0.595660 \\ 
		-3.334270 0.595700 \\ 
		-3.302480 0.644870 \\ 
		-3.302450 0.644910 \\ 
		-3.302420 0.644950 \\ 
		-3.302380 0.644990 \\ 
		-3.302340 0.645030 \\ 
		-3.302300 0.645060 \\ 
		-3.300820 0.646250 \\ 
		-3.300770 0.646290 \\ 
		-3.299150 0.647470 \\ 
		-3.299090 0.647510 \\ 
		-3.297380 0.648640 \\ 
		-3.298700 0.650090 \\ 
		-3.298930 0.650370 \\ 
		-3.298960 0.650400 \\ 
		-3.298990 0.650440 \\ 
		-3.299010 0.650470 \\ 
		-3.299020 0.650510 \\ 
		-3.299040 0.650550 \\ 
		-3.299050 0.650580 \\ 
		-3.299050 0.650620 \\ 
		-3.299060 0.650660 \\ 
		-3.299050 0.650700 \\ 
		-3.299050 0.650740 \\ 
		-3.299040 0.650780 \\ 
		-3.299020 0.650820 \\ 
		-3.299000 0.650860 \\ 
		-3.298980 0.650900 \\ 
		-3.298950 0.650940 \\ 
		-3.261580 0.700080 \\ 
		-3.261540 0.700120 \\ 
		-3.261510 0.700160 \\ 
		-3.261460 0.700200 \\ 
		-3.261420 0.700240 \\ 
		-3.261370 0.700280 \\ 
		-3.259750 0.701450 \\ 
		-3.259690 0.701490 \\ 
		-3.257930 0.702660 \\ 
		-3.257870 0.702700 \\ 
		-3.256010 0.703810 \\ 
		-3.257190 0.705300 \\ 
		-3.257400 0.705580 \\ 
		-3.257430 0.705620 \\ 
		-3.257450 0.705650 \\ 
		-3.257460 0.705690 \\ 
		-3.257480 0.705720 \\ 
		-3.257490 0.705760 \\ 
		-3.257490 0.705800 \\ 
		-3.257500 0.705840 \\ 
		-3.257490 0.705870 \\ 
		-3.257490 0.705910 \\ 
		-3.257480 0.705950 \\ 
		-3.257460 0.705990 \\ 
		-3.257440 0.706030 \\ 
		-3.257420 0.706070 \\ 
		-3.257390 0.706110 \\ 
		-3.257360 0.706150 \\ 
		-3.214340 0.755180 \\ 
		-3.214310 0.755220 \\ 
		-3.214260 0.755260 \\ 
		-3.214220 0.755300 \\ 
		-3.214170 0.755340 \\ 
		-3.214110 0.755380 \\ 
		-3.212350 0.756540 \\ 
		-3.212290 0.756580 \\ 
		-3.210380 0.757730 \\ 
		-3.208340 0.758860 \\ 
		-3.208330 0.758870 \\ 
		-3.209360 0.760380 \\ 
		-3.209540 0.760670 \\ 
		-3.209560 0.760710 \\ 
		-3.209580 0.760740 \\ 
		-3.209590 0.760780 \\ 
		-3.209600 0.760820 \\ 
		-3.209610 0.760850 \\ 
		-3.209610 0.760890 \\ 
		-3.209610 0.760930 \\ 
		-3.209600 0.760970 \\ 
		-3.209590 0.761010 \\ 
		-3.209570 0.761050 \\ 
		-3.209550 0.761090 \\ 
		-3.209530 0.761130 \\ 
		-3.209500 0.761170 \\ 
		-3.209470 0.761210 \\ 
		-3.209430 0.761250 \\ 
		-3.160740 0.810080 \\ 
		-3.160700 0.810120 \\ 
		-3.160650 0.810160 \\ 
		-3.160600 0.810200 \\ 
		-3.160540 0.810240 \\ 
		-3.160480 0.810270 \\ 
		-3.158580 0.811420 \\ 
		-3.156530 0.812550 \\ 
		-3.154350 0.813660 \\ 
		-3.154270 0.813700 \\ 
		-3.155150 0.815250 \\ 
		-3.155300 0.815550 \\ 
		-3.155320 0.815580 \\ 
		-3.155330 0.815620 \\ 
		-3.155340 0.815660 \\ 
		-3.155350 0.815690 \\ 
		-3.155350 0.815730 \\ 
		-3.155350 0.815770 \\ 
		-3.155340 0.815810 \\ 
		-3.155330 0.815850 \\ 
		-3.155310 0.815890 \\ 
		-3.155290 0.815930 \\ 
		-3.155270 0.815970 \\ 
		-3.155240 0.816010 \\ 
		-3.155210 0.816050 \\ 
		-3.155170 0.816090 \\ 
		-3.155130 0.816130 \\ 
		-3.100730 0.864670 \\ 
		-3.100680 0.864710 \\ 
		-3.100630 0.864750 \\ 
		-3.100570 0.864790 \\ 
		-3.100510 0.864830 \\ 
		-3.098470 0.865960 \\ 
		-3.096280 0.867070 \\ 
		-3.093950 0.868160 \\ 
		-3.093800 0.868220 \\ 
		-3.094530 0.869810 \\ 
		-3.094650 0.870110 \\ 
		-3.094660 0.870140 \\ 
		-3.094670 0.870180 \\ 
		-3.094680 0.870220 \\ 
		-3.094680 0.870260 \\ 
		-3.094680 0.870300 \\ 
		-3.094670 0.870330 \\ 
		-3.094660 0.870370 \\ 
		-3.094640 0.870410 \\ 
		-3.094630 0.870450 \\ 
		-3.094600 0.870490 \\ 
		-3.094570 0.870530 \\ 
		-3.094540 0.870570 \\ 
		-3.094500 0.870610 \\ 
		-3.094460 0.870650 \\ 
		-3.094410 0.870690 \\ 
		-3.034280 0.918860 \\ 
		-3.034230 0.918900 \\ 
		-3.034170 0.918940 \\ 
		-3.034110 0.918980 \\ 
		-3.031930 0.920090 \\ 
		-3.029600 0.921180 \\ 
		-3.027130 0.922240 \\ 
		-3.026910 0.922330 \\ 
		-3.027470 0.923950 \\ 
		-3.027560 0.924260 \\ 
		-3.027570 0.924290 \\ 
		-3.027570 0.924330 \\ 
		-3.027580 0.924370 \\ 
		-3.027570 0.924410 \\ 
		-3.027570 0.924440 \\ 
		-3.027560 0.924480 \\ 
		-3.027540 0.924520 \\ 
		-3.027520 0.924560 \\ 
		-3.027500 0.924600 \\ 
		-3.027470 0.924640 \\ 
		-3.027440 0.924680 \\ 
		-3.027400 0.924720 \\ 
		-3.027360 0.924760 \\ 
		-3.027310 0.924800 \\ 
		-3.027260 0.924840 \\ 
		-2.961380 0.972550 \\ 
		-2.961330 0.972580 \\ 
		-2.961270 0.972620 \\ 
		-2.958940 0.973710 \\ 
		-2.956470 0.974770 \\ 
		-2.953870 0.975810 \\ 
		-2.953570 0.975920 \\ 
		-2.953960 0.977580 \\ 
		-2.954020 0.977890 \\ 
		-2.954020 0.977920 \\ 
		-2.954030 0.977960 \\ 
		-2.954020 0.978000 \\ 
		-2.954020 0.978040 \\ 
		-2.954010 0.978080 \\ 
		-2.953990 0.978120 \\ 
		-2.953970 0.978160 \\ 
		-2.953950 0.978200 \\ 
		-2.953920 0.978240 \\ 
		-2.953890 0.978280 \\ 
		-2.953850 0.978320 \\ 
		-2.953810 0.978350 \\ 
		-2.953760 0.978390 \\ 
		-2.953710 0.978430 \\ 
		-2.953650 0.978470 \\ 
		-2.882030 1.025620 \\ 
		-2.881970 1.025660 \\ 
		-2.879500 1.026720 \\ 
		-2.876900 1.027760 \\ 
		-2.874160 1.028760 \\ 
		-2.873770 1.028890 \\ 
		-2.874000 1.030590 \\ 
		-2.874020 1.030900 \\ 
		-2.874020 1.030940 \\ 
		-2.874020 1.030970 \\ 
		-2.874010 1.031010 \\ 
		-2.874000 1.031050 \\ 
		-2.873990 1.031090 \\ 
		-2.873970 1.031130 \\ 
		-2.873950 1.031170 \\ 
		-2.873920 1.031210 \\ 
		-2.873880 1.031250 \\ 
		-2.873850 1.031290 \\ 
		-2.873810 1.031330 \\ 
		-2.873760 1.031370 \\ 
		-2.873710 1.031410 \\ 
		-2.873650 1.031450 \\ 
		-2.873590 1.031480 \\ 
		-2.796220 1.077980 \\ 
		-2.793620 1.079020 \\ 
		-2.790880 1.080020 \\ 
		-2.788010 1.081000 \\ 
		-2.787540 1.081140 \\ 
		-2.787580 1.082870 \\ 
		-2.787570 1.083190 \\ 
		-2.787570 1.083230 \\ 
		-2.787570 1.083270 \\ 
		-2.787560 1.083310 \\ 
		-2.787540 1.083340 \\ 
		-2.787520 1.083380 \\ 
		-2.787500 1.083420 \\ 
		-2.787470 1.083460 \\ 
		-2.787440 1.083500 \\ 
		-2.787400 1.083540 \\ 
		-2.787360 1.083580 \\ 
		-2.787310 1.083620 \\ 
		-2.787260 1.083660 \\ 
		-2.787200 1.083700 \\ 
		-2.787140 1.083740 \\ 
		-2.704050 1.129500 \\ 
		-2.701310 1.130500 \\ 
		-2.698440 1.131470 \\ 
		-2.695440 1.132410 \\ 
		-2.694880 1.132570 \\ 
		-2.694740 1.134340 \\ 
		-2.694700 1.134650 \\ 
		-2.694690 1.134690 \\ 
		-2.694680 1.134730 \\ 
		-2.694660 1.134770 \\ 
		-2.694650 1.134810 \\ 
		-2.694620 1.134850 \\ 
		-2.694590 1.134890 \\ 
		-2.694560 1.134930 \\ 
		-2.694520 1.134970 \\ 
		-2.694480 1.135010 \\ 
		-2.694440 1.135050 \\ 
		-2.694380 1.135090 \\ 
		-2.694330 1.135120 \\ 
		-2.694270 1.135160 \\ 
		-2.605480 1.180090 \\ 
		-2.602610 1.181060 \\ 
		-2.599600 1.182000 \\ 
		-2.596470 1.182900 \\ 
		-2.595840 1.183070 \\ 
		-2.595490 1.184870 \\ 
		-2.595420 1.185190 \\ 
		-2.595410 1.185230 \\ 
		-2.595390 1.185270 \\ 
		-2.595370 1.185310 \\ 
		-2.595350 1.185340 \\ 
		-2.595320 1.185380 \\ 
		-2.595290 1.185420 \\ 
		-2.595250 1.185460 \\ 
		-2.595210 1.185500 \\ 
		-2.595160 1.185540 \\ 
		-2.595110 1.185580 \\ 
		-2.595060 1.185620 \\ 
		-2.594990 1.185660 \\ 
		-2.500550 1.229650 \\ 
		-2.497540 1.230590 \\ 
		-2.494410 1.231490 \\ 
		-2.491150 1.232350 \\ 
		-2.490430 1.232530 \\ 
		-2.489890 1.234360 \\ 
		-2.489780 1.234680 \\ 
		-2.489760 1.234720 \\ 
		-2.489740 1.234760 \\ 
		-2.489720 1.234800 \\ 
		-2.489690 1.234840 \\ 
		-2.489660 1.234880 \\ 
		-2.489620 1.234920 \\ 
		-2.489580 1.234960 \\ 
		-2.489530 1.235000 \\ 
		-2.489480 1.235040 \\ 
		-2.489430 1.235080 \\ 
		-2.489370 1.235110 \\ 
		-2.389310 1.278090 \\ 
		-2.386170 1.278990 \\ 
		-2.382910 1.279850 \\ 
		-2.379530 1.280670 \\ 
		-2.378740 1.280850 \\ 
		-2.377980 1.282720 \\ 
		-2.377830 1.283040 \\ 
		-2.377810 1.283080 \\ 
		-2.377790 1.283120 \\ 
		-2.377760 1.283160 \\ 
		-2.377730 1.283200 \\ 
		-2.377690 1.283240 \\ 
		-2.377650 1.283270 \\ 
		-2.377600 1.283310 \\ 
		-2.377550 1.283350 \\ 
		-2.377490 1.283390 \\ 
		-2.377430 1.283430 \\ 
		-2.377430 1.283430 \\ 
		-2.271820 1.325280 \\ 
		-2.268560 1.326140 \\ 
		-2.265180 1.326970 \\ 
		-2.261680 1.327740 \\ 
		-2.260800 1.327920 \\ 
		-2.259820 1.329820 \\ 
		-2.259640 1.330140 \\ 
		-2.259620 1.330180 \\ 
		-2.259590 1.330220 \\ 
		-2.259560 1.330260 \\ 
		-2.259520 1.330300 \\ 
		-2.259480 1.330340 \\ 
		-2.259430 1.330380 \\ 
		-2.259380 1.330420 \\ 
		-2.259320 1.330460 \\ 
		-2.259260 1.330490 \\ 
		-2.259260 1.330500 \\ 
		-2.148160 1.371140 \\ 
		-2.144780 1.371960 \\ 
		-2.141280 1.372730 \\ 
		-2.137660 1.373470 \\ 
		-2.136720 1.373640 \\ 
		-2.135500 1.375570 \\ 
		-2.135280 1.375900 \\ 
		-2.135250 1.375940 \\ 
		-2.135220 1.375970 \\ 
		-2.135190 1.376010 \\ 
		-2.135140 1.376050 \\ 
		-2.135100 1.376090 \\ 
		-2.135050 1.376130 \\ 
		-2.134990 1.376170 \\ 
		-2.134930 1.376210 \\ 
		-2.134930 1.376210 \\ 
		-2.018430 1.415540 \\ 
		-2.014920 1.416320 \\ 
		-2.011310 1.417050 \\ 
		-2.007580 1.417730 \\ 
		-2.006560 1.417900 \\ 
		-2.005110 1.419860 \\ 
		-2.004850 1.420190 \\ 
		-2.004820 1.420220 \\ 
		-2.004780 1.420260 \\ 
		-2.004740 1.420300 \\ 
		-2.004690 1.420340 \\ 
		-2.004640 1.420380 \\ 
		-2.004590 1.420420 \\ 
		-2.004530 1.420460 \\ 
		-2.004520 1.420460 \\ 
		-1.882710 1.458390 \\ 
		-1.879090 1.459120 \\ 
		-1.875360 1.459800 \\ 
		-1.871520 1.460440 \\ 
		-1.870440 1.460600 \\ 
		-1.868740 1.462590 \\ 
		-1.868440 1.462910 \\ 
		-1.868400 1.462950 \\ 
		-1.868360 1.462990 \\ 
		-1.868310 1.463030 \\ 
		-1.868260 1.463060 \\ 
		-1.868210 1.463100 \\ 
		-1.868150 1.463140 \\ 
		-1.868140 1.463140 \\ 
		-1.741120 1.499580 \\ 
		-1.737390 1.500260 \\ 
		-1.733550 1.500900 \\ 
		-1.729610 1.501480 \\ 
		-1.728470 1.501630 \\ 
		-1.726500 1.503640 \\ 
		-1.726170 1.503960 \\ 
		-1.726130 1.504000 \\ 
		-1.726080 1.504040 \\ 
		-1.726030 1.504080 \\ 
		-1.725970 1.504120 \\ 
		-1.725910 1.504150 \\ 
		-1.725910 1.504160 \\ 
		-1.593780 1.539000 \\ 
		-1.589950 1.539640 \\ 
		-1.586000 1.540220 \\ 
		-1.581970 1.540750 \\ 
		-1.580770 1.540880 \\ 
		-1.578530 1.542920 \\ 
		-1.578160 1.543240 \\ 
		-1.578110 1.543280 \\ 
		-1.578060 1.543320 \\ 
		-1.578010 1.543350 \\ 
		-1.577950 1.543390 \\ 
		-1.577940 1.543390 \\ 
		-1.440840 1.576560 \\ 
		-1.436900 1.577140 \\ 
		-1.432860 1.577670 \\ 
		-1.428730 1.578140 \\ 
		-1.427480 1.578260 \\ 
		-1.424960 1.580320 \\ 
		-1.424550 1.580640 \\ 
		-1.424500 1.580680 \\ 
		-1.424450 1.580710 \\ 
		-1.424390 1.580750 \\ 
		-1.424380 1.580750 \\ 
		-1.282440 1.612150 \\ 
		-1.282440 1.612150 \\ 
		-1.278400 1.612680 \\ 
		-1.274270 1.613150 \\ 
		-1.270040 1.613560 \\ 
		-1.268760 1.613670 \\ 
		-1.265940 1.615740 \\ 
		-1.265500 1.616050 \\ 
		-1.265440 1.616090 \\ 
		-1.265380 1.616130 \\ 
		-1.265380 1.616130 \\ 
		-1.118750 1.645670 \\ 
		-1.118740 1.645670 \\ 
		-1.114610 1.646140 \\ 
		-1.110380 1.646550 \\ 
		-1.106080 1.646910 \\ 
		-1.104760 1.646990 \\ 
		-1.101640 1.649080 \\ 
		-1.101160 1.649390 \\ 
		-1.101100 1.649430 \\ 
		-1.101090 1.649430 \\ 
		-0.949920 1.677020 \\ 
		-0.949920 1.677020 \\ 
		-0.945700 1.677430 \\ 
		-0.941390 1.677780 \\ 
		-0.937010 1.678070 \\ 
		-0.935660 1.678140 \\ 
		-0.932230 1.680240 \\ 
		-0.931710 1.680540 \\ 
		-0.931700 1.680540 \\ 
		-0.931700 1.680540 \\ 
		-0.776170 1.706100 \\ 
		-0.776160 1.706100 \\ 
		-0.771850 1.706450 \\ 
		-0.767470 1.706740 \\ 
		-0.763020 1.706970 \\ 
		-0.761650 1.707020 \\ 
		-0.757960 1.709080 \\ 
		-0.757400 1.709380 \\ 
		-0.757400 1.709380 \\ 
		-0.757390 1.709380 \\ 
		-0.597670 1.732820 \\ 
		-0.597660 1.732820 \\ 
		-0.593280 1.733110 \\ 
		-0.588820 1.733340 \\ 
		-0.584300 1.733500 \\ 
		-0.582920 1.733530 \\ 
		-0.578980 1.735550 \\ 
		-0.578380 1.735850 \\ 
		-0.578370 1.735850 \\ 
		-0.578370 1.735850 \\ 
		-0.414640 1.757080 \\ 
		-0.414640 1.757080 \\ 
		-0.410180 1.757310 \\ 
		-0.405660 1.757470 \\ 
		-0.401080 1.757560 \\ 
		-0.399680 1.757570 \\ 
		-0.395480 1.759560 \\ 
		-0.394850 1.759840 \\ 
		-0.394850 1.759840 \\ 
		-0.394840 1.759850 \\ 
		-0.227310 1.778800 \\ 
		-0.227310 1.778800 \\ 
		-0.222790 1.778960 \\ 
		-0.218210 1.779050 \\ 
		-0.213580 1.779080 \\ 
		-0.212160 1.779060 \\ 
		-0.207710 1.781000 \\ 
		-0.207040 1.781280 \\ 
		-0.207040 1.781280 \\ 
		-0.207030 1.781290 \\ 
		-0.035920 1.797870 \\ 
		-0.035910 1.797870 \\ 
		-0.031330 1.797970 \\ 
		-0.026700 1.797990 \\ 
		-0.022020 1.797950 \\ 
		-0.020600 1.797910 \\ 
		-0.015890 1.799800 \\ 
		-0.015180 1.800080 \\ 
		-0.015180 1.800080 \\ 
		-0.015180 1.800080 \\ 
		0.159310 1.814230 \\ 
		0.159310 1.814230 \\ 
		0.163950 1.814250 \\ 
		0.168620 1.814210 \\ 
		0.173340 1.814090 \\ 
		0.174770 1.814040 \\ 
		0.179720 1.815870 \\ 
		0.180470 1.816140 \\ 
		0.180470 1.816140 \\ 
		0.180480 1.816140 \\ 
		0.358100 1.827780 \\ 
		0.358100 1.827780 \\ 
		0.362780 1.827730 \\ 
		0.367500 1.827620 \\ 
		0.372240 1.827430 \\ 
		0.373680 1.827350 \\ 
		0.378880 1.829130 \\ 
		0.379660 1.829380 \\ 
		0.379660 1.829380 \\ 
		0.379670 1.829380 \\ 
		0.560180 1.838440 \\ 
		0.560190 1.838440 \\ 
		0.564900 1.838320 \\ 
		0.569650 1.838140 \\ 
		0.574420 1.837870 \\ 
		0.575870 1.837770 \\ 
		0.581300 1.839490 \\ 
		0.582120 1.839740 \\ 
		0.582120 1.839740 \\ 
		0.582130 1.839740 \\ 
		0.765280 1.846140 \\ 
		0.765290 1.846140 \\ 
		0.770040 1.845950 \\ 
		0.774810 1.845690 \\ 
		0.779600 1.845350 \\ 
		0.781040 1.845230 \\ 
		0.786710 1.846880 \\ 
		0.787560 1.847120 \\ 
		0.787560 1.847120 \\ 
		0.787570 1.847120 \\ 
		0.973110 1.850810 \\ 
		0.973120 1.850810 \\ 
		0.977890 1.850540 \\ 
		0.982680 1.850210 \\ 
		0.987470 1.849790 \\ 
		0.988920 1.849650 \\ 
		0.994820 1.851230 \\ 
		0.995700 1.851450 \\ 
		0.995700 1.851450 \\ 
		0.995710 1.851450 \\ 
		1.183360 1.852370 \\ 
		1.183370 1.852370 \\ 
		1.188150 1.852030 \\ 
		1.192950 1.851620 \\ 
		1.197740 1.851130 \\ 
		1.199190 1.850960 \\ 
		1.205310 1.852460 \\ 
		1.206230 1.852680 \\ 
		1.206230 1.852680 \\ 
		1.206240 1.852680 \\ 
		1.395720 1.850760 \\ 
		1.395730 1.850760 \\ 
		1.400520 1.850340 \\ 
		1.405320 1.849860 \\ 
		1.410100 1.849290 \\ 
		1.411540 1.849090 \\ 
		1.417890 1.850520 \\ 
		1.418830 1.850720 \\ 
		1.418840 1.850720 \\ 
		1.418840 1.850720 \\ 
		1.609870 1.845920 \\ 
		1.609880 1.845910 \\ 
		1.614670 1.845430 \\ 
		1.619460 1.844860 \\ 
		1.624230 1.844220 \\ 
		1.625660 1.844000 \\ 
		1.632220 1.845340 \\ 
		1.633190 1.845530 \\ 
		1.633200 1.845530 \\ 
		1.633200 1.845530 \\ 
		1.825480 1.837790 \\ 
		1.825490 1.837780 \\ 
		1.830270 1.837220 \\ 
		1.835040 1.836580 \\ 
		1.839790 1.835850 \\ 
		1.841210 1.835610 \\ 
		1.847970 1.836870 \\ 
		1.848980 1.837040 \\ 
		1.848980 1.837050 \\ 
		1.848990 1.837050 \\ 
		2.042210 1.826310 \\ 
		2.042210 1.826310 \\ 
		2.046980 1.825670 \\ 
		2.051720 1.824950 \\ 
		2.056430 1.824150 \\ 
		2.057850 1.823880 \\ 
		2.064800 1.825050 \\ 
		2.065840 1.825210 \\ 
		2.065840 1.825210 \\ 
		2.065850 1.825210 \\ 
		2.259700 1.811460 \\ 
		2.259700 1.811460 \\ 
		2.264450 1.810730 \\ 
		2.269160 1.809930 \\ 
		2.273820 1.809050 \\ 
		2.275220 1.808760 \\ 
		2.282370 1.809830 \\ 
		2.283430 1.809980 \\ 
		2.283440 1.809980 \\ 
		2.283440 1.809980 \\ 
		2.477610 1.793160 \\ 
		2.482310 1.792360 \\ 
		2.486980 1.791490 \\ 
		2.491600 1.790530 \\ 
		2.492980 1.790210 \\ 
		2.500310 1.791190 \\ 
		2.501390 1.791320 \\ 
		2.501400 1.791320 \\ 
		2.501400 1.791320 \\ 
		2.695550 1.771400 \\ 
		2.700220 1.770530 \\ 
		2.704830 1.769570 \\ 
		2.709390 1.768530 \\ 
		2.710750 1.768190 \\ 
		2.718250 1.769060 \\ 
		2.719360 1.769180 \\ 
		2.719370 1.769180 \\ 
		2.719370 1.769180 \\ 
		2.913170 1.746140 \\ 
		2.917780 1.745180 \\ 
		2.922340 1.744150 \\ 
		2.926820 1.743030 \\ 
		2.928160 1.742670 \\ 
		2.935820 1.743430 \\ 
		2.936960 1.743540 \\ 
		2.936960 1.743540 \\ 
		2.936970 1.743540 \\ 
		3.130070 1.717350 \\ 
		3.134630 1.716310 \\ 
		3.139110 1.715200 \\ 
		3.143520 1.714000 \\ 
		3.144830 1.713620 \\ 
		3.152650 1.714270 \\ 
		3.153810 1.714360 \\ 
		3.153810 1.714360 \\ 
		3.153820 1.714360 \\ 
		3.345880 1.685010 \\ 
		3.350370 1.683890 \\ 
		3.354770 1.682700 \\ 
		3.359090 1.681430 \\ 
		3.360370 1.681020 \\ 
		3.368340 1.681560 \\ 
		3.369510 1.681630 \\ 
		3.369520 1.681630 \\ 
		3.560200 1.649090 \\ 
		3.564610 1.647900 \\ 
		3.568930 1.646630 \\ 
		3.573150 1.645280 \\ 
		3.574400 1.644850 \\ 
		3.582490 1.645280 \\ 
		3.583690 1.645330 \\ 
		3.583700 1.645330 \\ 
		3.772630 1.609600 \\ 
		3.776950 1.608330 \\ 
		3.781170 1.606990 \\ 
		3.785290 1.605560 \\ 
		3.786510 1.605100 \\ 
		3.794720 1.605410 \\ 
		3.795940 1.605450 \\ 
		3.795940 1.605450 \\ 
		3.982780 1.566530 \\ 
		3.987000 1.565180 \\ 
		3.991110 1.563760 \\ 
		3.995110 1.562260 \\ 
		3.995160 1.562230 \\ 
		3.995610 1.561760 \\ 
		4.004620 1.561970 \\ 
		4.005850 1.561990 \\ 
		4.005850 1.561990 \\ 
		4.190220 1.519880 \\ 
		4.194340 1.518450 \\ 
		4.198340 1.516950 \\ 
		4.198380 1.516930 \\ 
		4.200120 1.514850 \\ 
		4.211780 1.514940 \\ 
		4.213020 1.514940 \\ 
		4.213020 1.514940 \\ 
		4.394560 1.469650 \\ 
		4.398560 1.468150 \\ 
		4.398600 1.468120 \\ 
		4.402900 1.462190 \\ 
		4.418730 1.163800 \\ 
		4.418730 1.163350 \\ 
		4.418670 1.163310 \\ 
		4.418600 1.163290 \\ 
		4.402760 1.161680 \\ 
		3.607060 1.155740 \\ 
		3.605820 1.155740 \\ 
	};

	% initial output set
	\addplot[semithick,draw=black,fill=white] table[row sep=crcr] {
		3.60000 1.16250 \\
		3.60000 1.46250 \\
		4.40000 1.46250 \\
		4.40000 1.16250 \\
		3.60000 1.16250 \\
	};
\end{axis}

% support function
\draw[->,>=stealth'] (0,0) -- (0.3707,-0.9267)
	node[midway,xshift=0.1cm,yshift=0.15cm] {$\ell$};
%\node[anchor=east] at (0.1,-0.3) {$\widehat{y}(\ell)$};
% separating hyperplane
\draw[dashed] (5,0.924979) -- (-2.31245,-2);
	
\node[anchor=north west,xshift=-0.05cm,yshift=0.05cm] at (4.5,1.9) {$\mathcal{Y}(t_0)$};
\draw (4.5,1.7) edge[bend right, ->, >=stealth'] (4.2,1.3);

\node[anchor=east,xshift=0.05cm] at (-3.1,1.65) {$\widehat{\mathcal{Y}}([0,t_{\text{end}}])$};
\draw (-3.1,1.65) edge[bend left, ->, >=stealth'] (-2.5,1.35);

\node[anchor=west,xshift=-0.05cm] at (3.15,-1.5) {$\mathcal{L}_0 + (\widehat{\eta}(\ell) - s)$};
\draw (3.15,-1.5) edge[bend right, ->, >=stealth'] (2.2,-1.2);
\begin{scope}[xshift=1.14cm,yshift=-0.619cm]
	\draw[semithick,draw=unsafeborder,fill=unsafe,solid] (1,-1) rectangle (0,0);
	\filldraw (0.5,-0.5) circle(1pt) node[below right,xshift=-0.05cm,yshift=0.05cm] {$c$};
\end{scope}

\node[anchor=east] at (-1.05,-1.3)
	{$\big\{ y \, \big| \, \ell^\top y = \widehat{y}(\ell)\}$};

% % left part of the figure
% \begin{scope}[xshift=-6.7cm]
% 	\draw[thin,gray,step=1] (-1.25,-1.25) grid (1.25,1.25);
% 	\draw[->,>=stealth'] (-1.5,0) -- (1.5,0) node[above] {$y_1$};
% 	\draw[->,>=stealth'] (0,-1.5) -- (0,1.5) node[above] {$y_2$};
	
% 	\draw[unsafeset,solid] (-0.5,-0.5) rectangle (0.5,0.5);
% 	\filldraw (0,0) circle(1pt);
% 	\draw[dashed] (-1,0.3) -- (0.75,1);
% 	\draw[->,>=stealth'] (0,0) -- (-0.2414,0.6034)
% 		node[midway,xshift=0.25cm,yshift=0cm] {$-\ell$};
% 	\filldraw (-0.5,0.5) circle(1pt);
% 	\node at (-0.62,0.62) {$s$};
	
% 	\node[anchor=west,xshift=-0.05cm] at (0.6,-0.75) {$\mathcal{L}_0$};
% 	\draw (0.6,-0.75) edge[bend left, ->, >=stealth'] (0.2,-0.55);
	
% \end{scope}

% support vector
\filldraw (1.14,-0.619) circle(1pt);
\node at (1.1,-0.35) {$\widehat{y}(\ell)$};
	
\end{tikzpicture}

    \begin{small}
    Figure 1: The support vectors of the zero-centered unsafe set $\mathcal{L}_0$ and the output set $\widehat{\mathcal{Y}}([0,t_{\text{end}}])$ are used for placing unsafe sets.
    \end{small}
}

\noindent
Figures enable the simultaneous presentation of multiple pieces of information, unlike paragraphs, which require sequential reading and sustained focus from the reader.
When illustrating workflows, it is often helpful to break down each step into separate subfigures, allowing the full process to emerge from the sequence.
Such figures can be dense, as their purpose is to convey the overarching idea clearly and accurately—precise labeling is essential to achieve this.
