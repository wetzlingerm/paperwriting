
\guideline[g:problemstatement:preliminaries]
    {Consider moving the problem statement to the end of the preliminaries.}

\goodbadexample[{\cite[Sec.~II.B.]{Wetzlinger2024CSL}}]{
    \textit{III. Problem Statement} \\
    As the exact reachable set $\mathcal{R}(t)$ cannot be computed for general nonlinear systems [2], our goal is to compute a tight inner approximation $\widecheck{\mathcal{R}}(t) \subseteq \mathcal{R}(t)$ instead.
}{
    (...) \\
    As the exact reachable set $\mathcal{R}(t)$ cannot be computed for general nonlinear systems [2], our goal is to compute a tight inner approximation $\widecheck{\mathcal{R}}(t) \subseteq \mathcal{R}(t)$ instead.
}

\noindent 
Short problem statements sare best placed at the end of the preliminaries, as dedicating a full section may be disproportionate.
However, a separate section is appropriate if the statement requires additional definitions or detailed explanation.
