
\guideline[g:problemstatement:abstract]
    {Relate the problem statement to the research question stated in the abstract.}

\goodexample[{\cite[Abstract, Sec.~III.]{Wetzlinger2023TAC}}]{
    A substantial bottleneck of all reachability algorithms is the \highlightpart{necessity to adequately tune specific algorithm parameters}, such as the time step size, which requires expert knowledge.
    In this work, we solve this issue with a fully automated reachability algorithm that tunes all algorithm parameters internally such that the \highlightpart{reachable set enclosure respects a user-defined approximation error bound in terms of the Hausdorff distance to the exact reachable set.} \\
    (...) \\
    (...) by \highlightpart{automatically tuning these algorithm parameters} such that the \highlightpart{Hausdorff distance between the computed enclosure $\widehat{\mathcal{R}}(t)$ and the exact reachable set $\mathcal{R}(t)$ remains below a desired threshold $\varepsilon_{\max}$ at all times:} \\
    \begin{equation*}
        \text{Tune} \; \Delta t, \eta, \rho \; \text{s.t.} \; \forall t \in [0,t_{\text{end}}]\colon d_H(\mathcal{R}(t), \widehat{\mathcal{R}}(t)) \leq \varepsilon_{\max} .
    \end{equation*}
}

\noindent 
A clear link between the abstract and the problem statement is often best achieved by using similar wording.
Repeating key phrases is acceptable, as a well-defined problem statement is essential for understanding the main content of the paper.
