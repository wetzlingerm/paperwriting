
\chapter{Title, Abstract, and Keywords}
\label{ch:abstract}

The abstract provides a concise summary of the paper:
In order, it motivates and states the research question, highlights the limitations of existing methods, presents the main approach and its advantages, discusses the results of the evaluation, and provides a brief conclusion.
A well-written abstract helps readers quickly determine the paper's relevance to their interests.

We will use a running example for guidelines \ref{g:abstract:motivate_research_field} through \ref{g:abstract:summarize_results}.


\guideline[g:abstract:title]
    {Make each word in the title matter.}

\goodbadexample[{\cite[Title]{Wetzlinger2024CSL}}]{
    ``\highlightpart{Computing} Inner Approximations of Reachable Sets for Nonlinear \highlightpart{Dynamical} Systems \highlightempty{}''
}{
    ``Inner Approximations of Reachable Sets for Nonlinear Systems Using the Minkowski Difference''
}

\noindent The title summarizes the topic of the paper, and potentially even the applied methodology, into its densest form.
We want the title to be as short as possible, while being as informative as possible.
Avoid words that do not add much value (like ``computing'' in the above example) and use short versions where the loss of information is negligible (like ``nonlinear systems'' instead of ``nonlinear dynamical systems'').
Still, include bits of methodology that set your paper apart from existing work (in the above example, ``using the Minkowski difference'' was the core element of our novel approach).


\guideline{Set a broader frame, motivating the topic of the paper.}

\goodbadexample{
    ...
}{
    ...
}

\noindent ...
% - start out with a sentence that sets a frame for the topic of the paper, why it is important, and can be understood by a broad audience


\guideline[g:abstract:research_question]
    {State the research question addressed in your paper.}

\goodbadexample{
    ...
}{
    ...
}

\noindent ...
% ...should naturally follow from the introductory sentence by narrowing down the general frame


\guideline{...}
% - outline limitations of start-of-the-art approaches: specific question has not yet been addresses, approaches don't perform well, approach don't yield precise enough results, ...

\goodbadexample{
    ...
}{
    ...
}

\noindent ...


\guideline[g:abstract:summarize_approach]
    {Summarize the key steps of your approach.}

\goodbadexample{
    ...
}{
    ...
}

\noindent ...
% - summarize the main idea of your work in 2 sentences: sequential explanation/walkthrough


\guideline{...}
% - short 1-sentence summary of results (evaluated on what? what did the evaluation show?)

\goodbadexample{
    ...
}{
    ...
}

\noindent ...


\guideline[g:abstract:plain_text]
    {Use only plain text.}

\goodbadexample{
    ...
}{
    ...
}

\noindent ...
% no italics, boldface, underline, etc.
% no abbreviations
% only use mathematical notation unless it cannot be phrased in words and is the core contribution


\guideline[g:abstract:keywords]
    {Use the keywords/index terms to list terms that are not in the title or abstract.}

\goodexample[{\cite[Title, Abstract, Index Terms]{Wetzlinger2024CSL}}]{
    \textit{Inner Approximations of Reachable Sets for Nonlinear Systems Using the Minkowski Difference} \\
    Abstract---Reachability analysis is a formal method that rigorously proves whether a dynamical system can reach certain states.
    Inner approximations of the exact reachable set contain only states that are definitely reachable and are therefore used to falsify specifications.
    Whilethe majority of state-of-the-art approaches for nonlinear systems obtain an inner approximation via first computing an outer approximation of the reachable set, we directly obtain sound inner approximations by using the Minkowski difference in a reachability algorithm for nonlinear systems.
    Our implementation uses a combination of polytopes and constrained zonotopes as set representations, resulting in a low polynomial time complexity in the state dimension.
    A comparison with state-of-the-art approaches on several benchmarks demonstrates the advantages of our approach. \\
    Index Terms---\highlightpart{Formal verification}, \highlightpart{falsification}, reachability analysis, nonlinear systems, \highlightpart{set-based computing}.
}

\noindent
Keywords are used by databases, search engines, and indexing services to match articles with relevant search queries.
To increase the chances of your paper being found, include a broad and relevant set of terms that reflect your topic.

