
\chapter{Title, Abstract, and Keywords}
\label{ch:abstract}

Every paper starts with the same three text elements---title, abstract, and keywords.
The title is the prime reference point of each paper, capturing its essence in few yet expressive words.
We will use a running example for the guidelines in this chapter \cite[Abstract]{Wetzlinger2024CSL}.


\guideline[g:abstract:title]
    {Make each word in the title matter.}

\goodbadexample[{\cite[Title]{Wetzlinger2024CSL}}]{
    ``Computing Inner Approximations of Reachable Sets for Nonlinear Dynamical Systems''
    \badexpl{The word ``computing'' does not add much value as most papers are concerned with computation, the term ``nonlinear dynamical systems'' is wordy yet inaccurate;
    crucially, the paper's unique selling point is missing, such as the core idea of the algorithm.}
}{
    ``Inner Approximations of Reachable Sets for Nonlinear Systems Using the Minkowski Difference''
}

\noindent The title summarizes the topic of the paper, and potentially even the applied methodology, into its densest form.
We want the title to be as short as possible while being as informative as possible.
Avoid words that do not add much value and use short versions where the loss of information is negligible.
Still, include bits of methodology that set your paper apart from existing work.


\smallskip

The abstract provides a concise summary of the paper:
In order, it motivates and states the research question, highlights the limitations of existing methods, presents the main approach and its advantages, discusses the results of the evaluation, and provides a brief conclusion.
A well-written abstract helps readers quickly determine the paper's relevance to their interests.


\guideline{...}
% - start out with a sentence that sets a frame for the topic of the paper, why it is important, and can be understood by a broad audience

\goodbadexample{
    ...
}{
    ...
}

\noindent ...


\guideline[g:abstract:research_question]
    {State the research question addressed in your paper.}

\goodexample[{\cite[Abstract]{Wetzlinger2024CSL}}]{
    Inner approximations of the exact reachable set contain \highlightpart{only states that are definitely reachable} and are therefore \highlightpart{used to falsify specifications}.
}

\noindent
The next sentence(s) narrow down the general frame opened by the first sentence by concretizing the overarching research field to the specific topic of the paper.


\guideline[g:abstract:outline_limitations]
    {Outline the limitations of the state of the art.}

\goodexample[{\cite[Abstract]{Wetzlinger2024CSL}}]{
    While the majority of state-of-the-art approaches for nonlinear systems obtain an \highlightpart{inner approximation via first computing an outer approximation} of the reachable set, (...)
}

\noindent
Following the concretization of the topic, introduce the key motivation for your proposed contribution via a succinct overview of the state of the art.
For example, the specific research question has not yet been addressed, current approaches do not provide good results or do not scale well.
Crucially, the mentioned limitation(s) should directly lead to the next sentence, which summarizes the proposed contribution.


\guideline[g:abstract:summarize_approach]
    {Summarize the key steps and properties of your approach.}

\goodexample[{\cite[Abstract]{Wetzlinger2024CSL}}]{
    (...), we \highlightpart{directly obtain sound inner approximations by using the Minkowski difference} in a reachability algorithm for nonlinear systems.
    Our implementation uses a \highlightpart{combination of polytopes and constrained zonotopes} as set representations, resulting in a \highlightpart{low polynomial time complexity in the state dimension}.
}

\noindent
Provide the shortest summary of your proposed contribution.
This comprises an overview of the high-level idea behind your approach, or even a sequential explanation of the main steps.
You may think of this part of the abstract as the bit of information that you most want your readers to remember.
This part is commonly referred to by other researchers' introductions about the research topic.


\guideline{...}
% - short 1-sentence summary of results (evaluated on what? what did the evaluation show?)... how does it matter?

\goodbadexample{
    ...
}{
    ...
}

\noindent ...


\guideline{Use only plain text.}

\goodbadexample{
    ...
}{
    ...
}

\noindent ...
% no italics, boldface, underline, etc.
% no abbreviations
% only use mathematical notation unless it cannot be phrased in words and is the core contribution


\guideline[g:abstract:keywords]
    {Use the keywords to list terms unused in the title or abstract.}

\goodexample[{\cite[Title, Abstract, Index Terms]{Wetzlinger2024CSL}}]{
    \textit{Inner Approximations of Reachable Sets for Nonlinear Systems Using the Minkowski Difference} \\
    Abstract---Reachability analysis is a formal method that rigorously proves whether a dynamical system can reach certain states.
    Inner approximations of the exact reachable set contain only states that are definitely reachable and are therefore used to falsify specifications.
    Whilethe majority of state-of-the-art approaches for nonlinear systems obtain an inner approximation via first computing an outer approximation of the reachable set, we directly obtain sound inner approximations by using the Minkowski difference in a reachability algorithm for nonlinear systems.
    Our implementation uses a combination of polytopes and constrained zonotopes as set representations, resulting in a low polynomial time complexity in the state dimension.
    A comparison with state-of-the-art approaches on several benchmarks demonstrates the advantages of our approach. \\
    Index Terms---\highlightpart{Formal verification}, \highlightpart{falsification}, reachability analysis, nonlinear systems, \highlightpart{set-based computing}.
    \goodexpl{The three highlighted terms are semantically related to the topic, but not used verbatim in the title or abstract.}
}

\noindent
Keywords, also called index terms, are used by databases, search engines, and indexing services to match articles with relevant search queries.
To increase the chances of your paper being found, include a broad and relevant set of terms that reflect your topic.

