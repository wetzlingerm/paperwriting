
\guideline[g:abstract:summarize_approach]
    {Summarize the key steps and properties of your approach.}

\goodexample[{\cite[Abstract]{Wetzlinger2024CSL}}]{
    (...), we \highlightpart{directly obtain sound inner approximations by using the Minkowski difference} in a reachability algorithm for nonlinear systems.
    Our implementation uses a \highlightpart{combination of polytopes and constrained zonotopes} as set representations, resulting in a \highlightpart{low polynomial time complexity in the state dimension}.
}

\noindent 
Provide the shortest summary of your proposed contribution.
This comprises an overview of the high-level idea behind your approach, or even a sequential explanation of the main steps.
You may think of this part of the abstract as the bit of information that you most want your readers to remember.
Other papers' related work sections commonly use this part when citing your work.
This part is commonly referred to by other researchers' introductions about the research topic.
