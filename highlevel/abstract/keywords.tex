
\guideline[g:abstract:keywords]
    {Use the keywords to list terms unused in the title or abstract.}

\goodexample[{\cite[Title, Abstract, Index Terms]{Wetzlinger2024CSL}}]{
    \textit{Inner Approximations of Reachable Sets for Nonlinear Systems Using the Minkowski Difference} \\
    Abstract---Reachability analysis is a formal method that rigorously proves whether a dynamical system can reach certain states.
    Inner approximations of the exact reachable set contain only states that are definitely reachable and are therefore used to falsify specifications.
    Whilethe majority of state-of-the-art approaches for nonlinear systems obtain an inner approximation via first computing an outer approximation of the reachable set, we directly obtain sound inner approximations by using the Minkowski difference in a reachability algorithm for nonlinear systems.
    Our implementation uses a combination of polytopes and constrained zonotopes as set representations, resulting in a low polynomial time complexity in the state dimension.
    A comparison with state-of-the-art approaches on several benchmarks demonstrates the advantages of our approach. \\
    Index Terms---\highlightpart{Formal verification}, \highlightpart{falsification}, reachability analysis, nonlinear systems, \highlightpart{set-based computing}.
    \goodexpl{The three highlighted terms are semantically related to the topic, but not used verbatim in the title or abstract.}
}

\noindent
Keywords, also called index terms, are used by databases, search engines, and indexing services to match articles with relevant search queries.
To increase the chances of your paper being found, include a broad and relevant set of terms that reflect your topic.
