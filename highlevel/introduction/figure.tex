
\guideline[g:introduction:graphical_abstract]
    {Consider creating a graphical abstract.}

\goodexample[{\cite[Fig.~1]{Wetzlinger2024CSL}}]{
    \input{data/headfigure.tikz}

    \begin{small}
    Fig. 1: Main idea: The Minkowski difference between the reachable sets for the affine dynamics $\mathcal{R}_{\text{aff}}$ and due to the error dynamics $\mathcal{R}_{\text{err}}$ returns a sound inner approximation $\widecheck{\mathcal{R}}$.
    \end{small}
}

\noindent 
As the saying goes, a picture is worth a thousand words.
A graphical abstract treads a fine line between information and simplicity---it captures both the problem and the methodology of your approach, but leaves out technical details.


