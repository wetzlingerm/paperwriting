
\guideline[g:introduction:graphical_abstract]
    {Consider creating a graphical abstract.}

\goodexample[{\cite[Fig.~1]{Wetzlinger2024CSL}}]{
    \begin{tikzpicture}[
	every node/.style={font=\small, align=left},
	myarrow/.style={<-, shorten <=0.05cm, >=stealth', semithick}
	]
	% bounding box, grid
	%\draw[red] (0,0) rectangle (8.8,4);
	%\draw[red] (0,0.3) grid (8.8,3.6);
	
	% start set
	\draw (0.15,0.3) rectangle (1.15,1.35);
	\draw[myarrow] (0.65,1.35) -- ++(0,1.2) node[above] {Initial \\ set $\mathcal{X}_0$};
	
	% indication of first step
	\draw[myarrow,dotted] (2.8,0.9) .. controls (2.0,0.65) and (1.6,0.65) .. (1.25,0.85);
	\node[anchor=west] at (1.3,0.45) {Time step};
	
	% non-convex exact reachable set (solid)
	%	\draw[rounded corners=8pt,xshift=7cm,yshift=2cm,rotate=0]   (-1.2,-0.9) -- (0,-0.4) -- (1.2,-0.7) -- (1.0,0.9) -- (-1.0,0.6) -- cycle;
	\draw[rounded corners=8pt,xshift=3cm,yshift=2cm,rotate=-30] (-1.2,-1.0) -- (0,-0.45) -- (1.3,-0.8) -- (1.0,0.9) -- (-1.0,0.6) -- cycle;
	\draw[myarrow] (4.2,2.225) -- ++(0.6,0.3) node[right]
		{Exact reachable set $\mathcal{R}$};
	
	% reachable set of the linearized dynamics (dashed)
	\draw[dashed,xshift=3cm,yshift=2cm,rotate=-30] (-1.2,-0.7) rectangle (1.2,0.7);
	\draw[myarrow] (4.35,1.9) -- ++(0.45,-0.225) node[right] 
		{Reachable set for the \\ affine dynamics $\mathcal{R}_{\text{aff}}$};
	
	% outer approximation (more complex representation)
	\draw[xshift=3cm,yshift=2cm,rotate=-30] (-1.5,-1) rectangle (1.5,1);
	\draw[myarrow] (3.8,2.7) -- ++(1.0,0.5) node[right]
		{Outer approximation of the \\ reachable set $\widehat{\mathcal{R}} = \mathcal{R}_{\text{aff}} \oplus \mathcal{R}_{\text{err}}$};
	
	% inner approximation (also only parallelotope)
	\draw[xshift=3cm,yshift=2cm,rotate=-30] (-0.7,-0.4) rectangle (0.7,0.4);
	\draw[myarrow] (3.5,1.4) -- ++(1.3,-0.65) node[right] 
		{Inner approximation of the \\ reachable set $\widecheck{\mathcal{R}} = \mathcal{R}_{\text{aff}} \oplus \mathcal{R}_{\text{err}}$};
	
\end{tikzpicture}

    \begin{small}
    Fig. 1: Main idea: The Minkowski difference between the reachable sets for the affine dynamics $\mathcal{R}_{\text{aff}}$ and due to the error dynamics $\mathcal{R}_{\text{err}}$ returns a sound inner approximation $\widecheck{\mathcal{R}}$.
    \end{small}
}

\noindent 
As the saying goes, a picture is worth a thousand words.
A graphical abstract treads a fine line between information and simplicity---it captures both the problem and the methodology of your approach, but leaves out technical details.


