
\guideline[g:introduction:outline_structure]
    {Outline the structure of the paper.}

\goodexample[{\cite[Adapted from Sec.~I.]{Wetzlinger2025TAC}}]{
    In this article, we compute minimal and maximal backward reachable sets for continuous-time linear time-invariant (LTI) systems.
    As there are many similar definitions of backward reachable sets as well as related concepts, we postpone the literature review to \highlightpart{Section IV}.
    This allows us to use the preliminary information from \highlightpart{Sections II and III} for a more concise overview.
    Our contributions are as follows.
    \begin{compactitemize}
        \item An inner and outer approximation of the time-point minimal backward reachable set (\highlightpart{see Section V-A}).
        \item An outer approximation of the time-interval minimal backward reachable set (\highlightpart{see Section V-B}).
        \item An inner and outer approximation of the time-point maximal backward reachable set (\highlightpart{see Section VI-A}).
        \item An inner approximation of the time-interval maximal backward reachable set (\highlightpart{see Section VI-B}).
    \end{compactitemize}
    \goodexpl{Seamless integration of the structural outline into the contributions.}
    Crucially, all proposed algorithms scale only polynomially with respect to the state dimension.
    In addition, we discuss the approximation errors of each computed set.
    Our evaluation \highlightpart{in Section VII} is followed by closing remarks \highlightpart{in Section VIII}.
}

\goodexample[{\cite[Sec.~1]{Wetzlinger2024ARCH2}}]{
    \highlightpart{In Section 2}, we define the halfspace representation and the vertex representation.
    Then, we introduce the object class implemented in CORA \highlightpart{in Section 3}, which also contains several set properties to improve computational efficiency.
    Afterward \highlightpart{in Section 4}, we define all implemented operations, including conversions between representations, predicates, as well as unary and binary set operations.
    Our numerical evaluation \highlightpart{in Section 5} compares our implementation with the multi-parametric toolbox [10].
    \goodexpl{Separate paragraph for the outline.}
}

\noindent
There are two main ways of outlining the structure of the paper.
First, by integrating the structural outline into the contributions.
This is recommendable for shorter works or a more compact list of contributions.
Second, by using a separate paragraph for the outline of the paper.
This is often the last paragraph of the introduction, or even a separate subsection if the introduction has subsections.
For papers of short length (6 pages or less), it can sometimes be advisable to skip the structure outline and use that space in other parts of the paper.
