
\guideline[g:introduction:outline_structure]
    {Outline the structure of the paper.}

\goodexample[{\cite[Sec.~1.2]{Wetzlinger2023HSCC}}]{
    We first introduce the general notation as well as set representations and operations \highlightpart{in Sec. 2} and formally define the problem statement \highlightpart{in Sec. 3}.
    Afterward, we provide a comprehensive summary of the reachability algorithm in [4] for computing outer-approximations \highlightpart{in Sec. 4.1}.
    Our contributions are as follows:
    \begin{itemize}
        \item First, we present a novel reachability algorithm using support functions to compute inner-approximations (\highlightpart{Sec. 4.2}).
        \item Next, we design a fully-automated verification algorithm for the special case of unsafe sets given as halfspaces (\highlightpart{Sec. 5.1}).
        \item Moreover, we propose a fully-automated verification algorithm for arbitrary convex unsafe sets (\highlightpart{Sec. 5.2}).
        \item In case of a safety violation, our verification algorithms return a counterexample, which provides valuable insights to system engineers (\highlightpart{Sec. 5.1-5.2}).
    \end{itemize}
    Overall, our paper provides a complete description of support function reachability, combining outer- and inner-approximation with automated verification in a self-contained presentation. 
    In contrast to previous work on reachability analysis using support functions, we provide the first approach that automatically verifies a given problem in decidable cases.
    Finally, the practical benefits of our novel algorithms are demonstrated on several challenging benchmark problems \highlightpart{in Sec. 6}.
}

\goodexample[{\cite{Wetzlinger2024ARCH2}}]{
    \highlightpart{In Section 2}, we define the halfspace representation and the vertex representation.
    Then, we introduce the object class implemented in CORA \highlightpart{in Section 3}, which also contains several set properties to improve computational efficiency.
    Afterward \highlightpart{in Section 4}, we define all implemented operations, including conversions between representations, predicates, as well as unary and binary set operations.
    Our numerical evaluation \highlightpart{in Section 5} compares our implementation with the multi-parametric toolbox [10].
}

\noindent 
As shown in the first example, one can seamlessly integrate the outline of the structure into the contributions. This is recommendable for shorter works or a more compact list of contributions.

The second example shows the more classical approach of using a separate paragraph for the outline of the paper.
This is often the last paragraph of the introduction, or even a separate subsection if the introduction has subsections.

For papers of short length (6 pages or less), it can sometimes be advisable to skip the structure outline and use that space in other parts of the paper.