
\guideline[g:introduction:relatedwork_structure]
    {Organize the related work into paragraphs to separate subtopics.}

\goodexample[{\cite[Sec.~1.1]{Wetzlinger2023HSCC}}]{
    Since the reachable set is \highlightpart{a zero sublevel set solution of a Hamilton-Jacobi-Isaacs partial differential equation} [41], (...)
    \goodexpl{Paragraph on approaches based on Hamilton-Jacobi reachability.}
    \highlightpart{Simulations} from a sample of initial states within the initial set can be used to construct reachable sets [19]; (...)
    \goodexpl{Paragraph on approaches based on simulation.}
    Another group of methods is \highlightpart{based on set propagation} [5].
    These methods either (...)
    \goodexpl{Paragraph on approaches based on set propagation.}
    In contrast to the above algorithms for outer-approximations, \highlightpart{approaches for inner approximations} (...)
    \goodexpl{Paragraph on approaches for inner approximations.}
    For successful \highlightpart{verification}, the computed outer-approximation of the reachable set (...)
    \goodexpl{Paragraph on approaches for verification.}
}

\noindent
Related work often divides naturally into subtopics that offer distinct perspectives on the research question.
Help readers clearly distinguish these perspectives by dedicating one paragraph to each subtopic, and let the opening phrase of the first sentence act as an implicit header---as highlighted in the above example.
If the discussion of individual subtopics becomes extensive, use subsections instead of paragraphs to maintain clarity and structure.
