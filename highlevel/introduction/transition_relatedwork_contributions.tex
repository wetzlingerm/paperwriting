
\guideline[g:introduction:transition_relatedwork_contributions]
    {Relate the limitations from the related work to the contributions.}

\goodexample[{\cite[Sec.~I.A.-B.]{Wetzlinger2023TAC}}]{
    \location{Related work} \\
    These tools \highlightpart{still require manual tuning} of the algorithm parameters to obtain tight approximations. (...) \\
    In conclusion, there \highlightpart{does not yet exist a fully automated parameter tuning algorithm} for linear systems that \highlightpart{satisfies an error bound} in terms of the Hausdorff distance to the exact reachable set. \par
    \location{Contributions} \\
    (...) we provide an \highlightpart{automated reachability algorithm} (Algorithm 2) that adaptively \highlightpart{tunes all the algorithm parameters} so that \highlightpart{any desired error bound} in terms of the Hausdorff distance between the exact reachable set and the computed outer approximation is respected at all times (see Section IV).
    \goodexpl{Clear link between limitations and contributions: manual vs.\ automated, no automated algorithm vs.\ automated algorithm.}
}

\noindent
Readers must clearly understand what problem the paper addresses, so the contributions should be explicitly tied to the limitations of prior work.
Each contribution should respond directly to a specific shortcoming in the state of the art, using consistent wording or even restating the limitation when appropriate.
Avoid leaving these connections implicit.
