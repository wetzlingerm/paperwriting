
\guideline[g:introduction:contribution_list]
    {Consider presenting your contributions in a list.}

\goodexample[{\cite[Adapted from Sec.~I.]{Wetzlinger2025TAC}}]{
    Our contributions are as follows.
    \begin{compactitemize}[label=\colorbox{color_highlight}{\textbullet}]
        \item An inner and outer approximation of the time-point minimal backward reachable set (see Section V-A).
        \item An outer approximation of the time-interval minimal backward reachable set (see Section V-B).
        \item An inner and outer approximation of the time-point maximal backward reachable set (see Section VI-A).
        \item An inner approximation of the time-interval maximal backward reachable set (see Section VI-B).
    \end{compactitemize}
}

\noindent
A list is well-suited for presenting multiple contributions for several reasons:
First, a list stands out visually from the running text, thus framing the content in a highlight.
Second, a list emphasizes the individuality of each contribution---a paragraph more commonly
Third, a list demands concise wording for each list entry, which in turn helps to communicate your work to the readers.
You can use either an ordered or an unordered list, although the latter is likely more common.
For shorter works where the content revolves around presenting a single idea, a paragraph explaining that contribution on a high level may be better suited.
