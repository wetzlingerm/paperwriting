
\guideline[g:introduction:summarize_approaches]
    {Briefly summarize related approaches, addressing their applicability and highlighting potential limitations.}

\goodbadexample[{\cite[Sec.~I.]{Wetzlinger2024CSL}}]{
    Other prominent approaches for computing inner approximations include Hamilton-Jacobi reachability [3] \highlightempty{} and dissipativity-based approaches \highlightempty{} [12], [13].
}{
    One can also obtain inner approximations via \highlightpart{optimization-based techniques}.
    Prominent approaches include Hamilton-Jacobi reachability [3], which \highlightpart{scales exponentially in the state dimension} \highlightpart{due to gridding}, and dissipativity-based approaches \highlightpart{using sum-of-squares programming} [12], [13], which \highlightpart{scales polynomially in the state dimension, but exponentially in the degree of the polynomial} representing the reachable set [14].
}

\noindent
When providing readers with an overview of the state of the art, it is not enough to merely mention the approaches by name.
Instead, briefly summarize key properties such as the main idea of each approach, its scalability across different dimensions, and its potential applicability in various domains.
To optimally resolve the trade-off between depth and brevity, phrase these properties concisely and omit further details.

