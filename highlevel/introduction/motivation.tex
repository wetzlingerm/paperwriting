
\guideline[g:introduction:motivation]
    {Start by moving from the broad research field to the precise topic.}

\goodbadexample[{\cite[Sec.~I.]{Wetzlinger2024CSL}}]{
    \location{Beginning of introduction} \\
    Inner approximations of reachable sets for uncertain dynamical systems are used as a formal method to falsify safety specifications.
    \badexpl{The terms ``inner approximations'', ``uncertain dynamical systems'', and ``formal method'' are not properly introduced.}
}{
    \location{Beginning of introduction} \\
    Formal methods can determine whether an uncertain dynamical system meets a given specification.
    Reachability analysis computes all states that are reachable under the given uncertainties, and, thus, can be used for formal verification [1].
    However, the exact reachable set cannot be computed except for special system classes [2];
    if a computed outer approximation cannot verify safety, an inner approximation may falsify safety, instead.
    \goodexpl{The context is set up step by step: First, we introduce formal methods and uncertain dynamical systems, then present reachability analysis as a method and discuss the computation of reachable sets; finally, we explain how inner approximations fit into the context of formal verification.}
}

\noindent
Commonly, the first sentence introduces the purpose and relevance of the research field.
After that, guide the reader from that broad context to the specific topic of the paper in a strictly sequential manner, with each key term in the opening sentences marking a clear step from coarse to fine.
Avoid circling back or shifting focus---move linearly toward the precise research topic.
