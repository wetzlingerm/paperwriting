
\chapter{Introduction}
\label{ch:introduction}

Commonly, the introduction section consists of three parts:
First, it presents the research field, motivatives its importance, and narrows down to the specific research question addressed by this paper.
Next, an overview of related work contextualizes the research question by discussing related questions and approaches, while identifying the gap in current knowledge or limitations of existing methods.
Finally, it concisely outlines the paper's contributions in addressing that gap.


\guideline[g:introduction:structural_separation]
    {Structurally separate the motivation, related work, and contributions.}

% Will be a fairly long example -> use shortest introduction section
% Rewrite in a way that mixes the three parts
\goodbadexample{
    ...
}{
    ...
}

\noindent ...
% - type of separation depends on length of paper



\guideline[g:introduction:motivation]
    {Explain the purpose of the research field, then narrow the focus to your specific topic.}

\goodbadexample{
    ...
}{
    ...
}

\noindent ...
% - the first paragraph must explain the purpose of the research field, and go from broad to specific until the topic of this paper is reached



\guideline[g:introduction:relatedwork_scope]
    {For broad topics, explicitly restrict the scope of the related work overview.}

\goodexample[{\cite[Sec.~I.A.]{Wetzlinger2023TAC}}]{
    While there exist many different approaches, e.g., stochastic techniques [8] or data-driven/learning methods [9], [10], the following review focuses on the model-based reachability analysis of linear systems.
}

\noindent
The purpose of the related work overview is to embed your work in the scientific landscape.
Opinions will differ on where the boundary is between what is relevant and what is irrelevant.
Therefore, clearly communicate where you set that boundary.



\guideline[g:introduction:relatedwork_structure]
    {Organize the related work into paragraphs, each focused on a specific research field.}

\goodbadexample{
    ...
}{
    ...
}

\noindent ...


\guideline{...}
% - summarize related approaches in 1-2 sentences, explaining their applicability and limitations

\goodbadexample{
    ...
}{
    ...
}

\noindent ...



\guideline[g:introduction:transition_relatedwork_contributions]
    {Relate the limitations from the related work to the contributions.}

\goodbadexample{
    ...
}{
    ...
}

\noindent ...



\guideline[g:introduction:contribution_bullet_points]
    {Consider presenting your contributions using bullet points.}

\goodbadexample{
    ...
}{
    ...
}

\noindent ...


\guideline[g:introduction:noncontributions]
    {Avoid mentioning methods and code releases as contributions.}

% bad: separate bullet points for experiments -> better: mention in roadmap
% bad: selling methods that already exist as contribution because you use them for a new problem
\goodbadexample{
    ...
}{
    ...
}

\noindent ...
% what is a method: "we use X to do Y"... there must be more to it than just "i did it differently"
% put "we provide code" immediately after the contributions


\guideline[g:introduction:outline_structure]
    {Outline the structure of the paper.}

\goodexample[{\cite[Sec.~1.2]{Wetzlinger2023HSCC}}]{
    We first introduce the general notation as well as set representations and operations \highlightpart{in Sec. 2} and formally define the problem statement \highlightpart{in Sec. 3}.
    Afterward, we provide a comprehensive summary of the reachability algorithm in [4] for computing outer-approximations \highlightpart{in Sec. 4.1}.
    Our contributions are as follows:
    \begin{itemize}
        \item First, we present a novel reachability algorithm using support functions to compute inner-approximations (\highlightpart{Sec. 4.2}).
        \item Next, we design a fully-automated verification algorithm for the special case of unsafe sets given as halfspaces (\highlightpart{Sec. 5.1}).
        \item Moreover, we propose a fully-automated verification algorithm for arbitrary convex unsafe sets (\highlightpart{Sec. 5.2}).
        \item In case of a safety violation, our verification algorithms return a counterexample, which provides valuable insights to system engineers (\highlightpart{Sec. 5.1-5.2}).
    \end{itemize}
    Overall, our paper provides a complete description of support function reachability, combining outer- and inner-approximation with automated verification in a self-contained presentation. 
    In contrast to previous work on reachability analysis using support functions, we provide the first approach that automatically verifies a given problem in decidable cases.
    Finally, the practical benefits of our novel algorithms are demonstrated on several challenging benchmark problems \highlightpart{in Sec. 6}.
}

\goodexample[{\cite[Sec.~1]{Wetzlinger2024ARCH2}}]{
    \highlightpart{In Section 2}, we define the halfspace representation and the vertex representation.
    Then, we introduce the object class implemented in CORA \highlightpart{in Section 3}, which also contains several set properties to improve computational efficiency.
    Afterward \highlightpart{in Section 4}, we define all implemented operations, including conversions between representations, predicates, as well as unary and binary set operations.
    Our numerical evaluation \highlightpart{in Section 5} compares our implementation with the multi-parametric toolbox [10].
}

\noindent 
As shown in the first example, one can seamlessly integrate the outline of the structure into the contributions. This is recommendable for shorter works or a more compact list of contributions.

The second example shows the more classical approach of using a separate paragraph for the outline of the paper.
This is often the last paragraph of the introduction, or even a separate subsection if the introduction has subsections.

For papers of short length (6 pages or less), it can sometimes be advisable to skip the structure outline and use that space in other parts of the paper.

\guideline{Avoid empty sentences when outlining the structure.}

\goodbadexample{
    ...
}{
    ...
}

\noindent ...
% only relevant if the structure is a separate paragraph/section
% however, no empty sentence ("section 4 provides numerical experiments"), rather a summary of the content of each section

