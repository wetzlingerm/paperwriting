
\chapter{Introduction}
\label{ch:introduction}

Commonly, the introduction consists of three parts:
First, we present the research field and narrow the topic to the specific research question addressed by the paper.
Next, we contextualize that research question by discussing related work, while identifying a gap in current knowledge or limitations of existing methods.
Finally, we concisely outline the paper's contributions in addressing that gap.


\guideline{Structurally separate the motivation, related work, and contributions.}

\goodbadexample{
    ...
}{
    ...
}

\noindent ...
% - type of separation depends on length of paper


\newpage


\guideline[g:introduction:motivation]
    {Explain the purpose of the research field, then narrow the focus to your specific topic.}

% use first paragraph of one of the papers (inner approx)
% bad: no explain at all? or jump directly to specific topic without providing any context
\goodbadexample{
    ...
}{
    ...
}

\noindent ...
% - the first paragraph must explain the purpose of the research field, and go from broad to specific until the topic of this paper is reached


\guideline{...}
% - for broad topics, explicitly restrict the scope of the related work overview

\goodbadexample{
    ...
}{
    ...
}

\noindent ...


\guideline[g:introduction:relatedwork_structure]
    {Organize the related work into paragraphs to separate subtopics.}

\goodexample[{\cite[Sec.~1.1]{Wetzlinger2023HSCC}}]{
    \tcolorboxindent{} Since the reachable set is \highlightpart{a zero sublevel set solution of a Hamilton-Jacobi-Isaacs partial differential equation} [41], (...) \\
    \tcolorboxindent{} \highlightpart{Simulations} from a sample of initial states within the initial set can be used to construct reachable sets [19]; (...) \\
    \tcolorboxindent{} Another group of methods is \highlightpart{based on set propagation} [5].
    These methods either (...) \\
    \tcolorboxindent{} In contrast to the above algorithms for outer-approximations, \highlightpart{approaches for inner-approximations} (...) \\
    \tcolorboxindent{} For successful \highlightpart{verification}, the computed outer-approximation of the reachable set (...)
}

\noindent
Related work often divides naturally into subtopics that offer distinct perspectives on the research question.
Help readers clearly distinguish these perspectives by dedicating one paragraph to each subtopic, and let the opening phrase of the first sentence act as an implicit header---as highlighted in the above example.
If the discussion of individual subtopics becomes extensive, use subsections instead of paragraphs to maintain clarity and structure.


\guideline{...}
% - summarize related approaches in 1-2 sentences, explaining their applicability and limitations

\goodbadexample{
    ...
}{
    ...
}

\noindent ...


\guideline[g:introduction:transition_relatedwork_contributions]
    {Relate the limitations from the related work to the contributions.}

\goodexample[{\cite[Sec.~I.A.-B.]{Wetzlinger2023TAC}}]{
    \location{Related work} \\
    These tools \highlightpart{still require manual tuning} of the algorithm parameters to obtain tight approximations. (...) \\
    In conclusion, there \highlightpart{does not yet exist a fully automated parameter tuning algorithm} for linear systems that \highlightpart{satisfies an error bound} in terms of the Hausdorff distance to the exact reachable set. \par
    \location{Contributions} \\
    (...) we provide an \highlightpart{automated reachability algorithm} (Algorithm 2) that adaptively \highlightpart{tunes all the algorithm parameters} so that \highlightpart{any desired error bound} in terms of the Hausdorff distance between the exact reachable set and the computed outer approximation is respected at all times (see Section IV).
    \goodexpl{Clear link between limitations and contributions: manual vs.\ automated, no automated algorithm vs.\ automated algorithm.}
}

\noindent
Readers must clearly understand what problem the paper addresses, so the contributions should be explicitly tied to the limitations of prior work.
Each contribution should respond directly to a specific shortcoming in the state of the art, using consistent wording or even restating the limitation when appropriate.
Avoid leaving these connections implicit.


\newpage


\guideline[g:introduction:graphical_abstract]
    {Consider creating a graphical abstract.}

\goodexample[{\cite[Fig.~1]{Wetzlinger2024CSL}}]{
    \input{data/headfigure.tikz}

    \begin{small}
    Fig. 1: Main idea: The Minkowski difference between the reachable sets for the affine dynamics $\mathcal{R}_{\text{aff}}$ and due to the error dynamics $\mathcal{R}_{\text{err}}$ returns a sound inner approximation $\widecheck{\mathcal{R}}$.
    \end{small}
}

\noindent 
As the saying goes, a picture is worth a thousand words.
A graphical abstract treads a fine line between information and simplicity---it captures both the problem and the methodology of your approach, but leaves out technical details.




\guideline[g:introduction:contribution_list]
    {Consider presenting your contributions in a list.}

\goodexample[{\cite[Adapted from Sec.~I.]{Wetzlinger2025TAC}}]{
    Our contributions are as follows.
    \begin{compactitemize}[label=\colorbox{color_highlight}{\textbullet}]
        \item An inner and outer approximation of the time-point minimal backward reachable set (see Section V-A).
        \item An outer approximation of the time-interval minimal backward reachable set (see Section V-B).
        \item An inner and outer approximation of the time-point maximal backward reachable set (see Section VI-A).
        \item An inner approximation of the time-interval maximal backward reachable set (see Section VI-B).
    \end{compactitemize}
}

\noindent
A list is well-suited for presenting multiple contributions for several reasons:
First, a list stands out visually from the running text, thus framing the content in a highlight.
Second, a list emphasizes the individuality of each contribution---a paragraph more commonly
Third, a list demands concise wording for each list entry, which in turn helps to communicate your work to the readers.
You can use either an ordered or an unordered list, although the latter is likely more common.
For shorter works where the content revolves around presenting a single idea, a paragraph explaining that contribution on a high level may be better suited.


\guideline{Avoid mentioning methods and code releases as contributions.}

\goodbadexample{
    ...
}{
    ...
}

\noindent ...
% what is a method: "we use X to do Y"... there must be more to it than just "i did it differently"
% put "we provide code" immediately after the contributions


\guideline[g:introduction:outline_structure]
    {Outline the structure of the paper.}

\goodexample[{\cite[Sec.~1.2]{Wetzlinger2023HSCC}}]{
    We first introduce the general notation as well as set representations and operations \highlightpart{in Sec. 2} and formally define the problem statement \highlightpart{in Sec. 3}.
    Afterward, we provide a comprehensive summary of the reachability algorithm in [4] for computing outer-approximations \highlightpart{in Sec. 4.1}.
    Our contributions are as follows:
    \begin{itemize}
        \item First, we present a novel reachability algorithm using support functions to compute inner-approximations (\highlightpart{Sec. 4.2}).
        \item Next, we design a fully-automated verification algorithm for the special case of unsafe sets given as halfspaces (\highlightpart{Sec. 5.1}).
        \item Moreover, we propose a fully-automated verification algorithm for arbitrary convex unsafe sets (\highlightpart{Sec. 5.2}).
        \item In case of a safety violation, our verification algorithms return a counterexample, which provides valuable insights to system engineers (\highlightpart{Sec. 5.1-5.2}).
    \end{itemize}
    Overall, our paper provides a complete description of support function reachability, combining outer- and inner-approximation with automated verification in a self-contained presentation. 
    In contrast to previous work on reachability analysis using support functions, we provide the first approach that automatically verifies a given problem in decidable cases.
    Finally, the practical benefits of our novel algorithms are demonstrated on several challenging benchmark problems \highlightpart{in Sec. 6}.
}

\goodexample[{\cite{Wetzlinger2024ARCH2}}]{
    \highlightpart{In Section 2}, we define the halfspace representation and the vertex representation.
    Then, we introduce the object class implemented in CORA \highlightpart{in Section 3}, which also contains several set properties to improve computational efficiency.
    Afterward \highlightpart{in Section 4}, we define all implemented operations, including conversions between representations, predicates, as well as unary and binary set operations.
    Our numerical evaluation \highlightpart{in Section 5} compares our implementation with the multi-parametric toolbox [10].
}

\noindent 
As shown in the first example, one can seamlessly integrate the outline of the structure into the contributions. This is recommendable for shorter works or a more compact list of contributions.

The second example shows the more classical approach of using a separate paragraph for the outline of the paper.
This is often the last paragraph of the introduction, or even a separate subsection if the introduction has subsections.

For papers of short length (6 pages or less), it can sometimes be advisable to skip the structure outline and use that space in other parts of the paper.

\guideline[g:introduction:roadmap_empty_sentences]
    {Avoid empty sentences when outlining the structure.}

\goodbadexample{
    ...
}{
    ...
}

\noindent ...
% only relevant if the structure is a separate paragraph/section
% however, no empty sentence ("section 4 provides numerical experiments"), rather a summary of the content of each section

