
\guideline[g:results:data]
    {Provide all data required for running the numerical experiments.}

\goodexample[{\cite[Sec.~VII.A.]{Wetzlinger2025TAC}}]{
    First, we compare (...) on a 4-D pursuit-evasion game defined by the double integrator dynamics [32, eq. (24)]
    \begin{equation*}
        \highlightpartmath{A =} \begin{pmatrix} 0 & 1 & 0 & 0 \\ 0 & 0 & 0 & 0 \\ 0 & 0 & 0 & 1 \\ 0 & 0 & 0 & 0 \end{pmatrix},
        \highlightpartmath{B =} \begin{pmatrix} 0 & 0 \\ 1 & 0 \\ 0 & 0 \\ 0 & 1 \end{pmatrix},
        \highlightpartmath{E =} \begin{pmatrix} 0 & 0 \\ -1 & 0 \\ 0 & 0 \\ 0 & -1 \end{pmatrix}.
    \end{equation*}
    (...) We choose
    \begin{align*}
        \highlightpartmath{\mathcal{X}_{\text{end}}} &\,\highlightpartmath{=}\, [-0.5,0.5] \times ... \times [-0.5,0.5] \subset \mathcal{R}^4 , \\
        \highlightpartmath{\mathcal{U}} &\,\highlightpartmath{=}\, [-0.5,0.1] \times [-0.1,0.5] \subset \mathcal{R}^2 , \\
        \highlightpartmath{\mathcal{W}} &\,\highlightpartmath{=}\, [-0.1,0.5] \times [-0.5,0.1] \subset \mathcal{R}^2 ,
    \end{align*}
    where $\mathcal{X}_{\text{end}}$ (...)
    Furthermore, we set the time horizon to $\highlightpartmath{\tau = [0, 1]}$.
}

\noindent
It is indispensable to provide all necessary information to let others replicate and verify your results.
Furthermore, any comparison between approaches requires full transparency regarding the example.
Providing the data can take on many forms---in the running text, more compactly in a table, by a reference, or even in a repeatability package.
