
\guideline[g:results:data]
    {Provide all data required for running the numerical experiments.}

\goodexample[{\cite[Sec.~VII.A.]{Wetzlinger2025TAC}}]{
    To showcase the general concept of our approach, we first consider the deliberately simple example of an electric circuit consisting of a resistance $R = \SI{2}{\ohm}$, a capacitor with capacity $C = \SI{1.5}{\milli\farad}$, and a coil with inductance $L = \SI{2.5}{\milli\henry}$:
    \begin{equation*}
        \begin{pmatrix} \dot{u}_C(t) \\ \dot{i}_L(t) \end{pmatrix}
        =
        \begin{pmatrix} -\frac{1}{RC} & \frac{1}{C} \\ -\frac{1}{L} & 0 \end{pmatrix}
        \begin{pmatrix} u_C(t) \\ i_L(t) \end{pmatrix}
        +
        \begin{pmatrix} 0 \\ \frac{1}{L} \end{pmatrix} u_I(t)
    \end{equation*}
    where the state is defined by the voltage at the capacitor $u_C(t)$ and the current at the coil $i_L(t)$.
    The initial set is $\mathcal{X}^0 = [1, 3]\SI{}{\volt} \times [3, 5] \SI{}{\ampere}$, the input voltage to the circuit $u_I(t)$ is uncertain within the set $\mathcal{U} = [-0.1, 0.1] \SI{}{\volt}$, and the time horizon is $t_{\text{end}} = \SI{2}{\second}$.
}

\noindent ...
% either directly (vectors, matrices, ...) or as repeatability package
% can also be in table
