
\guideline[g:results:subsections]
    {Organize examples into separate subsections of increasing complexity.}

\goodexample[{\cite[Sec.~VII.A.-C.]{Wetzlinger2023TAC}}]{
    \textit{A. Electric Circuit} \\
    \highlightpart{To showcase the general concept of our approach, we first consider} the deliberately simple example of an electric circuit (...) \\
    \textit{B. ARCH Benchmarks} \\
    \highlightpart{Next, we evaluate} our verification algorithm on benchmarks from the 2021 ARCH competition [53], where state-of-the-art reachability tools compete with one another to solve challenging verification tasks.
    We consider all linear continuous-time systems, which are (...) \\
    \textit{C. Autonomous Car} \\
    \highlightpart{Finally, we show} that our verification algorithm can handle complex verification tasks featuring time-varying specifications.
    To this end, we consider the benchmark proposed in [54], where (...)
}

\noindent
Examples numerically analyze the contributions from the main body.
As these contributions form a story of increasing complexity, even culminating in a single theorem or algorithm, let the examples follow that story explicitly, using subsections for each part.
Ideally, the resulting subsections are roughly of equal length, thereby giving similar importance to each example.
