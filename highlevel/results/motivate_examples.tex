
\guideline[g:results:motivate_examples]
    {Motivate the selection of examples.}

\goodbadexample[{\cite[Sec.~VII.A.-C.]{Wetzlinger2023TAC}}]{
    \location{Sec.~VII.A.} \\
    We first consider the deliberately simple example (...) \\
    \location{Sec.~VII.B.} \\
    Next, we evaluate our verification algorithm on benchmarks from the 2021 ARCH competition [53], which are (...) \\
    \location{Sec.~VII.C.} \\
    Finally, we evaluate our verification algorithm on the benchmark proposed in [54], where (...)
    \badexpl{The choice of examples is only stated, but not motivated.}
}{
    \location{Sec.~VII.A.} \\
    To \highlightpart{showcase the general concept of our approach}, we first consider the deliberately simple example (...) \\
    \location{Sec.~VII.B.} \\
    Next, we evaluate our verification algorithm on \highlightpart{benchmarks from the 2021 ARCH competition} [53], where \highlightpart{state-of-the-art reachability tools} compete with one another to solve challenging verification tasks. (...) \\
    \location{Sec.~VII.C.} \\
    Finally, we show that our verification algorithm can \highlightpart{handle complex verification tasks featuring time-varying specifications.} To this end, we consider the benchmark proposed in [54], where (...)
    \goodexpl{Each example is clearly motivated by a specific purpose, as highlighted.}
}

\noindent
Explain why each example was chosen.
For instance, you might use a simple example to illustrate a core concept, a larger one to demonstrate scalability, or a challenging benchmark for comparison with the state of the art.
Avoid merely listing examples or their sources without clarifying their purpose or relevance.
