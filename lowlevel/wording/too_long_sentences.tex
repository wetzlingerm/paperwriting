
\guideline[g:wording:long_sentences]
    {Avoid overly long sentences.}

\goodbadexample{
    The fundamental concept of the model predictive robustness (cf. Def. 1) is to systematically assess the model capability for adhering to rules, implying a probability that the considered predicate will be satisfied or violated, taking into account any future actions of the ego vehicle over a finite prediction horizon.
}{
    The fundamental concept of model predictive robustness (cf. Def. 1) is to systematically assess the model capability for adhering to rules\highlightpart{.}
    As a result, we obtain a probability that the considered predicate will be satisfied or violated, while taking into account any future actions of the ego vehicle over a finite time prediction horizon.
}

\noindent As a simple rule of thumb, four lines in double-column format is a suitable threshold to determine whether a sentence is too long.
This threshold is less rigid in case of lengthy terms such as ``backward reachability analysis of nonlinear systems''.
One may also use the number of clauses as a metric to decide whether a sentence should be split in two sentences.
Finally, one can also read a sentence out loud and decide whether it is digestible.
