
\guideline[g:wording:long_sentences]
    {Avoid overly long sentences.}

\goodbadexample[{\cite[Sec.~IV.B.]{Wetzlinger2025TAC}}]{
    As a wide range of different yet similar definitions are labeled \textit{backward reachable set}, the following literature review discusses the various types in order of increasing complexity, in discrete and continuous time as well as for linear and nonlinear dynamics, where uniqueness of solution trajectories and sufficient differentiability are assumed.
}{
    A wide range of different yet similar definitions are labeled \textit{backward reachable set}\highlightpart{.}
    The following literature review introduces the various types in order of increasing complexity\highlightpart{.}
    We discuss approaches in discrete and continuous time as well as for linear and nonlinear dynamics, where uniqueness of solution trajectories and sufficient differentiability are assumed.
}

\noindent
As a simple rule of thumb, four lines in double-column format is a suitable threshold to determine whether a sentence is too long;
this threshold is less rigid in case of lengthy terms.
One may also use the number of clauses as a metric to decide whether a sentence should be split, or simply read the sentence out loud and check the auditory impression.
