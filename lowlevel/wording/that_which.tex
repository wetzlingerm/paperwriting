
\guideline[g:wording:that_which]
    {Use ``that'' for necessary details and ``which'' for supplementary details.}

\goodbadexample{
    For this analysis, we require system models\highlightpart{, which} are able to represent the behavior of the corresponding real system.
}{
    For this analysis, we require system models \highlightpart{that} are able to represent the behavior of the corresponding real system.
}

\goodbadexample{
    One of the main techniques to provide safety guarantees is reachability analysis \highlightpart{that} predicts all possible future system behaviors under uncertainty in the initial state
    and input.
}{
    One of the main techniques to provide safety guarantees is reachability analysis\highlightpart{,} which predicts all possible future system behaviors under uncertainty in the initial state
    and input.
}

\noindent As a test, one can try removing the clause starting with ``that'' or ``which'' and check whether the sentence remains complete.
The commas are determined by their usage:
The conjunction ``that'' is used in restrictive clauses, which require no comma, whereas ``which'' is used in non-restrictive clauses, which require a comma.
