
\guideline[g:wording:that_which]
    {Use ``that'' for necessary details and ``which'' for supplementary details.}

\goodbadexample[{\cite[Sec.~I.A.]{Wetzlinger2023TAC}}]{
    In conclusion, there does not yet exist a fully automated parameter tuning algorithm for linear systems\highlightpart{, which} satisfies an error bound in terms of the Haussdorf distance to the exact reachable set.
}{
    In conclusion, there does not yet exist a fully automated parameter tuning algorithm for linear systems \highlightpart{that} satisfies an error bound in terms of the Haussdorf distance to the exact reachable set.
}

\goodbadexample[{\cite[Sec.~I.]{Wetzlinger2023TAC}}]{
    One of the main techniques to provide safety guarantees is reachability analysis \highlightpart{that} predicts all possible future system behaviors under uncertainty in the initial state
    and input.
}{
    One of the main techniques to provide safety guarantees is reachability analysis\highlightpart{,} which predicts all possible future system behaviors under uncertainty in the initial state
    and input.
}

\noindent As a test, one can try removing the clause starting with ``that'' or ``which'' and check whether the sentence remains complete.
The commas are determined by their usage:
The conjunction ``that'' is used in restrictive clauses, which require no comma, whereas ``which'' is used in non-restrictive clauses, which require a comma.
