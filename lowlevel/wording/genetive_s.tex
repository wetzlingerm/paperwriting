
\guideline[g:wording:genitives]
    {Prefer prepositional genitives over possessive genitives.}

\goodbadexample[{\cite[Sec.~IV.]{Wetzlinger2024CSL}}]{
    The choice of set representations for computing an inner approximation with the formulae (20) and (23) determines the \highlightpart{resulting reachability algorithm's time complexity} in the state dimension.
}{
    The choice of set representations for computing an inner approximation with the formulae (20) and (23) determines the \highlightpart{time complexity of the resulting reachability algorithm} in the state dimension.
}

\noindent Descriptions often address abstract concepts where a possessive relationship is less suitable because the "owner" is inanimate.
Furthermore, prepositional genitives offer more flexibility and better express associative relationships, as is often the case.
One notable exception to this guideline is proper nouns.
