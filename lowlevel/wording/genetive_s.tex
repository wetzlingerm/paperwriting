
\guideline[g:wording:genitives]
    {Prefer prepositional genitives over possessive genitives.}

\goodbadexample[{\cite[Sec.~4.1]{Wetzlinger2024ARCH1}}]{
    For the \highlightpart{eigenvalues'} real and imaginary parts, we use the intervals $\mathcal{I}_{\text{real}} = [-5,-1]$ and $\mathcal{I}_{\text{imag}} = [-0.5, 0.5]$.
}{
    For the real and imaginary parts \highlightpart{of the eigenvalues}, we use the intervals $\mathcal{I}_{\text{real}} = [-5,-1]$ and $\mathcal{I}_{\text{imag}} = [-0.5, 0.5]$.
}

\noindent
A possessive relationship is less suitable because the owning entity in scientific texts is often inanimate.
Furthermore, prepositional genitives offer more flexibility and better express associative relationships.
One notable exception to this guideline is proper nouns.
