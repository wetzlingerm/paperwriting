
\guideline[g:references:in_environment_title]
    {Cite the reference in the title of environments.}

\goodbadexample[{\cite[Definition 3]{Wetzlinger2025TAC}}]{
    Next, we introduce zonotopes \highlightpart{[8, Def. 1]}. \\
    \textit{Definition 3 (Zonotope):}
    Given a center $c \in \mathbb{R}^n$ and $\gamma \in \mathbb{N}$ generators stored in the columns in the generator matrix $G \in \mathbb{R}^{n \times \gamma}$, a zonotope $\mathcal{Z} \subset \mathbb{R}^n$ is
    \begin{equation*}
        \mathcal{Z} \coloneqq \bigg\{ c + \sum_{i=1}^{\gamma} G_{(\cdot,i)} \alpha_i \, \Big| \, \alpha_i \in [-1,1] \bigg\} .
    \end{equation*}
    We use the shorthand $\mathcal{Z} = \langle c, G \rangle_Z$.
}{
    Next, we introduce zonotopes. \\
    \textit{Definition 3 (Zonotope \highlightpart{[8, Def. 1]}):}
    Given a center $c \in \mathbb{R}^n$ and $\gamma \in \mathbb{N}$ generators stored in the columns in the generator matrix $G \in \mathbb{R}^{n \times \gamma}$, a zonotope $\mathcal{Z} \subset \mathbb{R}^n$ is
    \begin{equation*}
        \mathcal{Z} \coloneqq \bigg\{ c + \sum_{i=1}^{\gamma} G_{(\cdot,i)} \alpha_i \, \Big| \, \alpha_i \in [-1,1] \bigg\} .
    \end{equation*}
    We use the shorthand $\mathcal{Z} = \langle c, G \rangle_Z$.
}

\noindent
Environments stand out from the surrounding text, letting readers expect self-contained information.
Hence, the reference is better positioned within the environment than before or after.
