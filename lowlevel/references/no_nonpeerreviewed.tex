
\guideline[g:references:no_nonpeerreviewed]
    {Avoid citing non-peer-reviewed work.}

\goodbadexample[{Adapted from \cite[Sec.~II.A.]{Wetzlinger2024CSL}}]{
    \location{Preliminaries} \\
    For three compact, convex, nonempty sets $\mathcal{S}_1, \mathcal{S}_2, \mathcal{S}_3 \subset \mathbb{R}^n$, it holds that [15, eq. (2)]
    \begin{equation*}
        \mathcal{S}_1 \ominus ( \mathcal{S}_2 \oplus \mathcal{S}_3 )
        = ( \mathcal{S}_1 \ominus \mathcal{S}_2 ) \ominus \mathcal{S}_3
        = ( \mathcal{S}_1 \ominus \mathcal{S}_3 ) \ominus \mathcal{S}_2 .
    \end{equation*}
    \location{References} \\
    $[15]$
    F. Firstauthor, S. Secondauthor
    ``A non-peer-reviewed paper,''
    2024, \highlightpart{\textit{arXiv:2401.99999}}.
}{
    \location{Preliminaries} \\
    For three compact, convex, nonempty sets $\mathcal{S}_1, \mathcal{S}_2, \mathcal{S}_3 \subset \mathbb{R}^n$, it holds that [15, eq. (2)]
    \begin{equation*}
        \mathcal{S}_1 \ominus ( \mathcal{S}_2 \oplus \mathcal{S}_3 )
        = ( \mathcal{S}_1 \ominus \mathcal{S}_2 ) \ominus \mathcal{S}_3
        = ( \mathcal{S}_1 \ominus \mathcal{S}_3 ) \ominus \mathcal{S}_2 .
    \end{equation*}
    \location{References} \\
    $[15]$
    F. Firstauthor, S. Secondauthor
    ``A peer-reviewed paper,'' in
    \highlightpart{\textit{Proceedings of the Conference on Science}}, 2024, pp. 1--10.
}

\noindent
Non-peer-reviewed work has not undergone formal review to ensure the correctness of its content and is therefore more likely to contain errors.
This is especially important when the citation supports key claims or derivations, but less critical when using numerical examples or content that can be independently verified.
