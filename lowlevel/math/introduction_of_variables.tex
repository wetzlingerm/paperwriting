
\guideline[g:math:introduction_of_variables]
    {Introduce variables with a name and a domain.}

\goodbadexample[{\cite[Sec.~III.]{Wetzlinger2025TAC}}]{
    A solution to (31) starting from the initial state $\highlightpartmath{x_0}$ using an input trajectory $\highlightpartmath{u(\cdot)}$ and a disturbance trajectory $\highlightpartmath{w(\cdot)}$ is written as $\xi(t; x_0, u(\cdot),w(\cdot))$.
}{
    \highlightpart{We use $\mathbb{U}$ to denote the set of all input trajectories $u(\cdot)$ for which $\forall t \in [0, t_{\text{end}}]\colon u(t) \in \mathbb{U}$ holds and analogously $\mathbb{W}$ for the set of all disturbances trajectories $w(\cdot)$.}
    A solution to (31) \highlightpart{at time $t$} starting from the initial state $x_0 \, \highlightpartmath{\in \mathbb{R}^n}$ using an input trajectory $u(\cdot) \, \highlightpartmath{\in \mathbb{U}}$ and a disturbance trajectory $w(\cdot) \, \highlightpartmath{\in \mathbb{W}}$ is written as $\xi(t; x_0, u(\cdot),w(\cdot))$.
}

\noindent Explicitly mentioning the name and the domain is helpful because it prevents misunderstandings, especially for readers who may not be familiar with certain conventions or who may follow different ones.
The introduction of variables including a domain is of particular importance in definitions.
New variables in non-inlined equations should be introduced before or immediately after the equation environment.
