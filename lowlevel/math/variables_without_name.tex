
\guideline[g:math:variable_name]
    {Use variables along with their name, unless the variable is repeated multiple times within the same paragraph.}

\goodbadexample{
    We define a driving corridor $K_n$ as an extracted set \highlightpart{of }\highlightpart{$\mathcal{R}_{n,k,i}^S$} \highlightpart{from }\highlightpart{$G_n^S$} with \highlightpart{their }\highlightpart{$D_{n,k,i}$} being spatially connected at step $k$ and temporally connected through the edges \highlightpart{in }\highlightpart{$G_n^S$}.
}{
    We define a driving corridor $K_n$ as an extracted set \highlightpart{of the rule-compliant reachable set} \highlightpart{$\mathcal{R}_{n,k,i}^S$} \highlightpart{from the rule-compliant reachability graph} \highlightpart{$G_n^S$} where \highlightpart{the drivable areas} \highlightpart{$D_{n,k,i}$} are spatially connected at step $k$ and temporally connected through the edges \highlightpart{in }\highlightpart{$G_n^S$}.
}

\noindent Repeating the name of the variable enhances the fluidity of the running text, as the name conveys much more meaning than the variable alone.
Even more so, as one usually thinks of the meaning of a variable when reading it---which is precisely the name.
If the name has been mentioned recently (in the same sentence, the same paragraph, or within an explanation that spans several paragraphs but outlines a single concept), one can expect the reader to remember it and therefore omit the name.

