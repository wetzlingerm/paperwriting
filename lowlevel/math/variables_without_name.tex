
\guideline[g:math:variable_name]
    {In general, use variables along with their name.}

\goodbadexample[{\cite[Sec.~III.B.]{Wetzlinger2024CSL}}]{
    Each $\highlightpartmath{x(\Delta t) \in \mathcal{R}(\Delta t)}$ can be expressed using $\highlightpartmath{x(0) \in \mathcal{X}_0}$ and a $\highlightpartmath{z(x(0)) \in \widehat{\mathcal{R}}(\mathcal{L}(\mathcal{R}(\tau_0)))}$ that may depend on \highlightpart{$x(0)$}:
}{
    Each \highlightpart{successor state $x(\Delta t) \in \mathcal{R}(\Delta t)$} can be expressed using an \highlightpart{initial state $x(0) \in \mathcal{X}_0$} and an \highlightpart{error vector $z(x(0)) \in \widehat{\mathcal{R}}(\mathcal{L}(\mathcal{R}(\tau_0)))$} that may depend on \highlightpart{$x(0)$}:
}

\noindent Repeating the name of the variable enhances the fluidity of the running text, as the name conveys much more meaning than the variable alone.
Even more so, as one usually thinks of the meaning of a variable when reading it---which is precisely the name.
If the name has been mentioned recently (in the same sentence, the same paragraph, or within an explanation that spans several paragraphs but outlines a single concept), one can expect the reader to remember it and therefore omit the name.

