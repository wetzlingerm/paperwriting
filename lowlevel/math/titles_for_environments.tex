
\guideline[g:math:titles_for_environments]
    {Consider giving your environments a title.}

\goodexample[{\cite[Proposition 8]{Wetzlinger2023TAC}}]{
    \textit{Proposition 8 (\highlightpart{Intersection check}):}
    A polytope $\mathcal{P} = \langle C, d \rangle_H \subset \mathbb{R}^n$ and a constrained zonotope $\mathcal{CZ} = \langle c, G, A, b \rangle_{CZ} \subset \mathbb{R}^n$ intersect if $\nu \leq 0$ computed by the linear program
    \begin{align*}
        &\nu = \min_{x \in \mathbb{R}^n, \, \alpha \in \mathbb{R}^\gamma, \, \delta \in \mathbb{R}} \delta \\
        &\hspace{42pt} \text{s.t.} ~ \forall i \in \{1,\dots,a\}\colon C_{(i,\cdot)}x - d_{(i)} \leq \delta \\
        &\hspace{42pt} x = c + G\alpha, A\alpha = b, \alpha \in [-\mathbf{1}, \mathbf{1}].
    \end{align*}
    \textit{Proof:} If $\forall i \in \{1,\dots,a\}\colon C_{(i,\cdot)}x - d_{(i)} \leq \delta \leq 0$, then there exists a point $x \in \mathcal{CZ}$ that is also contained in $\mathcal{P}$.
}

\noindent
A title sets a frame for the content of the environment, letting the reader integrate the information more easily into the flow of the running text.
Titles are strongly recommended for environments with concise semantic meaning, such as assumptions, definitions, propositions, and theorems; only to a lesser degree for more technical environments like lemmas or corollaries.
