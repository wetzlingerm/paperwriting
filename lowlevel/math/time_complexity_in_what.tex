
\guideline[g:math:time_space_complexity]
    {Use time/space complexities with explicit dependencies.}

\goodbadexample[{\cite[Sec.~IV.]{Wetzlinger2024CSL}}]{
    The choice of set representations for computing an inner approximation with the formulae (20) and (23) determines the \highlightpart{time complexity} of the resulting reachability algorithm.
}{
    The choice of set representations for computing an inner approximation with the formulae (20) and (23) determines the \highlightpart{time complexity} of the resulting reachability algorithm \highlightpart{in the state dimension}.
}

\noindent Often, the time/space complexity may be inferred from context.
However, the statement is incomplete without mentioning the variable with respect to which a certain behavior is exhibited.
This is particularly useful for readers who are skimming the text for this specific information.
