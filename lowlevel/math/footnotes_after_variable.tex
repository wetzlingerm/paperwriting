
\guideline[g:math:footnotes_after_variables]
    {Avoid placing footnote marks after variables.}

\goodbadexample{
    ...
}{
    ...
}
% \goodbadexample{
%     Once the trajectory $x$ violates the rule \highlightpart{$\phi^3$}, our repairer aims to solve: (...) \smallskip

%     \textsuperscript{3}\begin{footnotesize}%
%     Multiple violated rules are conjoined by $\phi = \bigwedge_{v=1}^{n_v} \phi_v$.
%     \end{footnotesize}
% }{
%     Once the trajectory $x$ violates the rule \highlightpart{$\phi$}, our repairer aims to solve\highlightpart{\textsuperscript{3}}: (...) \smallskip

%     \textsuperscript{3}\begin{footnotesize}%
%     Multiple violated rules are conjoined by $\phi = \bigwedge_{v=1}^{n_v} \phi_v$.
%     \end{footnotesize}
% }
% TODO: Come up with new example

\noindent Importantly, footnote marks can be confused with exponentiation.
Although this may not be a mathematical issue---in the context of the above example, the ``exponentiation of a rule'' would be undefined---, it is still visually misleading.
