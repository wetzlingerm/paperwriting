
\guideline[g:typography:hyphen_non_standard_words]
    {Avoid writing non-standard words with hyphens.}

\goodbadexample[{\cite[Sec.~I.]{Wetzlinger2024CSL}}]{
    One can also obtain \highlightpart{inner-approximations} via \highlightpart{optimization-based} techniques.
}{
    One can also obtain \highlightpart{inner approximations} via \highlightpart{optimization-based} techniques.
}

\noindent \LaTeX{} implements the typographic convention that hyphenated words are not split across lines.
Since double-column formats are prone to splitting words due to the relatively short text width, one may easily encounter formatting issues when using (long) hyphenated words, such as ``outer-approximation'' in the example above.
The example also illustrates that some hyphenated words are standard adjectives where the hyphen is necessary, such as ``well-known'', ``high-level'', and ``real-time''.
