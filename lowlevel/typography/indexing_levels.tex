
\guideline[g:typography:indexing_hierarchies]
    {Avoid multiple hierarchies of indices.}

\goodbadexample[{\cite[Eq.~(47)]{Wetzlinger2025TAC}}]{
    For each $\mathcal{R}_{\exists}(-\tau; u_j(\cdot))$, we evaluate the support function in the direction $\ell_j$ [53, Sec.~4.1]
    \begin{align*}
        &\rho_{\highlightpartmath{\widehat{\mathcal{R}}_{\exists}(-\tau_k; u_j(\cdot))}} (\ell_j) \\
        &= \max \big\{ \rho_{\highlightpartmath{\mathcal{X}_{\text{end}}}} \big( (e^{-At_k})^\top \ell_j \big) ,
           \rho_{\highlightpartmath{\mathcal{X}_{\text{end}}}} \big( (e^{-At_{k+1}})^\top \ell_j \big) \big\} \\
           &\quad+ \rho_{\highlightpartmath{\boldsymbol{\mathcal{F}} \mathcal{X}_{\text{end}}}} \big( (e^{-At_{k+1}})^\top \ell_j \big)
           + \rho_{\highlightpartmath{\widehat{\mathcal{Z}}_{\mathcal{W}}(-\tau_k)}} \big( \ell_j \big)
           + \beta_j
        \end{align*}
        }{
    For each $\mathcal{R}_{\exists}(-\tau; u_j(\cdot))$, we evaluate the support function in the direction $\ell_j$ [53, Sec.~4.1]
    \begin{align*}
        &\rho \big( \highlightpartmath{\widehat{\mathcal{R}}_{\exists}(-\tau_k; u_j(\cdot))}, \ell_j \big) \\
        &= \max \big\{ \rho \big( \highlightpartmath{\mathcal{X}_{\text{end}}}, (e^{-At_k})^\top \ell_j \big) ,
            \rho \big( \highlightpartmath{\mathcal{X}_{\text{end}}}, (e^{-At_{k+1}})^\top \ell_j \big) \big\} \\
        &\quad+ \rho \big( \highlightpartmath{\boldsymbol{\mathcal{F}} \mathcal{X}_{\text{end}}}, (e^{-At_{k+1}})^\top \ell_j \big)
        + \rho \big( \highlightpartmath{\widehat{\mathcal{Z}}_{\mathcal{W}}(-\tau_k)}, \ell_j \big)
        + \beta_j
    \end{align*}
}

\noindent
Multiple levels of indices in equations are cumbersome to read because they create visual clutter.
To improve readability, functions or variables with complex indices can be redefined using simpler notations or intermediate variables.
