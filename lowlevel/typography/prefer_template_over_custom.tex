
\guideline[g:typography:custom_formatting]
    {Avoid custom formatting.}

\goodbadexample[{\cite[Sec.~VIII.]{Wetzlinger2023TAC}}]{
    Our proposed algorithm has the following advantages and disadvantages:
    \begin{compactitemize}
        \item[\highlightpart{$+$}] Our verification algorithm can handle complex verification tasks involving time-varying specifications.
        \item[\highlightpart{$+$}] The autonomy of our approach enables practitioners to verify
        or falsify safety specifications with requiring specialized knowledge.
        \item[\highlightpart{$-$}] Other tools solve high-dimensional benchmarks often faster than our approach through tailored algorithms.
        \item[\highlightpart{$-$}] In cases where there is no initial uncertainty, our algorithm is unable to falsify the system, as the computed inner approximation of the reachable set will always be empty.
    \end{compactitemize}
    Let us briefly address the strengths and weaknesses of our proposed approach:
    \highlightpart{A major advantage} of our verification algorithm is its ability to handle complex verification tasks involving time-varying specifications, and its autonomous nature allows practitioners to easily verify or falsify safety specifications without requiring specialized knowledge.
    \highlightpart{However}, our approach may be slower compared to other tools, which often solve high-dimensional benchmarks faster through tailored algorithms.
    \highlightpart{Additionally}, in cases where there is no initial uncertainty, our algorithm is unable to falsify the system, as the computed inner approximation of the reachable set will always be empty.
}

\noindent The formatting options provided by the \LaTeX{} template---such as headings, paragraphs, (un)ordered lists, and special environments (e.g., algorithms, tables, figures)---offer sufficient flexibility.
The need for custom formatting may suggest that the content is not well-structured or clearly phrased.
There are multiple ways to convert custom formatting into built-in formatting by using transitional phrases to establish a clear relationship between parts of the text (as used in the example above), by using a more rigid structural element that can be referenced, or by using an explanatory figure for more complex relationships.
