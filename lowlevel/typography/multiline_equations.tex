
\guideline[g:typography:multiline_equations]
    {Vertically align relational operators in multiline right-hand sides.}

\goodbadexample[{\cite[Appendix]{Wetzlinger2025TAC}}]{
    We insert this in (57) to obtain
    \begin{align*}
        &\mathcal{S}_2
        \overset{(58)}{\highlightpartmath{\subseteq}}
        \big( \widehat{\mathcal{R}}_\exists (-\tau_0; u_0) \cup \dots 
            \cup \widehat{\mathcal{R}}_\exists (-\tau_{\sigma-1}; u_0) \big) \\
        &\highlightpartmath{\cap} \; \langle N, p^{(1)} \rangle_H \cap \dots \cap \langle N, p^{(q)} \rangle_H \\
        &\highlightpartmath{=} \big( \widehat{\mathcal{R}}_\exists (-\tau_0; u_0) \cup \dots 
            \cup \widehat{\mathcal{R}}_\exists (-\tau_{\sigma-1}; u_0) \big) \cap \langle N, p \rangle_H .
    \end{align*}
}{
    We insert this in (57) to obtain
    \begin{align*}
        \mathcal{S}_2
        \overset{(58)}&{\highlightpartmath{\subseteq}}
        \big( \widehat{\mathcal{R}}_\exists (-\tau_0; u_0) \cup \dots 
            \cup \widehat{\mathcal{R}}_\exists (-\tau_{\sigma-1}; u_0) \big) \\
        &\qquad \highlightpartmath{\cap} \; \langle N, p^{(1)} \rangle_H \cap \dots \cap \langle N, p^{(q)} \rangle_H \\
        &\highlightpartmath{=} \, \big( \widehat{\mathcal{R}}_\exists (-\tau_0; u_0) \cup \dots 
            \cup \widehat{\mathcal{R}}_\exists (-\tau_{\sigma-1}; u_0) \big) \cap \langle N, p \rangle_H .
    \end{align*}
}

\noindent 
Maintaining vertical alignment of relational operators enhances the readability of expressions.
If a line exceeds the right margin, indent the continuation line to indicate that it follows from the previous one. 
This guideline may be relaxed for formatting considerations, such as preventing awkward line breaks.
