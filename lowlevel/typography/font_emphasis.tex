
\guideline[g:typography:font_emphasis]
    {Avoid using italics or underlines for emphasis.}

\goodbadexample[{\cite[Sec.~VI.B.2)]{Wetzlinger2025TAC}}]{
    According to [12, Thm. 3], the exact conversion operation $\texttt{CZ}(\mathcal{P})$ in Algorithm 1 works with \highlightpart{\textit{any}} enclosure of $\mathcal{P}$.
}{
    According to [12, Thm. 3], the exact conversion operation $\texttt{CZ}(\mathcal{P})$ in Algorithm 1 works with \highlightpart{any} enclosure of $\mathcal{P}$.
}

\noindent The context and phrasing of a sentence sufficiently decide its emphasis.
In the example above, highlighting the interrogative pronoun/adverb is unnecessary, as the sentences are already constructed in a way that avoids any potential misunderstandings.
Also, it is generally not necessary to use italics for text within quotation marks.
