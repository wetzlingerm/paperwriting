
\guideline[g:typography:en_em_dash]
    {Use dashes with proper spacing.}

\goodbadexample[{\cite[Sec.~1]{Wetzlinger2021HSCC}}]{
    The performance of these algorithms heavily relies on the correct setting of algorithm parameters \highlightpart{-} a safety property may not be verified although it is satisfied by the exact reachable set.
}{
    The performance of these algorithms heavily relies on the correct setting of algorithm parameters\highlightpart{---}a safety property may not be verified although it is satisfied by the exact reachable set.
}

\noindent There are two types of dashes: en dashes (--) and em dashes (---).
Their names originate from the relation of the length of the dash to the width of the letters ``N'' and ``M'', respectively; both are longer than the standard hyphen.
En dashes are used for ranges and connections, e.g., ``pages 1--10'' and ``North--South divide'', whereas em dashes are used for breaks in thought and emphasis, as shown in the example above.
Some venues explicity require British English, where it is more common to use commas or parentheses instead of em dashes.
