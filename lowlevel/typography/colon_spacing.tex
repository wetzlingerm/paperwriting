
\guideline[g:typography:colon_spacing]
    {Check the spacing in quantified expressions.}

\goodbadexample[{\cite[Sec.~2.2]{Wetzlinger2021HSCC}}]{
    The presented techniques for automated parameter adaptation are applied to nonlinear systems
    \begin{equation*}
        \dot{x}(t) = f(x(t),u(t)),
    \end{equation*}
    where $\highlightpartmath{f: \mathbb{R} \to \mathbb{R}^n}$ is a sufficiently smooth nonlinear function, $x(t) \in \mathbb{R}^n$ is the state vector, and $u(t) \in \mathbb{R}^m$ is the input vector.
}{
    The presented techniques for automated parameter adaptation are applied to nonlinear systems
    \begin{equation*}
        \dot{x}(t) = f(x(t),u(t)),
    \end{equation*}
    where $\highlightpartmath{f\colon \mathbb{R} \to \mathbb{R}^n}$ is a sufficiently smooth nonlinear function, $x(t) \in \mathbb{R}^n$ is the state vector, and $u(t) \in \mathbb{R}^m$ is the input vector.
}

\noindent The regular colon ``:'' is used in set builder notation as an alternative to a vertical bar and, less commonly, to express proportions.
In both cases, equal spacing on either side of the colon is appropriate to reflect the relationship between the elements.
Quantified expressions, however, do not consist of two independent parts, and are better represented with the less spacing to the left than to the right, as provided in \LaTeX{} by ``\textbackslash colon''.
