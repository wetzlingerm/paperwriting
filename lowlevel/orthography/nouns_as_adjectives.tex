
\guideline[g:orthography:hyphenate_nouns_as_adjectives]
    {Hyphenate non-standard compound modifiers.}

\goodbadexample[{\cite[Sec.~I.]{Wetzlinger2024CSL}}]{
    Using an \highlightpart{on the fly} linearization of the nonlinear autonomous dynamics, we obtain an affine dynamics and a \highlightpart{higher order} error dynamics.
}{
    Using an \highlightpart{on-the-fly} linearization of the nonlinear autonomous dynamics, we obtain an affine dynamics and a \highlightpart{higher-order} error dynamics.
}

\noindent
A compound modifier functions as an adjective formed from multiple words, e.g., nouns or adjective-noun combinations.
Adverbs ending in ``-ly'' are a notable exception and are not hyphenated.

\goodbadexample[{\cite[Sec.~I.]{Wetzlinger2023TAC}}]{
    To overcome this limitation, we present the first \highlightpart{fully-automated} verification algorithm for linear time-invariant systems.
}{
    To overcome this limitation, we present the first \highlightpart{fully automated} verification algorithm for linear time-invariant systems.
}
