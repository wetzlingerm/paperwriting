
\guideline[g:orthography:comma_clauses]
    {Use commas when introducing restrictive clauses, omit any when introducing non-restrictive clauses.}

\goodbadexample{
    Temporal logic is suited to formalize traffic rules [1--5]\highlightempty{} as it can capture their spatial and temporal dependencies well.
}{
    Temporal logic is suited to formalize traffic rules [1--5]\highlightpart{,} as it can capture their spatial and temporal dependencies well.
}

\goodbadexample{
    We note that an increased timeout would not help\highlightpart{,} as the path planner only looks for possible solutions and does not optimize the found path.
}{
    We note that an increased timeout would not help\highlightempty{} as the path planner only looks for possible solutions and does not optimize the found path.
}

\noindent Restrictive clauses provide essential information defining or limiting the modified noun, they cannot be removed without changing the meaning of the sentence.
In contrast, non-restrictive clauses provide additional information that can be removed without changing the meaning of the sentence; they are often used to add extra details or explanations.
