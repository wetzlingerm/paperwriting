
\guideline[g:orthography:comma_clauses]
    {Omit commas for restrictive clauses; use commas for non-restrictive clauses.}

\goodbadexample[{\cite[Sec.~I.]{Wetzlinger2023TAC}}]{
    Deploying cyber-physical systems in safety-critical environments requires formal verification techniques to ensure correctness with respect to the desired functionality\highlightpart{} as failures can lead to severe economic or ecological consequences and loss of human life.
}{
    Deploying cyber-physical systems in safety-critical environments requires formal verification techniques to ensure correctness with respect to the desired functionality\highlightpart{,} as failures can lead to severe economic or ecological consequences and loss of human life.
}

\goodbadexample[{\cite[Sec.~VII.A.]{Wetzlinger2025TAC}}]{
    While the runtime complexity of our proposed algorithms only scales linearly with the number of time steps, the computation time of HJ reachability strongly depends on the partitioning on the grid\highlightpart{,} as it suffers from the curse of dimensionality.
}{
    While the runtime complexity of our proposed algorithms only scales linearly with the number of time steps, the computation time of HJ reachability strongly depends on the partitioning on the grid\highlightpart{} as it suffers from the curse of dimensionality.
}

\noindent Restrictive clauses provide essential information defining or limiting the modified noun, they cannot be removed without changing the meaning of the sentence.
In contrast, non-restrictive clauses provide additional information that can be removed without changing the meaning of the sentence; they are often used to add extra details or explanations.
