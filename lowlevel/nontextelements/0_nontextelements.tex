
\chapter{Figures, Tables, and Pseudocode}
\label{ch:nontextelements}

Non-text elements such as figures, tables, and pseudocode play a crucial role in organizing the flow of the work.
By effectively using these components---in \LaTeX{}, via the use of ``environments''---, authors ensure an effective communication of their content.

% general

\guideline{Refer to all non-text elements in the running text.}

\goodbadexample{
    ...
}{
    ...
}

\noindent The placement of non-text elements like figures, tables, and algorithms with respect to the running text is typically managed by \LaTeX{}, so readers often skip over them until a reference directs them to a specific element.
Therefore, it is good practice to explicitly reference each non-text element, ideally in a separate paragraph explaining its content.
% An alternative formulation to this guideline is:
% ``Only use environments if they are referenced at least once later.''


\guideline[g:nontext:placement_top]
    {Place non-text elements on the top of a page or column.}

\goodbadexample[{\cite[Sec.~II.B.]{Wetzlinger2025TAC}}]{
    \location{Top of the page} \\
    (...) \\
    The exact conversion from a polytope to a constrained zonotopes, denoted by $\textsc{CZ}(\mathcal{P})$, is computed using Algorithm 1, which implements [12, Thm. 1].
    
    \algruletop{}
\textbf{Algorithm 1} ~ Conversion: Polytope to constrained zonotope. \\
\textbf{Input:} Polytope $\mathcal{P} = \langle H, d \rangle_P$ \\
\textbf{Output:} Constrained zonotope $\mathcal{CZ} = \langle c, G, K, l \rangle_{CZ}$

\begin{algorithmic}[1]
	\State $\langle c, G \rangle_Z \gets \textsc{box}( \mathcal{P} )$
    \State $\forall j \in \mathbb{N}_{[1,...,h]}\colon o_{(j)} \gets -\rho\big( \langle c, G \rangle_Z,-H_{(j,\cdot)}^\top \big)$
    \State $G \gets [G \;\, \mathbf{0}]$, $K \gets [H G \;\, \tfrac{1}{2}\mathrm{diag}(o - d)]$, $l \gets \tfrac{1}{2}(d+o) - Hc$
    \State $\mathcal{CZ} \gets \langle c, G, K, l \rangle_{CZ}$
\end{algorithmic}
\algrulebottom{}


    The convex hull can be computed according to [13, Thm. 5] and the multiplication with an intervalmatrix $\mathbf{M}\mathcal{CZ}$ follows from (16).
}{
    \location{Top of the page} \\
    \algruletop{}
\textbf{Algorithm 1} ~ Conversion: Polytope to constrained zonotope. \\
\textbf{Input:} Polytope $\mathcal{P} = \langle H, d \rangle_P$ \\
\textbf{Output:} Constrained zonotope $\mathcal{CZ} = \langle c, G, K, l \rangle_{CZ}$

\begin{algorithmic}[1]
	\State $\langle c, G \rangle_Z \gets \textsc{box}( \mathcal{P} )$
    \State $\forall j \in \mathbb{N}_{[1,...,h]}\colon o_{(j)} \gets -\rho\big( \langle c, G \rangle_Z,-H_{(j,\cdot)}^\top \big)$
    \State $G \gets [G \;\, \mathbf{0}]$, $K \gets [H G \;\, \tfrac{1}{2}\mathrm{diag}(o - d)]$, $l \gets \tfrac{1}{2}(d+o) - Hc$
    \State $\mathcal{CZ} \gets \langle c, G, K, l \rangle_{CZ}$
\end{algorithmic}
\algrulebottom{}

    (...) \\
    The exact conversion from a polytope to a constrained zonotopes, denoted by $\textsc{CZ}(\mathcal{P})$, is computed using Algorithm 1, which implements [12, Thm. 1].
    The convex hull can be computed according to [13, Thm. 5] and the multiplication with an intervalmatrix $\mathbf{M}\mathcal{CZ}$ follows from (16).
}

\noindent Placing non-text elements at the top of the page or column offers several advantages:
First, these elements are easily accessible when referenced in the text.
Second, non-text elements often condense important information, warranting prominent placement, such as at the top of the page.
Third, this placement preserves the flow of the running text, contributing to optimizing space usage.

% figures

\guideline{Figures: Use vector graphics.}

\goodbadexample{
    ...
}{
    ...
}

\noindent No exceptions.
Since figures typically capture the reader's attention first, investing in high-quality figures is well worth the effort.
Many applications offer the ability to export a drawn canvas as a vector graphics file.



\guideline{Figures: Use the same font and font sizes as in the running text.}

\goodbadexample{
    ...
}{
    ...
}

\noindent Using the same font and font sizes in figures as in the running text allows the figure to blend seamlessly with the rest of the document, contributing to a cohesive overall appearance.
This is most easily achieved with the \LaTeX{} package \emph{TikZ}, as it compiles the text in figures using the same settings as the rest of the document.


\guideline[g:nontext:figure_colorblind]
    {Figures: Ensure colors are distinguishable by colorblind people.}

\goodbadexample{
    ...
}{
    ...
}

\noindent Color blindness affects approximately $8\%$ of men and $0.5\%$ of women worldwide, or roughly $1$ in $12$ men and $1$ in $200$ women \footfullcite{Gordon1998}.
As a result, it is highly likely that some readers would benefit from color choices that accommodate this condition.


\guideline[g:nontext:figure_linestyles]
    {Figures: Consider using different line styles.}

\goodbadexample{
    ...
}{
    ...
}

\noindent Varying line styles help differentiate information, making figures more accessible and easier to interpret.
Additionally, line styles are easier to reference in the caption, as colors may appear differently in black-and-white prints.


\guideline[g:nontext:figure_caption]
    {Figures: Captions should provide a brief, independent summary.}

% bad: caption that does not fully explain what is going on
% note: some parts may be explained by the legend, too
\goodbadexample{
    ...
}{
    ...
}

\noindent A good caption provides a concise description of the figure's content, enabling it to be understood without reference to the running text.
To achieve this, all objects, lines, and shading should be clearly explained.

% tables

\guideline[g:nontext:table_horizontal_lines]
    {Tables: Avoid horizontal lines.}

\goodbadexample{
    ...
}{
    ...
}

\noindent Excessive use of horizontal lines can make a table look cluttered and hinder the reader's ability to quickly scan and interpret the data, as the structure provided by rows and columns is often enough to visually separate data.
Horizontal lines are primarily used for sectioning, e.g., between column headers and data, or between different segments of similar data.


\guideline{Tables: Highlight the best values in bold when comparing results.}

\goodbadexample{
    ...
}{
    ...
}

\noindent Highlighting the best values adds a layer of visual hierarchy to the table, helping readers more easily compare data.
This can also support arguments about the quality of the results, as multiple bold entries suggest strong performance.


\guideline[g:nontext:table_caption]
    {Tables: Explain all variables used in the caption.}

\goodbadexample{
    ...
}{
    ...
}

\noindent To enable readers to understand the table independently of the running text, the caption should concisely describe all variables used, even if their meaning is consistent throughout the paper.
This explanation usually comes after a brief summary of the table's content.

% pseudocode

\guideline[g:nontext:pseudocode_language_specific]
    {Avoid language-specific functions in pseudocode.}

\goodbadexample{
    \textbf{Algorithm 1.} A good algorithm. \\
    Input: Matrix $G$, (...) \\
    Output: (...) \\
    \begin{scriptsize}
        1:
    \end{scriptsize}
    $m \gets \highlightpartmath{\textsc{shape}(G,2)}$ \\
    (...)
}{
    \textbf{Algorithm 1.} A good algorithm. \\
    Input: Matrix $G \in \mathbb{R}^{n \times \highlightpartmath{m}}$, (...) \\
    Output: (...) \\
    (...)
}

\noindent Using language-specific functions may hinder understanding across diverse audiences, as some readers may not be familiar with any given programming language.
In simple cases, there may be a more rigorous way to achieve the same goal, as shown in the example above.
If a language-specific function is particularly useful, its behavior can be briefly described in the running text.


\guideline[g:nontext:pseudocode_explain_operators]
    {Pseudocode: Ensure that all operators are explained or referenced.}

\goodbadexample{
    ...
}{
    ...
}

\noindent Since pseudocode aims to be concise, it's often better to use named operations rather than detailing every computation, especially for standard tasks, e.g., sorting.
However, it is still good practice to briefly explain what each operation does, particularly in terms of its input and output arguments.


\guideline[g:nontext:pseudocode_io_arguments]
    {Pseudocode: Strictly define the input and output arguments.}

% use a short pseudocode...
\goodexample{
    ...
}

\noindent A clear definition of input and output arguments integrates these variables more seamlessly into the surrounding text and facilitates cross-referencing between sections.
It also helps those re-implementing a specific pseudocode.

