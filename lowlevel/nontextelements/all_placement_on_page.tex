
\guideline[g:nontext:placement_top]
    {Place non-text elements on the top of a page or column.}

\goodbadexample[{\cite[Sec.~II.B.]{Wetzlinger2025TAC}}]{
    \location{Top of the page} \\
    (...) \\
    The exact conversion from a polytope to a constrained zonotopes, denoted by $\textsc{CZ}(\mathcal{P})$, is computed using Algorithm 1, which implements [12, Thm. 1].
    
    \algruletop{}
\textbf{Algorithm 1} ~ Conversion: Polytope to constrained zonotope. \\
\textbf{Input:} Polytope $\mathcal{P} = \langle H, d \rangle_P$ \\
\textbf{Output:} Constrained zonotope $\mathcal{CZ} = \langle c, G, K, l \rangle_{CZ}$

\begin{algorithmic}[1]
	\State $\langle c, G \rangle_Z \gets \textsc{box}( \mathcal{P} )$
    \State $\forall j \in \mathbb{N}_{[1,...,h]}\colon o_{(j)} \gets -\rho\big( \langle c, G \rangle_Z,-H_{(j,\cdot)}^\top \big)$
    \State $G \gets [G \;\, \mathbf{0}]$, $K \gets [H G \;\, \tfrac{1}{2}\mathrm{diag}(o - d)]$, $l \gets \tfrac{1}{2}(d+o) - Hc$
    \State $\mathcal{CZ} \gets \langle c, G, K, l \rangle_{CZ}$
\end{algorithmic}
\algrulebottom{}


    The convex hull can be computed according to [13, Thm. 5] and the multiplication with an intervalmatrix $\mathbf{M}\mathcal{CZ}$ follows from (16).
}{
    \location{Top of the page} \\
    \algruletop{}
\textbf{Algorithm 1} ~ Conversion: Polytope to constrained zonotope. \\
\textbf{Input:} Polytope $\mathcal{P} = \langle H, d \rangle_P$ \\
\textbf{Output:} Constrained zonotope $\mathcal{CZ} = \langle c, G, K, l \rangle_{CZ}$

\begin{algorithmic}[1]
	\State $\langle c, G \rangle_Z \gets \textsc{box}( \mathcal{P} )$
    \State $\forall j \in \mathbb{N}_{[1,...,h]}\colon o_{(j)} \gets -\rho\big( \langle c, G \rangle_Z,-H_{(j,\cdot)}^\top \big)$
    \State $G \gets [G \;\, \mathbf{0}]$, $K \gets [H G \;\, \tfrac{1}{2}\mathrm{diag}(o - d)]$, $l \gets \tfrac{1}{2}(d+o) - Hc$
    \State $\mathcal{CZ} \gets \langle c, G, K, l \rangle_{CZ}$
\end{algorithmic}
\algrulebottom{}

    
    (...) \\
    The exact conversion from a polytope to a constrained zonotopes, denoted by $\textsc{CZ}(\mathcal{P})$, is computed using Algorithm 1, which implements [12, Thm. 1].
    The convex hull can be computed according to [13, Thm. 5] and the multiplication with an intervalmatrix $\mathbf{M}\mathcal{CZ}$ follows from (16).
}

\noindent Placing non-text elements at the top of the page or column offers several advantages:
First, these elements are easily accessible when referenced in the text.
Second, non-text elements often condense important information, warranting prominent placement, such as at the top of the page.
Third, this placement preserves the flow of the running text, contributing to optimizing space usage.
