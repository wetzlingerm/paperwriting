
\guideline[g:nontext:pseudocode_explain_operators]
    {Pseudocode: Ensure that all operators are explained or referenced.}

\goodexample[Adapted from {\cite[Sec.~2]{Wetzlinger2021HSCC}}]{
    The operations $\highlightpartmath{\textsc{center}(\mathcal{S})}$, $\textsc{box}(\mathcal{S})$, and $\textsc{vol}(\mathcal{S})$ return the \highlightpart{geometric center}, the smallest box over-approximation, and the volume of a set $\mathcal{S} \subset \mathbb{R}^n$, respectively. (...) \\
    (...) \\
    At the start of each step $k$ (Line 4), \highlightpart{the operation $\textsc{taylor}$ evaluates the Taylor terms of the nonlinear dynamics (4) at the linearization point $z^*$}. (...) \\

    \textbf{Algorithm 1} ~ Reachability analysis of nonlinear systems using state-space abstraction. \\
\textbf{Input:} Nonlinear system $f(z)$, initial set $\mathcal{R}(t_0) = \mathcal{X}_0$, input set $\mathcal{U}$, time horizon $t_{\text{end}}$

\textbf{Output:} Outer approximation of the reachable set $\mathcal{R}([0,t_{\text{end}}])$

\begin{algorithmic}[1]
	\State $k \gets 0, t_k \gets 0$
	\State \textbf{while} $t_k < t_{\text{end}}$ \textbf{do}
	\State $\quad$ $z^*(t_k) \gets \highlightpartmath{\textsc{center}(\mathcal{R}(t_k))}$
	\State $\quad$ $w, A, B, D, \ldots \gets \highlightpartmath{\textsc{taylor}\big(f(z), z^*(t_k)\big)}$
	\State (...)
\end{algorithmic}
}

\noindent Since pseudocode aims to be concise, it's often better to use named operations rather than detailing every computation, especially for standard tasks, e.g., sorting.
However, it is still good practice to briefly explain what each operation does, particularly in terms of its input and output arguments.
