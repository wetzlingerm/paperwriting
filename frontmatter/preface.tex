\chapter*{Preface}

Publishing papers is essential to pursuing a doctoral degree.
While conducting research is an art, where creativity and intuition lead to uncovering new insights, paper writing is a craft, demanding a methodical approach to presenting those insights clearly.

I was very fortunate to have had both a PhD supervisor and a mentor that invested time and efforts in teaching me this craft.
Besides writing papers myself, I have also proofread numerous drafts of my junior colleagues, and I have reviewed many submissions to conferences and journals.
I consider the feedback loop of writing and reviewing to be the best teacher:
Reading and reviewing papers provides examples of both good and bad, whereas the actual learning comes from being forced to make decisions in one's own work, and having gone through this process may in turn change one's perspective on how others made their decisions.

So after more than five years, I humbly claim to have understood a thing or two about the craft of paper writing.
Of course, it would be presumptuous to claim there were one way of structuring content, outlining concepts, or explaining details.
Still, these tasks often raise similar questions, soliciting similar advice.
I have thus decided to comprise my knowledge and observations in this booklet, hoping that others may find it helpful.
The word \emph{guidelines} in the title is carefully chosen, as they may guide, not impose.
Some parts may be more contentious than others, but I believe any such discussion is ultimately in the reader's best interest as an author.

A few brief words on the content:
Early chapters focus more on high-level concepts, such as structure and how to manage different levels of depth, while later chapters list specific guidelines for specific questions.
% todo: be more specific once the structure is decided
The structure of this booklet is deliberately minute to allow for feeding specific parts to AI tools so that some changes can be automated.
Please also note that the scope is limited to computer science, as I am convinced it is better to stick with what I know best.
I do hope, however, that much of the content is transferable to other research disciplines as well.

\bigskip

\noindent Thank you for reading, \\
\noindent Mark.
