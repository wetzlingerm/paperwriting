\chapter*{Preface}

Publishing papers is the main task in pursuing a doctoral degree.
This task can be split in two stages: conducting research and writing papers.
The difference between the two stages lies in the qualities required for success.
Conducting research is more akin to art, with an emphasis on creativity and ingenuity to generate novel ideas and tackle unsolved problems.
In contrast, paper writing is a craft, with a focus on technical proficiency, attention to detail, and precision in execution.

I was very fortunate to have had both a PhD supervisor and a mentor that invested time and efforts in teaching me this craft.
Besides writing papers myself, I have also proofread numerous drafts of my junior colleagues, and I have reviewed many submissions to conference proceedings and journals.
The feedback loop between writing your own work and reviewing the work of others is essential for developing good writing skills.
Reading and reviewing papers allows you to consciously reflect on what makes a paper appealing to the reader.
Still, no skill can be attained without sufficient hands-on practice.
Hence, the true learning happens when you are forced to make decisions in your own work.

After more than five years of making such decisions, I believe to have understood a thing or two about the craft of paper writing.
Of course, it would be presumptuous to claim there were one way of writing a literature review, presenting a novel approach, or discussing numerical results.
Yet, one can discuss the how-to on an abstract level and provide generic advice applicable in many cases.
I have gained this conviction through repeatedly making the same suggestions for improvement during review.
And so, I have decided to comprise my knowledge and observations in this booklet.
The word ``guidelines'' in the title is carefully chosen, as there is no one truth but only a path towards it.
Some guidelines may be more contentious than others, but I believe any such discussion on the authors' side is ultimately in the best interest of the readers.

A few brief words on the content and its presentation:
The booklet is split into two parts---the first part offers guidelines on the level of sections and paragraphs,
while the second part addresses questions on the level of individual sentences and words.
For each guideline, we first provide a concise formulation, then present an example, and finally offer a brief explanation.
Examples are either a comparison between a preferable version and its disadvantageous counterpart or only show a possible implementation of the guideline.
The key parts in the examples are highlighted, unless there is no identifiable key part, usually because the example as a whole illustrates the guideline.
Furthermore, all examples are taken and/or adapted from my own papers;
where I could not find a fitting example, I have taken the liberty of constructing an example.

Let me conclude by restricting the scope of this booklet.
We specifically target the writing of regular papers, which explicitly excludes survey papers, tool papers, and benchmark papers, among others.
Still, I am convinced that many guidelines are applicable to those other types of papers, too.
Moreover, the scope is conservatively limited to computer science, as each research discipline has its own particularities in presenting their work.
I do hope, however, that much of the content is transferable to other research disciplines as well.

\bigskip

\noindent Thank you for reading, \\
\noindent Mark
