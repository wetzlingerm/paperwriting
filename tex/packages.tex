
\usepackage[utf8]{inputenc}
\usepackage[T1]{fontenc}
\usepackage[english]{babel}

% set font to Linux Libertine (non-proprietary font):
% this requires downloading the font, e.g., from
%    https://sourceforge.net/projects/linuxlibertine/
% and installing it manually
% then, compile using xelatex paperwriting.tex
% \usepackage{fontspec}
% \setmainfont{Linux Libertine O}
% alternative for pdflatex (does not work with certain font variants)
\usepackage{libertine}

%\usepackage[pass]{geometry}  % to make a5paper work
\usepackage[a5paper,bottom=3cm]{geometry}

\usepackage{setspace}
\setstretch{1.2}  % default value: 1.44

\usepackage{mathtools,amsfonts}  % for math (loads amsmath, amsfonts(?), amssymb)

\usepackage{pifont}  % for checkmark, cross
\usepackage{siunitx}  % for SI units guideline

\usepackage{xcolor}  % customized colors
\usepackage[most]{tcolorbox}  % for guidelines and examples
\tcbuselibrary{skins}  % enhanced formatting for tcolorbox
\tcbuselibrary{raster}  % for good/bad examples

\usepackage{longtable}  % for table of references for guidelines

\usepackage{soul}  % for highlighting multiline text

\usepackage{fancyhdr}  % for formatting headers
\pagestyle{fancy}  % enable custom header functionality
\renewcommand{\headrulewidth}{0pt}  % no horizontal line below header

\fancyhead[L]{\thepage}  % left header: page number
\fancyhead[C]{}
\fancyhead[R]{\nouppercase{\leftmark}}  % right header: chapter number and name
\fancyfoot[L]{}
\fancyfoot[C]{}
\fancyfoot[R]{}

% glossary / list of terms
\usepackage[nopostdot]{glossaries}
\newglossaryentry{emphasis}{name = {emphasis}, description={}}
%% \setglossarystyle{list}
%% \makeglossaries


\usepackage{csquotes}  % required by compilation?
\usepackage[backend=biber,
    url=false,
    doi=true,
    isbn=true,
    style=ext-numeric-comp,
    defernumbers=true,
    %giveninits,
    sorting=none,
    maxnames=100]{biblatex}
\addbibresource{tex/references.bib}

\DeclareFieldFormat{titlecase:title}{\MakeSentenceCase*{#1}}
\DeclareFieldFormat{titlecase:booktitle}{#1}
\DeclareFieldFormat{titlecase:journaltitle}{#1}
\DeclareFieldFormat{doi}{doi: \text{#1}}
