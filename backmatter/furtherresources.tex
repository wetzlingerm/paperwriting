\chapter*{Further Resources}

The resources in this chapter help you deepen your understanding of scientific paper writing.
Below, you find websites, research papers, and video talks---each accompanied by a short summary of its content, making it easier to quickly identify what might be most relevant to your needs.

\smallskip

\parwithcite{Stachniss2024,Milford2024}{
    The two talks below focused on effective paper structure, clear contributions, reader engagement, and high-level strategies like anticipating reader questions.
    As a key takeaway, consider pair writing with an experienced researcher to accelerate learning far more than remote feedback.
}

\parwithcite{Cormode2008}{
    Tongue in cheek, the next paper surveys the techniques of a mean-spirited minimal-effort reviewer.
    Despite being directed more toward improving review quality, you may extract hints between the lines about what to look out for when writing papers in order to minimize the attack surface for criticism.
}

\parwithcite{Adorno2022}{
    In similar style to this work, the guide below contains many helpful guidelines for academic writing.
}

\parwithcite{PueschelTables}{
    The following presentation slides offer tips and tricks for formatting tables, some of which are included in Chapter~\ref{ch:nontextelements}.
}

\parwithcite{Rougier2014}{
    The paper below addresses the formatting of figures, outlining a set of rules to enhance visual design and addressing common pitfalls.
}

\parwithcite{Hespanha2006}{
    The next resource traverses the individual sections of a paper---similar to Part~\ref{pt:highlevel} in this work---focussing on high-level organization and intent of each paper section.
    Also, you will find some detailed guidelines about wording patterns, \LaTeX{} environment usage, paragraph purposes, and common phrasing mistakes.
}

\parwithcite{ManchesterPhrasebank}{
    For all those looking for help when editing individual sentences, the resource below contains a multitude of useful context-independent phrases to improve your writing, especially targeted at non-native speakers.
}
