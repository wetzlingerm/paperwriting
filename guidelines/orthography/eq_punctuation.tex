
\guideline{Treat expressions in equation environments as nouns and use punctuation accordingly.}

\goodexample{
    For the integration of the right-hand side of (1) over a domain $\mathcal{X} \subset \mathbb{R}^{n}$, we use a Taylor expansion around the linearization point $x^* \in \mathbb{R}^{n}$,
    %
    \begin{equation*}
        \forall x \in \mathcal{X}\colon f(x) = f(x) \big|_{x=x^*} + D f(x) \big|_{x=x^*} (x - x^*) + l(x) \highlightpartmath{\text{,}}
    \end{equation*}
    where $l(x) \in \mathcal{L}(\mathcal{X})$ is a vector within the Lagrange remainder $\mathcal{X}(\mathcal{X})$ defined component-wise by [1, Eq.~(2)]
    %
    \begin{align*}
        \mathcal{L}_{(i)}(\mathcal{X}) \coloneqq \big\{ &\tfrac{1}{2} (x - x^*)^\top D^2 f_{(i)}(\tilde{x}) \big|_{\tilde{x}=\zeta} (x - x^*) \, \big| \\
        &\hspace{0pt} x \in \mathcal{X}, \zeta \in \{ x^* + \alpha(x-x^*) \, | \, \alpha \in [0,1] \}\!\big\}
        \highlightpartmath{\text{.}}
    \end{align*}
}

\noindent In the example above, the clause ``where $l(x) \in \mathcal{L}(\mathcal{X})$ (...)'' is preceeded by a comma since it provides an explanation.
The sentence is then properly ended by a full stop after the definition of $\mathcal{L}_{(i)}{\mathcal{X}}$.
In principle, equation environments are mere formatting, and the sentence should be punctuated in the same way as if it was written completely inline.
Please note the publisher rules apply; the editing process may change some punctuation.
