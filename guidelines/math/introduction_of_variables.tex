
\guideline{Introduce variables with a name and a domain.}

\goodbadexample{
    Given an \highlightpart{initial state} \highlightpart{$x_0 = x(t_0)$}, and an \highlightpart{input trajectory} \highlightpart{$u$}, we denote the solution at time $t \geq t_0$ of a system of the form (1) by $\xi(t; x_0, u(\cdot))$.
}{
    Given an \highlightpart{initial state} $x_0$ \highlightpart{$\in \mathbb{R}^n$} at the \highlightpart{initial time} $t_0$ \highlightpart{$\in \mathbb{R}$}, and an \highlightpart{input trajectory} $u$\highlightpart{$\colon \mathbb{R} \to \mathbb{R}^m$}, we denote the solution at time $t \geq t_0$ of a system of the form (1) by $\xi(t; x_0, u(\cdot))$.
}

\noindent Explicitly mentioning the name and the domain is helpful because it prevents misunderstandings, especially for readers who may not be familiar with certain conventions or who may follow different ones.
The introduction of variables including a domain is of particular importance in definitions.
New variables in non-inlined equations should be introduced before or immediately after the equation environment.
