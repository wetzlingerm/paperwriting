
\guideline{Avoid using different fonts or underlines to emphasize certain words.}

\goodbadexample{
    Recent work presented promising approaches to seamlessly incorporate task-level preferences [1--5], answering the question ``\highlightpart{\textbf{What} \textit{should the robot do?}}''.
    % However, to accurately reflect motion and control preferences, addressing the question ``\highlightpart{\textbf{How} \textit{should the robot execute its task?}}'', current approaches require human feedback in demonstrations [6-9].
}{
    Recent work presented promising approaches to seamlessly incorporate task-level preferences [1--5], answering the question ``\highlightpart{What should the robot do?}''.
    % However, to accurately reflect motion and control preferences, addressing the question ``\highlightpart{How should the robot execute its task?}'', current approaches require human feedback in demonstrations [6-9].
}

\noindent Ideally, the stress of the sentence follows naturally from its construction.
In the example above, highlighting the interrogative pronoun/adverb is unnecessary, as the sentences are already constructed in a way that avoids any potential misunderstandings.
Also, it is generally not necessary to use italics for text within quotation marks.
