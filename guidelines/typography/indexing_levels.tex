
\guideline{Avoid using multiple hierarchies of indices.}

\goodbadexample{
    (...) to ensure that the approximation of the support function is exact for the current candidate policy $\pi_{(\cdot)}^{(j)}$, i.e.,
    \begin{equation*}
        \tilde{\rho} (\ell)
        =
        \rho_{
            \text{\highlightpart{$\Delta \widehat{\mathcal{R}}(k,\bar{u}_{(\cdot)}^{(j)},K_{(\cdot)}^{(j)})$}}
        } (\ell)
        .
    \end{equation*}
}{
    (...) to ensure that the approximation of the support function is exact for the current candidate policy $\pi_{(\cdot)}^{(j)}$, i.e.,
    \begin{equation*}
        \tilde{\rho} (\ell)
        =
        \rho \highlightpartmath{\Big( \Delta \widehat{\mathcal{R}}\big(k,\bar{u}_{(\cdot)}^{(j)},K_{(\cdot)}^{(j)}\big), \ell \Big)}.
    \end{equation*}
}

\noindent Multiple levels of indices in equations can become hard to read because they create visual clutter.
To improve readability, functions or variables with complex indices can be redefined using simpler notations or intermediate variables.

