
\guideline{Use em dashes with proper spacing for breaks in thought.}

\goodbadexample{
    Alternatively, one can use the Hamilton-Jacobi reachability framework to compute the backward reachable set starting from the unsafe sets\highlightpart{ - }a state is safe for all possible actions if it is outside of the backward reachable set [1].
}{
    Alternatively, one can use the Hamilton-Jacobi reachability framework to compute the backward reachable set starting from the unsafe sets\highlightpart{---}a state is safe for all possible actions if it is outside of the backward reachable set [1].
}

\noindent There are two types of dashes: en dashes (--) and em dashes (---).
Their names originate from the relation of the length of the dash to the width of the letters ``N'' and ``M'', respectively; both are longer than the standard hyphen.
En dashes are used for ranges and connections, e.g., ``pages 1--10'' and ``North--South divide'', whereas em dashes are used for breaks in thought and emphasis, as shown in the example above.
Some venues explicity require British English, where it is more common to use commas or parentheses instead of em dashes.
