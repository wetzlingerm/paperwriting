
\guideline{Avoid language-specific functions in pseudocode.}

\goodbadexample{
    \textbf{Algorithm 1.} Random simulation. \\
    Input: Generator matrix $G$, (...) \\
    Output: (...) \\
    \begin{scriptsize}
        1:
    \end{scriptsize}
    $m \gets \highlightpartmath{\textsc{shape}(G,2)}$ \\
    (...)
}{
    \textbf{Algorithm 1.} Random simulation. \\
    Input: Generator matrix $G \in \mathbb{R}^{n \times \highlightpartmath{m}}$, (...) \\
    Output: (...) \\
    (...)
}

\noindent Using language-specific functions may hinder understanding across diverse audiences, as some readers may not be familiar with any given programming language.
In simple cases, there may be a more rigorous way to achieve the same goal, as shown in the example above.
If a language-specific function is particularly useful, its behavior can be briefly described in the running text.
