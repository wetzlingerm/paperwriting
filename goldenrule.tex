
\chapter*{The Golden Rule}
\label{ch:consistency}
\addcontentsline{toc}{chapter}{The Golden Rule}

To me, the single most important statement about scientific paper writing is the following:

\begin{tcolorbox}[
    title=The Golden Rule.,
    colback=color_guideline_background, colframe=color_guideline,
    sharp corners=northwest,bottomrule=0mm,rightrule=0mm,leftrule=0mm]
    \textbf{Be consistent.}
\end{tcolorbox}

\noindent
A few examples illustrate the importance of consistency.
One of the most important qualities of a scientific paper is the disciplined use of distinct words to convey distinct meanings.
When word usage is inconsistent, the boundaries between meanings become blurred as well.
It is therefore imperative to distinguish meanings as sharply as possible through the choice of different words.
The same principle applies to variables:
each variable should have a single, well-defined meaning and should not be reused for a different purpose elsewhere.
Likewise, establish a uniform presentation by formatting figures, tables, algorithms, and equations consistently with respect to typography, spacing, and coloring.

Ultimately, consistency establishes trust between the author and the reader.
It helps to match the reader's understanding with the author's intent.
Many guidelines in the remainder of this booklet are mere applications of the Golden Rule in specific contexts.
