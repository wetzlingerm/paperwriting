
\chapter*{The Golden Rule}
\label{ch:consistency}
\addcontentsline{toc}{chapter}{The Golden Rule}

The title of this booklet explicitly mentions guidelines, which are flexible and context-dependent.
So why start with something as rigid as a rule?

Paper writing consists of making decisions about individual parts that ultimately compose a whole.
Each paper has its own needs:
Some report theoretical findings, others focus on the experimental evaluation;
some propose novel algorithms, others combine existing approaches to harness the respective advantages.
Therefore, generic statements about paper writing must be flexible enough to accommodate these different contexts.
Unfortunately, the increased generality correlates with an irrevocable loss in meaning.
Statements like \emph{'think about the story you want to tell'} or \emph{'clearly state your contribution'} are as true as they are useless---they are condensed to the point of being meaningless, and the original intent is irretrievable.

There is, however, one statement that defies the above analysis, being remarkably short but completely unequivocal.
It is applicable to works of all formats and topics, and adhering to it will always improve the work:

\begin{tcolorbox}[
    title=The Golden Rule.,
    colback=color_guideline_background, colframe=color_guideline,
    sharp corners=northwest,bottomrule=0mm,rightrule=0mm,leftrule=0mm]
    \textbf{Be consistent.}
\end{tcolorbox}

\noindent Personally, I consider the Golden Rule the single most important statement about scientific paper writing.

In general, consistency is established through a one-to-one correspondence between entities:
On a structural level, the title of each section must correspond to its content.
A section entitled \emph{discussion} provides a critical assessment of the presented work, potentially in comparison to related work; thus, no such assessments or comparisons should be made prior to or after this section.
On a wording level, one should use scientific terms consistently and avoid using different terms to refer to the same entity.
If the term \emph{model} refers to a specific mathematical formulation for a real-world phenomenon, then any such phenomenon should always be described using the term \emph{model}, and not by alternatives like \emph{system}.
On a mathematical level, variables should be used consistently throughout the paper to facilitate switching back and forth between different sections, such as the problem statement and the evaluation section.
More precisely, a variable like $x$ should consistently have the same meaning, and anything else than that must not be called $x$.

Ultimately, consistency establishes a level of trust between the author and the reader.
It helps in avoiding misunderstandings between the author's intent and the reader's understanding.
Many guidelines in the remainder of this booklet are mere applications of the Golden Rule in specific contexts.

% TODO: Maybe silver rule for "don't assume anything"